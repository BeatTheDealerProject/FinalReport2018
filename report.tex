% プロジェクト学習中間報告書書式テンプレート ver.1.0 (iso-2022-jp)

% 両面印刷する場合は `openany' を削除する
\documentclass[11pt,a4paper,oneside]{jsbook}

\usepackage{bm}
\usepackage{float}
\usepackage{amsmath}
\usepackage{amsfonts}
\usepackage{amssymb}
\usepackage{blindtext}
\usepackage{here}
\usepackage[sectionbib]{chapterbib}
\usepackage[dvipdfmx]{graphicx}
\usepackage{funpro}
\usepackage{listings}
\usepackage{multirow}

\lstset{
 	language = Python,
 	%backgroundcolor={\color[white]},
 	breaklines = true,
 	breakindent = 10pt,
 	basicstyle = \ttfamily\scriptsize,
 	commentstyle = {\itshape \color[cmyk]{1,0.4,1,0}},
 	classoffset = 0,
 	keywordstyle = {\bfseries \color[cmyk]{0,1,0,0}},
 	stringstyle = {\ttfamily \color[rgb]{0,0,1}},
 	frame = TBrl,
 	framesep = 5pt,
 	numbers = left,
 	stepnumber = 1,
 	numberstyle = \tiny,
 	tabsize = 4,
 	captionpos = t
}

\thisYear{2018}
\jProjectName{ディーラーをやっつけろ! 複雑系の数理とシミュレーション}
\eProjectName{Beat the dealer! Mathematics of Complex Systems and Simulation.}
\ProjectNumber{3}
\jGroupName{グループ~1}
\eGroupName{Group~1}
\ProjectLeader{1014207}{菱田美紗紀}{Misaki~Hishida}
\GroupLeader  {1016123}{薩田凱斗}{Kaito~Satta}
\SumOfMembers{9}
\GroupMember  {1}{1016007}{柏田輝}{Hikaru~Kashiwada}
\GroupMember  {2}{1016042}{尾崎拓海}{Takumi~Ozaki}
\GroupMember  {3}{1016078}{伊藤晋之介}{Shinnosuke~Ito}
\GroupMember  {4}{1016087}{轟木文弥}{Fumiya~Todoroki}
\GroupMember  {5}{1016118}{葛西隼人}{Hayato~Kasai}
\GroupMember  {6}{1016175}{柿崎大輝}{Daiki~Kakizaki}
\GroupMember  {7}{1016184}{鳥谷航大}{Koudai~Toriya}
\GroupMember  {8}{1016207}{渡邊凛}{Rin~Watanabe}
\GroupMember  {9}{1016231}{米村祥裕}{Yoshihiro~Yonemura}
\jadvisor{川越敏司,川口聡,斎藤朝輝}
\eadvisor{Toshiji~Kawagoe,Satoshi~Kawaguchi,Asaki~Saitou}
\jdate{2019年1月16日}
\edate{January~16, 2019}

\renewcommand{\thesection}{\arabic{chapter}.\arabic{section}}%

% 画像ファイル (EPS, EPDF, PNG) を読み込むために
\usepackage[dvipdfmx]{graphicx,color}

\begin{document}
\maketitle
\pagenumbering{roman}
\fontsize{10}{18}\selectfont
%前付け
\frontmatter

% 和文概要
\begin{jabstract} 
\ \ 本プロジェクトでは、カジノにおいて最もポピュラーなゲームの一つであるブラックジャックを取り扱っている。ブラックジャックに関する代表的な行動戦略としてベーシックストラテジー、カウンティングという戦略が存在する。プレイヤーはこれらの戦略を用いることでブラックジャックの期待利得を増やすことができた。しかしカジノ側がカウンティングへの対策をしたことで、ブラックジャックにおいてカウンティングを使用することができなくなり、プレイヤーが利得を得ることは難しくなった。そこで我々は既存の戦略と比較し、以下の2つの観点から優れている戦略を新たに見つけ出すことをプロジェクトの目標とした。1つ目の観点はプレイヤーにとって扱いやすく、実行可能な戦略であること、2つ目は最終的な利得が他の戦略より大きくできることである。

\ \ 前期では、我々はベーシックストラテジーについてどの程度確かな戦略なのか、他の戦略に比べてどの程度プレイヤーに有利な戦略なのかを検証した。検証ではブラックジャックのシミュレータを作成しシミュレーションを行うことで、各戦略の勝率を調べた。さらに戦略の扱いやすさを判断するために、我々は戦略の複雑性を新しく定義した。そして戦略の勝率と複雑性を合わせた性能指標から戦略の評価を行なった。検証の結果から、複雑性を考慮した場合にはベーシックストラテジーには改善の余地があることがわかった。

\ \ 後期では前期に定義した複雑性の検証、遺伝的アルゴリズムを用いた新たな戦略の探索、シミュレータの改良を行った。複雑性の検証では実際に戦略を覚えてもらう実験を行った。実験の結果を分析したところ、複雑性が正しい指標であることが確認できた。遺伝的アルゴリズムでは複雑性を含めた性能指標を用いて戦略を探索し、優秀な戦略を得ることができた。シミュレータには賭け金の仕組みを導入することで、勝率だけではなく利得を計算できるようになった。その後、遺伝的アルゴリズムから得られた戦略とベーシックストラテジーを比較するためシミュレーションを行った。シミュレーションでは扱いやすさを考慮するため、プレイヤーが戦略を覚え間違える度合いをエラー率として新たに定義し、戦略間の比較を行った。シミュレーションの結果から遺伝的アルゴリズムから得られた戦略がベーシックストラテジーよりも扱いやすさと利得の両方の点から優れていることが分かった。

% 和文キーワード
\begin{jkeyword}
カジノ,\ ブラックジャック,\ ベーシックストラテジー,\ 複雑性
\end{jkeyword}
\bunseki{伊藤晋之介}
\end{jabstract}

%英語の概要
\begin{eabstract}
\ \  In this project, we deal with Blackjack, one of the most popular games in the casino. There are Basic Strategy and Counting as representative behavior strategy concerning Blackjack. Players were able to increase the expected gains of Blackjack by using these strategies.  However, as the casino side took countermeasures, Counting could not be used in Blackjack, and making it difficult for players to get gains. Therefore, we have set the goal of the project to discover superior strategies from the following two points in comparison with existing strategies. The first point is that it is easy for players to play and usable strategy, and the second is that the final gain can be larger than other strategies.

\ \ In the previous term, we examined how accurate the basic strategy, and how much it is advantageous to players compared with other strategies. In the verification, we created a simulator of Blackjack and conducted a simulation to investigate the winning percentage of each strategy. In addition, to determine the easy to handle of the strategy, we newly defined the complexity of the strategy. And we evaluated the strategy from the evaluation index which combined the winning percentage and the complexity of the strategy. From the results of the verification, it was found that there is room for improvement in the basic strategy when considering the complexity.

\ \ In the latter term, we verified the complexity defined in the previous term, searched for new strategies using Genetic algorithm, and improved the simulator. In the verification of complexity, we conducted an experiment to actually remember the strategy. From analyzed the results of the experiment, so we confirmed that the complexity is the correct index. In the Genetic algorithm, we were able to explore strategies using performance index including complexity and obtain excellent strategies. By introducing a mechanism of betting on the simulator, it became possible to calculate not only the winning percentage but also the gain. Then conducted a simulation to compare the strategy obtained from Genetic algorithm with the basic strategy. In this simulation, in order to handle ease of handling, the degree to which the player misunderstands the strategy is newly defined as an error rate, and comparison between the strategies was made. Simulation results show that Genetic algorithm's strategy is superior to basic strategy in terms of both ease of handling and final gain.
% 英文キーワード

\begin{ekeyword}
casino, blackjack, basic strategies, complexity
\end{ekeyword}
\bunseki{伊藤晋之介}
\end{eabstract}

\tableofcontents
\newpage

\pagenumbering{arabic}
\mainmatter

\chapter{研究背景}

\section{ブラックジャックの戦略の歴史}
斎藤(1999)によれば、ブラックジャックの戦略については1950年にメリーランド州のとある米国陸軍の研究所に所属していたRoger、Nash、Baldwinらが研究したものが始まりであるといわれている。その後計算機の性能向上より、ブラックジャックのシミュレーションが容易になったことでさらに戦略の研究は進んでいった。
ブラックジャックには主に有名な戦略が2つ存在する。

1つ目がベーシックストラテジーと呼ばれる戦略だ。ベーシックストラテジーはディーラーのアップカードと自分の手札の状況によってプレイヤーが選択するべき最適な行動を決定するという戦略である。

2つ目はカウンティングと呼ばれる戦略だ。カウンティングはブラックジャックのゲーム中で既に使われたカードを記憶することで、プレイヤーが有利になるように戦略を決定していくというものである。

前期ではベーシックストラテジーについて調査、検証をした。後期ではカウンティングについて詳しく調査し、シミュレーションを行った。
なお、カウンティングはカジノ側に対策をされており、この戦略を使用していることがカジノ側に気づかれた場合、プレイヤーはカジノから退場させられることもある。
\bunseki{伊藤晋之介}
\input{background/rule.tex}
\input{background/basicstrategy.tex}
\section{カウンティングについて}
カウンティングとは、ブラックジャックにおいてゲーム内で使用されているデックの状態がプレイヤーにとって有利な状態か、
あるいは不利な状況であるかを見極めるための高等テクニックである。プレイヤーにとって有利な時には、リスクを最小限に抑え
つつ、より多くの金額を賭けることができる。また、カウンティングを使用することによってデック内に残っているカードの構成に
基づいて、プレイヤーは自身の行動選択を変更させることも可能である。Thorp(1962)が最初に発明して以来その
カウンティングの手法ついては数多くの種類が誕生してきたが、本プロジェクトではHigh-Low法とKO法と呼ばれる2種類のカウンティング
手法を使用してシミュレーションなどを行った。ここでは、そのHigh-Low法とKO法の2つについて説明する。
\bunseki{薩田凱斗}

\subsection{High-Low法}
まずはHigh-Low法について説明する。High-Low法の特徴はランニングカウントと同じ分のユニットを賭けるということである。ラン
ニングカウントとは、各カードに割り当てられた値の暫定的な合計値のことである。自分自身の手札の値以外にも、他のプレイヤーの
カードの値とディーラーのアップカードの値も含む。そして、ユニットとは自分が決めるベット額の単位のことである。一般的には、
ミニマムベットを基準にする場合が多い。例えば、あるカジノのブラックジャックのテーブルでのミニマムベットが5ドルだとすると、
ユニットも5ドルに設定する。また、ランニングカウントを複数デックで使用する場合には、トゥルーカウントというものを算出する必要がある。
トゥルーカウントは、ランニングカウントを残りデック数で割ったものである。例えば、ランニングカウントが10で残りデック数が5の時、
トゥルーカウントは10割る2で5となる。では、なぜトゥルーカウントを使用する必要があるのかを説明する。仮にランニングカウントが+5
で残りデック数が6の場合とランニングカウントが+5で残りデック数が1の場合は全く違った状況となる。前者はデックごとに1枚のハイカード
が残っているため、プレイヤーにとっては有利な状況とは言えない。後者は、残り1デック、つまり52枚のうち10とAが5枚残っておりプレイヤー
にとって有利な状況であることがわかる。High-Low法のカウント方法は以下の\ref{hlc}のようにカードの値が2から6の時にはランニングカウントをプラス1、7から9の時はランニングカウントを0、そして10、Aの時にはマイナス1をランニングカウントに加えていくというものである。ただし、Thorp(1962)がHigh-Low法を発表したときは使用されていなかった。そのため今回はトゥルーカウントを使用しないでシミュレーションを行った。また、High-Low法においては、初期ランニングカウントを0から開始する。
  \begin{table}[H]
    \centering
    \label{hlc}
    \caption{High-Low法でのカードカウント}
    \begin{tabular}{|c|c|c|c|c|c|c|c|c|c|} \hline
      2&3&4&5&6&7&8&9&10&A \\ \hline
      +1&+1&+1&+1&+1&0&0&0&-1&-1 \\ \hline
    \end{tabular}
  \end{table}
このランニングカウントがプラスの場合は、10やAなどのカードがデック内に残っているためプレイヤーにとって手札の合計が21となる
ブラックジャックとなる可能性が高く、非常に有利になるためベッド額を増やしていき、反対にランニングカウントがマイナスの時は
不利な状況なためベッド額を抑えるといったものである。また、以下の\ref{hlb}はHigh-Low法を使用した際のベットシステムである。
  \begin{table}[H]
    \centering
    \label{hlb}
    \caption{High-Low法でのベットシステム}
    \begin{tabular}{|c|c|c|c|c|c|c|c|c|c|c|} \hline
      ランニングカウント&1以下&2&3&4&5&6&7&8&9&10以上 \\ \hline
      ベット(ユニット)&1&2&3&4&5&6&7&8&9&10 \\ \hline
    \end{tabular}
  \end{table}
先述した通り、High-Low法はカウント値と同じユニットを賭けていく。
\bunseki{薩田凱斗}

\subsection{KO法}
次にKO法についてである。KO法はVancura(1998)が発表したカウンティング手法である。以下の\ref{koc}のようにカードの値が2から7の時にはランニングカウントをプラス1、8と9の時は
ランニングカウントを0、そして10、Aの時にはマイナス1をランニングカウントに加えていくというものである。High-Low法との唯一の
違いは7をローカードとして扱っているため、カウント値が0ではなく+1となっている。そのため、それぞれのカウント値を足し合わせると
+4となる。これはアンバランスシステムとも呼ばれ、High-Low法のようにトゥルーカウントを算出する必要がない。
  \begin{table}[H]
    \centering
    \label{koc}
    \caption{KO法でのカードカウント}
    \begin{tabular}{|c|c|c|c|c|c|c|c|c|c|} \hline
      2&3&4&5&6&7&8&9&10&A \\ \hline
      +1&+1&+1&+1&+1&+1&0&0&-1&-1 \\ \hline
    \end{tabular}
  \end{table}
  
KO法の大きな特徴としてはデック数に応じて、初期ランニングカウントを変えていくということである。カジノで使用されるデック数で
一般的な6デックを例にとると、初期ランニングカウントは-20である。仮に6デックの場合には合計で24枚のAがあり、そこから1デック分のA、
つまり4枚を抜いた数を初期ランニングカウントとして扱う。なので1デックの場合の初期ランニングカウントは0、同様に2デックの場合は-4、
6デックの場合は-20が初期ランニングカウントとなる。以下の\ref{kob}はKO法を使用した際のベットシステムである。

  \begin{table}[H]
    \centering
    \label{kob}
    \caption{KO法でのベットシステム}
    \begin{tabular}{|c|c|c|c|c|c|c|c|c|c|c|} \hline
      ランニングカウント&-5以下&-4&-3&-2&-1&0&+1&+2&+3&+4以上 \\ \hline
      ベット(ユニット)&1&2&2&3&4&5&6&8&9&10 \\ \hline
    \end{tabular}
  \end{table}
ランニングカウントが-4と-3の時にはユニットを2にし、それ以降はユニットを1つずつ上げていくが、ランニングカウントが+1から+2になるときはユニットを2つ増やす。
\bunseki{薩田凱斗}
\section{ベッティングシステムの検討}
カウンティング時の適切なベッティングシステムが存在しないため、カウンティングに適用可能な手法を調査した。その結果ベッティングシステムは以下のようなグループに分けることができた。
\begin{itemize}
\item 勝ち負けに応じて一定の倍率で賭け額を変化させる手法
\item 数列を用意し、数列を操作しながら賭け額を変化させる手法
\item 資金の何割かを賭ける手法
\item 勝率に応じて賭け額を変化させる手法
\item その他の手法
\end{itemize}
以上の5つについてこれから説明する。
\bunseki{※柏田輝}

\subsection{勝ち負けに応じて一定の倍率で賭け額を変化させる手法}
勝ち負けに応じて賭け額を変化させる手法には以下のような手法がある。
\begin{itemize}
\item マーチンゲール法
\item グランマーチンゲール法
\item パーレー法
\item グランパーレー法
\end{itemize}
\subsubsection{マーチンゲール法}
マーチンゲール法とは、基準となる賭け金(ユニット)を決めて、負けるたびにユニットを倍に増やしていき、勝った時にユニット数を1に戻すという手法である。例えば1ユニットを100と設定した場合、1度負けると100×2となり200、2度負けると200×2となり400となる。この手法の特徴としては、1度勝てば負け額をすべて取り戻し、1ユニット分だけ利益を出すことが出来る。しかし、負け続けると賭けるユニット数が指数関数的に増えてしまい、破産しやすいというデメリットや、カジノでは賭け金の上限が設定されているため、上限に達すると使用できないといった問題点がある。
\subsubsection{グランマーチンゲール法}
グランマーチンゲール法とは、負けるたびにユニットを2倍しその値に1を加えて勝つまでユニット数を増やしていき、勝った時にユニット数を1に戻す手法である。例えば1ユニットを100と設定した場合、1度負けると100×2+100となり300、2度負けると300×2+100となり700となる。この手法の特徴として、最初の1ユニット目を賭けてから勝った場合までのゲーム回数×1ユニット分の利益を出すことができる。つまり、マーチンゲール法をハイリスクハイリターンにした手法であるため、勝ち続けることができれば利益を出すことができるが、負け続けるとマーチンゲール法より早く賭け金の上限に達してしまったり、より破産しやすくなってしまう。
\subsubsection{パーレー法}
パーレー法とは、勝つたびにユニット数を2倍にする手法である。例えば、1ユニットを100と設定すると、1度勝つと100×2となり200、2度勝つと200×2で400となる。この手法の特徴として、連勝することで少ない賭け金で大きな利益を出すことができる。しかし、1度でも負けてしまうと利益がマイナスになってしまうのである程度利益をした後、自分でユニット数を1に戻す必要がある。
\subsubsection{グランパーレー法}
グランパーレー法とは、勝つたびにユニット数を2倍にしその値に1を加えていく手法である。例えば1ユニットを100と設定した場合の賭け金は、1度勝つと100×2+100となり300、2度勝つと300×2+100となり700となる。この手法の特徴として、勝ち続けると利益を指数関数的に増やすことができる。しかし、負けた場合に最初の1ユニット目を賭けてからその時までのゲーム回数×1ユニット分負けてしまう。つまり、パーレー法をハイリスクハイリターンにした手法である。
\bunseki{※柏田輝}

\subsection{数列を用意し、その数列を操作しながら賭け額を変化させる手法}
 数列を用意し、その数列を操作しながら賭け額を変化させる手法には以下のような手法がある。
\begin{itemize}
 \item 2in1法
 \item モンテカルロ法
 \item バーネット法
\end{itemize}
\subsubsection{2in1法}
2in1法とは、数列の両端を足した数のユニット数を賭ける手法である。この手法は2連敗した後に適用され、負ける度にその直前に賭けたユニット数を右端に記録していき、勝つ度に記録の両端の数字を1つずつ削除する。毎ゲーム時に賭けるユニット数は、勝ち負けに関わらず記録の両端の数字を合計したユニット数を賭ける。例えば、2連続で負けた後の数列は1,1となり、ここから適用する。両端数字が1,1なので次は2ユニット賭ける。ここで負けた場合数列が1,1,2となり、次は3ユニット賭けることになる。更ににここで負けた場合数列が1,1,2,3となり次は4ユニット賭けることになる。ここで勝つと数列の両端の数字である1,3を削除し数列が1,2に変化するので、次は3ユニット賭けることになる。そこでまた勝つことにより、数列の両端の数字がなくなり、負けた分を全て回収できたことになる。この手法の特徴は、2回分の負け額を1度の賭けで回収のすることができる。また、マーチンゲール法ほど賭け額が増えないので破産率が低いのも特徴である。しかし、この手法ではマーチンゲール法のように1度勝っただけでは利益が増えない。つまり、連勝しなければ利益を得ることができない手法となっている。また実際のカジノでは、メモを見たりメモを取ることができないため、数列を暗記し、ややこしい計算を頭の中で行わなければならないといったデメリットもある。
\subsubsection{モンテカルロ法}
モンテカルロ法とは、これは最初の数列を1,2,3とし、数列の両端を足した数だけユニット数を賭ける手法である。負ける度に、数列に直前にその直前に賭けたユニット数を付け加え、勝つ度に配当が2倍のゲームでは記録の両端の数字を1つずつ削除し、配当が3倍のゲームでは両端の数字を2つずつ削除する。毎ゲーム時に賭けるユニット数は、勝ち負けに関わらず記録の両端の数字を合計したユニット数を賭ける。例えば、配当が2倍のゲームでは、1度目は1,2,3となっているので両端数字が1,3であり、4ユニット賭ける。この後負けた場合に、数列は1,2,3,4となり、両端数字が1,4となるので、次は5ユニット賭ける。ここで勝った場合は、両端数字の1,4を削除するので数列が2,3になるので次に賭けるユニット数は両端数字2,3なので5ユニット賭けることになる。ここで勝つと両端数字である。2,3が削除されるので、次に賭けるときは再び1,2,3の数列を使う。また配当が3倍のゲームでは、1度目は1,2,3となっているので両端数字が1,3であり、4ユニット賭ける。この後負けた場合に、数列は1,2,3,4となり、両端数字が1,4となるので、次は5ユニット賭ける。ここで負けた場合、数列が1,2,3,4,5となりり次に賭けるユニット数は両端数字が1,5なので6になる。ここで勝った場合、両端数字を2つずつ削除するので数列は3となり、数列の要素数が2個以下のため、次に賭けるときは再び1,2,3の数列を使う。特徴としては、2in1法と同じように、数列を利用する関係上カジノでは扱いにくい。また、この手法は配当が3倍のゲームでは儲かるが、2倍ゲームでは必ずしも利益が出て終わる手法ではないのでブラックジャックには向いていない。
\subsubsection{バーネット法}
バーネット法とは、賭けるユニット数を1,3,2,6のように変化させる手法である。最初に1ユニット賭け勝った場合に3,2,6のように賭けていく手法で、6ユニット賭けた後に勝った場合は、負けるまで6ユニットを賭け続け、負けた場合は数列の最初の1からまたかけ始めていく手法である。この手法の特徴として、連勝時には賭け金を上げて利益を得る。連敗時には賭け金を下げてリスクを減らすことができる。連勝や連敗に対しては非常に頼もしいが、勝ち負けが交互となったり、短いスパンで勝ち負けが同数で進行した場合は、効果を発揮することができない。
\bunseki{※柏田輝}

\subsection{資金の何割かを賭ける手法}
資金の何割かを賭ける手法には以下のような手法がある。
\begin{itemize}
\item 10%投資法
\item 全額投資法
\end{itemize}
\subsubsection{10\%投資法}
10\%投資法とは、自分がゲーム使える額全体のうちから1度のゲームにつき、その10\%を賭けるという手法である。この手法の特徴としては、絶対に破産しない上に計算が単純なので覚える必要がない。また、連敗すると賭け金が低くなるので、負けやすい間の不利益を抑えることができ、連勝すると賭け金が高くなるので、勝ちやすい間の利益が高くなる。しかし、今までのベッティング手法とは違い、負け額を確実に回収したり、利益を確実に出す手法ではないので、利益を得にくいといったデメリットがある。また、勝率が50\%を下回ってしまうと持ち金が0に収束していくというデメリットもある。
\subsubsection{全額投資法全額投資法}
とは、自分がゲーム使える額全体を1度のゲームですべて賭ける手法である。この手法の特徴としては、絶対に破産はしないが、1度負けると賭けることのできる金額を全て失うことになる。しかし、カジノでは基本的に長期的に勝負し続けると、少しずつプレイヤー側の利益がマイナスになることが多いのでその裏をかくことができる手法である。
\bunseki{※柏田輝}

\subsection{勝率に応じて賭け額を変化させる手法}
勝率に応じて賭け額を変化させる手法には以下のような手法がある。
\begin{itemize}
\item ケリー基準
\item ハーフケリー
\end{itemize}
\subsubsection{ケリー基準}
ケリー基準とは、以下のような公式から賭け額を決定する。
\begin{center}
(AP-Q)/A
\end{center}
Aは勝った場合に帰ってくる配当(デシマルオッズ)から1引いた値、つまり勝った場合の純粋な利益であり、Pは勝つ確率であり、Qは負ける確率である。これらを計算し出た値に1を足した値を賭け額のユニットとなる。例えば、基準となる賭け額が100、勝つ確率が52\%、負ける確率が48\%、デシマルオッズが2だった場合は、(1*0.52-0.48)/1=0.04となるので、賭け額は104となる。
\subsubsection{ハーフケリー}
ハーフケリーとは、以下のような公式をから賭け額を決定する。
\begin{center}
(AP-Q)/A/2
\end{center}
つまりケリー基準の出た値を半分にし、賭け額を決定する手法である。例えば、基準となる賭け額が100、勝つ確率が52\%、負ける確率が48\%、デシマルオッズが2だった場合は、(1*0.52-0.48)/1=0.04となり、その値の半分は0.02で、賭け額は102となる。つまり、ハーフケリーはケリー基準をローリスクローリターンにした手法であるといえる。これらの手法の特徴としては、他のベッティングシステムとは異なり勝率を参照するので、カウンティングを併用しやすいといったメリットがある。一方で、毎回勝率を計算しなければならないといったデメリットがある。
\bunseki{※柏田輝}

\subsection{その他の手法}
その他の方法には以下のような手法がある。
\begin{itemize}
\item ダランベール法
\item 逆ダランベール法
\item 31システム
\item ココモ法
\subsubsection{ダランベール法}
ダランベール法とは、勝負に負けたときにユニットを1増やし、勝ったときに1減らすという手法である。例えば、1ユニット100と設定した場合、1度負ければ100+100となり200、2度負ければ200+100となり300、1度勝てば300-100となり200となる。この手法は見た目の通りとても単純であり覚えやすいという特徴がある。また、勝っているときでも負けているときでも、利得の増減がゆるやかである。連敗しても長い目で見ればある程度マイナスになった分の利得を取り戻しやすいというメリットがある。しかし、勝ちと負けがが交互に繰り返された場合には効果がみられないというデメリットがある。この手法は、勝率が約50%であり、勝利した際の払い戻しが賭け金の2倍であるゲームで有効である。つまりブラックジャックで有効である。
\subsubsection{逆ダランベール法}
逆ダランベール法とは、勝負に勝ったときにユニットを1増やし、負けたときに1減らすという、ダランベール法と真逆の手法である。例えば、1ユニットを100と設定した場合、1度勝てば100+100となり200、2度勝てば200+100となり300、1度負ければ300-100となり200となる。ダランベール法と同じで、とても単純であり覚えやすいという特徴がある。また、連勝すれば急激に利得が増えるというメリットがある。逆に、連勝しなければ大きな利得を得ることができず、ダランベール法と同じく、勝ちと負けが繰り返された場合では利得がでにくいというデメリットがある。つまり、大勝ちはできないが逆に大負けもしなく、連敗してもダランベール法のように賭け金が膨大にならず、賭け金の上限に引っかかることもない。
\subsubsection{31システム}
21システムとは、負けた時にユニットを1,1,1,2,2,4,4,8,8の順番で増やし、勝ったときには現在の賭け金を2倍にし、2連勝したときに最初のユニットに戻る、というカウンティング手法である。また8連敗した場合も最初のユニットに戻る。これらの数字を全て足すと31になることから、31システムと呼ばれる。例えば、1ユニットを100と設定した場合、はじめの賭け金は100、1回目の勝負で負けた場合も100、2回目の勝負で負けた場合も100、3回目に勝利した場合には200となる。このカウンティング手法を使用し賭けていき、8敗したとしても、トータルの損失が31ユニットしかないため、資金管理がしやすく、また仕組み的に、2連勝すればどこのタイミングでも利益が出せるというメリットがある。この手法に似ているカウンティング手法としてマーチンゲール法があるが、そちらに比べて賭け金の増え方が緩やかであることに加え、賭け金の上限に引っかかる可能性も低い。
\subsubsection{ココモ法}
ココモ法とは、3回目以降の賭け金を、前回の賭け金+前々回の賭け金で導出するカウンティング手法である。1回目と2回目の賭け金は1ユニットである。例えば、1ユニットを100と設定した場合、1回目と2回目は100、3回目は100+100で200、4回目は200+100で300、5回目は300+200で500、6回目は500+300で800となる。例からわかる通り、ゲーム回数を重ねるにつれて、賭け金は指数関数的に増えてゆき、負けが連続しても一度勝利すればそれまでの損失分もしくはそれ以上の利益を得ることがメリットである。逆に賭け金の上限に引っかかった場合、破綻するというのがデメリットである。この手法に似ているカウンティング手法に、マーチンゲール法があるが、そちらに比べて賭け金の増え方は緩やかである。
この手法は、勝率が約30%であり、勝率した際の払い戻しが3倍のゲームで一番利得が出るようになっており、ブラックジャックのような勝率が約50%であり、勝利した際の払い戻しが2倍のゲームではあまり効果がみられないという特徴もある。
\bunseki{※菱田美紗紀}

\subsection{まとめ}
ベッティングシステムについて調査した結果、ベッティング手法は、勝ち負けに応じて賭け額を変化させる手法、数列を用意し、その数列を操作しながら賭け額を変化させる手法、資金の何割かを賭ける手法、勝率に応じて賭け額を変化させる手法、その他の手法の5種類のグループに分けることができた。そのうち、勝ち負けに応じて賭け額を変化させる手法と数列を用意し、その数列を操作しながら賭け額を変化させる手法、その他の手法は、カウンティング値を参照するのが最初の1度のみで、その後はカウンティング値が低くなっても、途中で賭け額を決めることができない。また、資金の何割かを賭ける手法についても、賭け額が資金によって決定されているのでカウンティング値を参照しない。これらの点から、勝率に応じて賭け額を変化させる手法がカウンティングに適用しやすい手法だと考えられる。
\bunseki{※柏田輝}


%既存戦略の検証に関する項目
\chapter{プロジェクトの目標}
%前期目標
%後期目標
本プロジェクトでは、次の3つの目標を設定した。
\begin{description}
\item[目標1] ブラックジャックの最終的な勝率が5割以上になること
\begin{itemize}
\item{後述するが、シミュレーションを行った結果、ブラックジャックでベーシックストラテジーを使用した場合の勝率は4割程度にしかならなかった。ブラックジャックにおける最終的な利得を増やすためには勝負に勝ち越す必要があると考え、勝率が5割以上になる戦略を新たに探索することを目標とした。}
\end{itemize}
\item[目標2] 人間にとって扱いやすい戦略であること
\begin{itemize}
\item{勝率が高い戦略を作り出すことができても、実際のカジノではコンピュータを使用することはできない。そしてカジノで実際に戦略を扱うのは人間である。そのため勝率だけではなく、人間にとっての戦略の扱いやすさという点でも優秀である戦略を見つけることとを目標とした。}
\end{itemize}
\item[目標3] 戦略を使用していることがカジノ側に検知されにくいこと
\begin{itemize}
\item{プレイヤーが戦略を使用していることがカジノ側に検知されてしまった場合、その戦略もまたカジノ側に対策されてしまう可能性がある。そのためカジノ側がどのような基準でプレイヤーが戦略を使用していることを検知しているのかを調べ、対策されないような戦略を作り出すことを目標とした。}
\end{itemize}
\end{description}
以上の3点を本プロジェクトの総合的な目標とし、それを達成するため前期と後期でそれぞれ活動目標を立てた。
\bunseki{伊藤晋之介}

\chapter{複雑性について}
\subsection{複雑性の定義について}

プレイヤーに扱いやすい戦略とは何かについて戦略の複雑性を定義し、複雑性の低い戦略をプレイヤーに扱いやすい
戦略であるとした。戦略の複雑性はChaitin(1969)によって定義されたコルモゴロフ複雑性を参考に新たに定義した。
コルモゴロフ複雑性とは、ある文字列があったときに、その文字列を生成するためのプログラムの内、最小の命令長を
その文字列の複雑性であると定義したものである。

新たに定義した複雑性を算出するためには、まず戦略の表を文字列として文字列を圧縮する。
圧縮の方法としては、元の表を「連続する文字+連続して文字が出た回数」と変換することで文字列を圧縮した。
例として、「HHSSSHHHHH」という10文字からなる文字列を圧縮すると、「H2S3H5」となり、圧縮した後の文字列は6文字
となる。また「HHHHHHHHHH」という文字列を「H10」と圧縮したときのように、連続して同じ文字が出た回数が2桁の場合には
数字部分を1文字として数える。つまり「H10」の文字列長は2文字である。

圧縮した後の文字列長を元の表の大きさ(80)で割った数値を複雑性の定義とした。
\bunseki{※米村祥裕}
\section{複雑性の検証実験}

\subsection{概要}

本プロジェクトの目標の要件を満たす戦略表を作成するには、人が扱いやすいということについて定量的に表し、既存、もしくは新たに作成した戦略表を評価しなければならない。そのため、前期ではコルモゴロフ複雑性を参考にし、連長圧縮を用いることで複雑性を定義し、これをある戦略表を記憶する難易度の評価指標として用いた。しかし、実際に人間がある戦略表を覚える時に感じる難易度と複雑性によって表した難易度は、異なっている可能性がある。そのため、後期では複雑性の検証実験として、いくつかの戦略表を用意し、記憶してもらうという実験を行うことで、複雑性が戦略表を記憶する難易度を正しく表しているかを確かめた。
\bunseki{※鳥谷航大}

\subsection{実験の目的}

本実験のもっとも重要な目的は、先述の通り、人が扱いやすい戦略表というのはどういう戦略表なのかを定量的に表すことである。そのため、本実験では、以下の2つを主な目的とした。
\begin{itemize}
    \item 目的1:本プロジェクトが定義した複雑性が戦略表を記憶する難易度を正しく表しているかを確認すること
    \item 目的2:目的1を達成出来なかった場合、可能な限り戦略表を記憶する難易度を正確に評価できる指標を見つけること
\end{itemize}
目的2について、具体的には、人のどのような能力がブラックジャックをプレイする能力と関連性を持つかや、個人の性格が実験に影響を及ぼすかなどの観点を中心に実験を行った。
\bunseki{※鳥谷航大}

\subsection{仮説}

私達は以下のような仮説を立て、実験を行った。
\begin{itemize}
    \item 仮説1:複雑性は人が戦略表を記憶する時に感じる難易度を正確に表すことが出来ない
    \item 仮説2:一般的な認知的判断能力と戦略表を扱う能力は同じ能力、もしくは深い関連性を持つ
    \item 仮説3:リスクを好む・好まない等の性格によって戦略表を記憶し扱う時に違いが生じる
\end{itemize}
それぞれの仮説に対して詳しく説明する。まず仮説1についてであるが、このような仮説を立てた背景として、前期でこの複雑性を用いて性能を評価した時、戦略表の勝率があまり良くない戦略表が高い性能と評価されてしまったことがある。このことから、他に戦略表を記憶する難易度を正確に表すことが出来る指標が存在するのではないか、と考えこのような仮説を立てた。そして、仮説2、仮説3は、先述の他の評価指標として考えられるものである。仮説2は、個人の一般的な認知判断能力が評価指標になるのではないかという仮説である。それに対して、仮説3は、個人のリスクに関する性格が評価指標になるのではないかという仮説である。
\bunseki{※鳥谷航大}

\subsection{実験準備}

今回の実験の目的や仮説に合わせ、実験に用いる戦略表や被験者、報酬などは以下のように準備を行った。
まず、今回の実験で用いた戦略表は以下の3つである。
% Please add the following required packages to your document preamble:

\begin{table}[H]
    \caption{戦略表A}
    \begin{center}
        \begin{tabular}{|c|c|c|c|c|c|c|c|c|c|c|c|}
        \hline
        \multicolumn{2}{|c|}{\multirow{2}{*}{}} & \multicolumn{10}{c|}{ディーラーのアップカード}     \\ \cline{3-12} 
        \multicolumn{2}{|c|}{}                  & 2 & 3 & 4 & 5 & 6 & 7 & 8 & 9 & 10 & A \\ \hline
        \multirow{10}{*}{手札の合計}      & 5〜8      & H & H & H & H & H & H & H & H & H  & H \\ \cline{2-12} 
                                    & 9        & H & H & H & H & H & H & H & H & H  & H \\ \cline{2-12} 
                                    & 10       & H & H & H & H & H & H & H & H & H  & H \\ \cline{2-12} 
                                    & 11       & H & H & H & H & H & H & H & H & H  & H \\ \cline{2-12} 
                                    & 12       & H & H & H & H & H & H & H & H & H  & H \\ \cline{2-12} 
                                    & 13       & H & H & H & H & H & H & H & H & H  & H \\ \cline{2-12} 
                                    & 14       & H & H & H & H & H & H & H & H & H  & H \\ \cline{2-12} 
                                    & 15       & H & H & H & H & H & H & H & H & H  & H \\ \cline{2-12} 
                                    & 16       & S & S & S & S & S & S & S & S & S  & S \\ \cline{2-12} 
                                    & 17以上     & S & S & S & S & S & S & S & S & S  & S \\ \hline
        \end{tabular}
    \end{center}
\end{table}

% Please add the following required packages to your document preamble:
% \usepackage{multirow}
\begin{table}[H]
    \begin{center}
    \caption{戦略表B}
    \begin{tabular}{|c|c|c|c|c|c|c|c|c|c|c|c|c|}
    \hline
    \multicolumn{3}{|c|}{\multirow{2}{*}{}}                     & \multicolumn{10}{c|}{ディーラーのアップカード}     \\ \cline{4-13} 
    \multicolumn{3}{|c|}{}                                      & 2 & 3 & 4 & 5 & 6 & 7 & 8 & 9 & 10 & A \\ \hline
    \multirow{28}{*}{手札の合計} & \multirow{9}{*}{ハードハンド}   & 9     & H & H & H & H & H & H & H & H & H  & H \\ \cline{3-13} 
                            &                           & 10    & H & D & D & D & D & H & H & H & H  & H \\ \cline{3-13} 
                            &                           & 11    & D & D & D & D & D & D & D & D & H  & H \\ \cline{3-13} 
                            &                           & 12    & D & D & D & D & D & D & D & D & D  & D \\ \cline{3-13} 
                            &                           & 13    & H & H & S & S & S & H & H & H & H  & H \\ \cline{3-13} 
                            &                           & 14    & S & S & S & S & S & H & H & H & H  & H \\ \cline{3-13} 
                            &                           & 15    & S & S & S & S & S & H & H & H & R  & H \\ \cline{3-13} 
                            &                           & 16    & S & S & S & S & S & H & H & R & R  & R \\ \cline{3-13} 
                            &                           & 17以上  & S & S & S & S & S & S & S & S & S  & S \\ \cline{2-13} 
                            & \multirow{9}{*}{ソフトハンド}   & A,2   & H & H & H & D & D & H & H & H & H  & H \\ \cline{3-13} 
                            &                           & A,3   & H & H & H & D & D & H & H & H & H  & H \\ \cline{3-13} 
                            &                           & A,4   & H & H & D & D & D & H & H & H & H  & H \\ \cline{3-13} 
                            &                           & A,5   & H & H & D & D & D & H & H & H & H  & H \\ \cline{3-13} 
                            &                           & A,6   & H & D & D & D & D & H & H & H & H  & H \\ \cline{3-13} 
                            &                           & A,7   & S & D & D & D & D & S & S & H & H  & H \\ \cline{3-13} 
                            &                           & A,8   & S & S & S & S & S & S & S & S & S  & S \\ \cline{3-13} 
                            &                           & A,9   & S & S & S & S & S & S & S & S & S  & S \\ \cline{3-13} 
                            &                           & A,10  & S & S & S & S & S & S & S & S & S  & S \\ \cline{2-13} 
                            & \multirow{10}{*}{スプリット可能} & A,A   & P & P & P & P & P & P & P & P & P  & P \\ \cline{3-13} 
                            &                           & 2,2   & P & P & P & P & P & P & H & H & H  & H \\ \cline{3-13} 
                            &                           & 3,3   & P & P & P & P & P & P & H & H & H  & H \\ \cline{3-13} 
                            &                           & 4,4   & H & H & H & P & P & H & H & H & H  & H \\ \cline{3-13} 
                            &                           & 5,5   & D & D & D & D & D & D & D & D & H  & H \\ \cline{3-13} 
                            &                           & 6,6   & P & P & P & P & P & H & H & H & H  & H \\ \cline{3-13} 
                            &                           & 7,7   & P & P & P & P & P & P & H & H & H  & H \\ \cline{3-13} 
                            &                           & 8,8   & P & P & P & P & P & P & P & P & P  & P \\ \cline{3-13} 
                            &                           & 9,9   & P & P & P & P & P & S & P & P & S  & S \\ \cline{3-13} 
                            &                           & 10,10 & S & S & S & S & S & S & S & S & S  & S \\ \hline
    \end{tabular}
    \end{center}
\end{table}

% Please add the following required packages to your document preamble:
% \usepackage{multirow}
\begin{table}[H]
    \begin{center}
    \caption{戦略表C}
    \begin{tabular}{|c|c|c|c|c|c|c|c|c|c|c|c|c|}
    \hline
    \multicolumn{3}{|c|}{\multirow{2}{*}{}}                     & \multicolumn{10}{c|}{ディーラーのアップカード}     \\ \cline{4-13} 
    \multicolumn{3}{|c|}{}                                      & 2 & 3 & 4 & 5 & 6 & 7 & 8 & 9 & 10 & A \\ \hline
    \multirow{28}{*}{手札の合計} & \multirow{9}{*}{ハードハンド}   & 9     & H & H & H & H & H & H & H & H & H  & H \\ \cline{3-13} 
                            &                           & 10    & H & D & D & D & D & H & H & H & H  & H \\ \cline{3-13} 
                            &                           & 11    & D & D & D & D & D & D & D & D & H  & H \\ \cline{3-13} 
                            &                           & 12    & D & D & D & D & D & D & D & D & D  & D \\ \cline{3-13} 
                            &                           & 13    & H & H & S & S & S & H & H & H & H  & H \\ \cline{3-13} 
                            &                           & 14    & S & S & S & S & S & H & H & H & H  & H \\ \cline{3-13} 
                            &                           & 15    & S & S & S & S & S & H & H & H & H  & H \\ \cline{3-13} 
                            &                           & 16    & S & S & S & S & S & H & H & H & H  & H \\ \cline{3-13} 
                            &                           & 17以上  & S & S & S & S & S & S & S & S & S  & S \\ \cline{2-13} 
                            & \multirow{9}{*}{ソフトハンド}   & A,2   & H & H & H & H & D & H & H & H & H  & H \\ \cline{3-13} 
                            &                           & A,3   & H & H & H & D & D & H & H & H & H  & H \\ \cline{3-13} 
                            &                           & A,4   & H & H & D & D & D & H & H & H & H  & H \\ \cline{3-13} 
                            &                           & A,5   & H & H & D & D & D & H & H & H & H  & H \\ \cline{3-13} 
                            &                           & A,6   & H & D & D & D & D & H & H & H & H  & H \\ \cline{3-13} 
                            &                           & A,7   & D & D & D & D & D & S & S & H & H  & H \\ \cline{3-13} 
                            &                           & A,8   & S & S & S & S & D & S & S & S & S  & S \\ \cline{3-13} 
                            &                           & A,9   & S & S & S & S & S & S & S & S & S  & S \\ \cline{3-13} 
                            &                           & A,10  & S & S & S & S & S & S & S & S & S  & S \\ \cline{2-13} 
                            & \multirow{10}{*}{スプリット可能} & A,A   & P & P & P & P & P & P & P & P & P  & P \\ \cline{3-13} 
                            &                           & 2,2   & H & H & P & P & P & P & H & H & H  & H \\ \cline{3-13} 
                            &                           & 3,3   & P & P & P & P & P & P & H & H & H  & H \\ \cline{3-13} 
                            &                           & 4,4   & H & H & H & P & P & H & H & H & H  & H \\ \cline{3-13} 
                            &                           & 5,5   & D & D & D & D & D & D & D & D & H  & H \\ \cline{3-13} 
                            &                           & 6,6   & P & P & P & P & P & H & H & H & H  & H \\ \cline{3-13} 
                            &                           & 7,7   & P & P & P & P & P & P & H & H & H  & H \\ \cline{3-13} 
                            &                           & 8,8   & P & P & P & P & P & P & P & P & P  & P \\ \cline{3-13} 
                            &                           & 9,9   & P & P & P & P & P & S & P & P & S  & S \\ \cline{3-13} 
                            &                           & 10,10 & S & S & S & S & S & S & S & S & S  & S \\ \hline
    \end{tabular}
    \end{center}
\end{table}
またこれらの戦略表の複雑性は以下の表のとおりである。
\begin{table}[H]
    \begin{center}
    \caption{各戦略表の複雑性}
    \begin{tabular}{|c|c|c|c|}
        \hline
            & 戦略表A & 戦略表B  & 戦略表B  \\ \hline
        複雑性 & 0.2  & 0.448 & 0.407 \\ \hline
    \end{tabular}   
    \end{center}
\end{table}

次に被験者についてである。被験者は本学の学生から募集をした。募集方法としては、ブラックジャックに関する実験であることや、戦略表を記憶するテストであるなどの実験の内容を事前に知らせずに募集した。募集の結果、男性15人、女性5人の計20人集めることが出来た。被験者の年齢は18〜21歳であった。

今回20人の被験者が集まったため実験時に2つのグループに分けて実験を行った。分けられたグループは記憶する戦略表が異なる。具体的な分け方は以下のとおりである。
\begin{itemize}
    \item グループX:戦略表A、Bを記憶するグループ
    \item グループY:戦略表A、Cを記憶するグループ
\end{itemize}
このようにグループ分けした理由としては、比較的複雑性の低い戦略表Aの成績を確認することで、戦略表Bと戦略表Cの成績の違いの要因が、戦略表の複雑性によってなのか個人差によってなのかを確認するためである。

最後に実験協力の報酬についてである。報酬は実験前に、実験結果の成績に応じて配られるという説明を行い、成績が高い被験者ほど多くの報酬が得られるように分配した。
\bunseki{※鳥谷航大}

\subsection{実験の手法・手順}

今回の実験は、異なる複雑性を持つ戦略表を3つ用意し、先述したグループごとに対応する戦略表を記憶してもらった後テストを行うというものである。その後、CRT(Cognitive Reflection Test)やリスク回避性のテストなどを行い、テストの成績との関連性を分析した。詳しい実験手順は以下のとおりである。
\begin{enumerate}
    \item ブラックジャックについての説明を行う
    \item 5〜10分程度の時間を用いて実際にプレイしゲームを理解してもらう
    \item 戦略表を配り10分間で記憶してもらう
    \item テストを配り7分間で回答してもらう
    \item 回答を配り被験者に採点をしてもらいテストを回収する
    \item 4〜6を戦略表を変更しもう一度繰り返す
    \item CRTとリスク回避性のテストについて回答してもらう。回答時間は両方合わせて10〜15分とした。
    \item 結果に応じた報酬を配り実験を終了する
\end{enumerate}
\bunseki{※鳥谷航大}

\subsection{実験結果}

今回の実験で得られた主な結果は以下の3つである。
\begin{itemize}
    \item 結果1:テストの成績と複雑性には強い負の相関があること
    \item 結果2:CRTの点数とテストの成績には相関がないこと
    \item 結果3:リスク回避性とテストの成績には相関がないこと
\end{itemize}
ここでは、私達が特に興味深く感じた結果1,2に関して詳しく説明していく。

以下はそれぞれのグループの成績である。また、戦略表のテストは満点が30点であり、CRTの満点は11点である。
\begin{table}[H]
    \begin{center}
    \caption{グループXの成績}
    \begin{tabular}{|c|c|c|c|}
    \hline
         & 戦略表A & 戦略表B  & CRT   \\ \hline
    平均   & 29.1 & 13.8  & 7.2   \\ \hline
    分散   & 2.29 & 12.96 & 3.96  \\ \hline
    標準偏差 & 1.51 & 3.6   & 2.098 \\ \hline
    \end{tabular}
    \end{center}
\end{table}

\begin{table}[H]
    \begin{center}
    \caption{グループYの成績}
    \begin{tabular}{|c|c|c|c|}
    \hline
         & 戦略表A & 戦略表C  & CRT   \\ \hline
    平均   & 29.8 & 17.5  & 8.8   \\ \hline
    分散   & 0.6  & 2.872 & 1.36  \\ \hline
    標準偏差 & 0.36 & 8.25  & 1.229 \\ \hline
    \end{tabular}
    \end{center}
\end{table}
両方のグループともに、戦略表Aは高い成績となった。そのため、成績とCRT正答率つまり、一般的な認知的判断能力の相関関係を見る時には、戦略表Aから戦略表BまたはCの点数を引いた差とCRT正答率との相関を考えた。以下がその散布図である。
\begin{figure}[H]   
    \begin{center}
        \includegraphics[width=10cm]{figure/groupX_crt_diffAB.png}
        \caption{CRT正答率と戦略表Aから戦略表Bの点数を引いた値との散布図}
    \end{center}
\end{figure}

\begin{figure}[H]
    \begin{center}
        \includegraphics[width=10cm]{figure/groupY_crt_diffAC.png}
        \caption{CRT正答率と戦略表Aから戦略表Cの点数を引いた値との散布図}
    \end{center}
\end{figure}
また、この散布図より相関係数を求めたところ以下のようになった。
\begin{table}[H]
    \begin{center}
    \caption{それぞれのグループの相関係数}
    \begin{tabular}{|c|c|c|}
    \hline
                  & グループX & グループY \\ \hline
    CRTと点数の差の相関係数 & 0.145 & 0.079 \\ \hline
    \end{tabular}
    \end{center}
\end{table}
上の表の通り、相関係数は非常に小さいという結果になった。ただし、相関係数が小さくても相関がある可能性があるため、無相関検定を行った。無相関検定の検定条件などは後述の複雑性と成績の相関係数と合わせて詳しく後で述べる。無相関検定の結果、グループX、グループYともに相関が無いことがわかった。

次に、複雑性とテストの成績の関連性についてである。以下の表が各戦略表の複雑性とテストの平均点である。
\begin{table}[H]
    \begin{center}
        \caption{各戦略表の複雑性とテストの平均点}
    \begin{tabular}{|c|c|c|c|c|}
    \hline
            & 戦略表A(グループX) & 戦略表A(グループY) & 戦略表B  & 戦略表C  \\ \hline
    テストの平均点 & 29.1        & 29.8        & 13.8  & 17.5  \\ \hline
    複雑性     & 0.2         & 0.2         & 0.448 & 0.407 \\ \hline
    \end{tabular}
    \end{center}
\end{table}
さらに、実験で得られたデータを用いて成績と複雑性の相関係数を求めたところ$-0.942$となり、ほぼ$-1$に近い値となった。こちらも無相関検定を行ったところ、非常に強い負の相関があるという結果になった。
\bunseki{※鳥谷航大}

\subsection{実験結果の検定と分析}

ここでは、先程の無相関検定についての説明と各戦略表のテストの平均点に差があるかどうかを調べるために行った分散分析と多重比較について述べる。

まずは無相関検定についてである。以下がグループX・グループYの成績とそれぞれのCRTの正答率との相関を調べる無相関検定の検定条件の表である。
\begin{table}[H]
    \begin{center}
        \caption{無相関検定の検定条件}        
        \begin{tabular}{|c|c|c|}
        \hline
            & グループX          & グループY          \\ \hline
        帰無仮説 & \multicolumn{2}{c|}{相関係数が0}     \\ \hline
        対立仮説 & \multicolumn{2}{c|}{相関係数が0ではない} \\ \hline
        相関係数 & 0.145          & 0.079          \\ \hline
        自由度  & \multicolumn{2}{c|}{8}          \\ \hline
        T値   & 0.415          & 0.224          \\ \hline
        有意水準 & \multicolumn{2}{c|}{0.05}       \\ \hline
        P値   & 0.689          & 0.829          \\ \hline
        \end{tabular}
    \end{center}
\end{table}
以上より、先述の通り、グループX・グループYともに相関係数は0となり、CRTの正答率とテストの成績には相関がないことが確認された。

次に、複雑性とテストの成績に関する無相関検定である。同様に、以下が検定条件の表である。
\begin{table}[H]
    \begin{center}
        \caption{無相関検定の検定条件}
        \begin{tabular}{|c|c|}
        \hline
            & 複雑性と成績     \\ \hline
        帰無仮説 & 相関係数が0     \\ \hline
        対立仮説 & 相関係数が0ではない \\ \hline
        相関係数 & -0.942     \\ \hline
        自由度  & 18         \\ \hline
        T値   & 7.963      \\ \hline
        有意水準 & 0.05       \\ \hline
        P値   & 0.000      \\ \hline
        \end{tabular}
    \end{center}
\end{table}
以上より、こちらも先述の通り、複雑性とテストの成績には強い負の相関があることが確認された。

次に、多重比較と分散分析についてである。今回の実験では、戦略表Aの複雑性が0.2となっていたためか、他のテストの成績に比べとても高い成績となった。そのため、先述の相関係数が負の相関を表したことが戦略表Aの結果に強く影響を受けている可能性や、B,C間の成績に違いがない可能性があるため分散分析と多重比較を行った。以下が、その検定条件と結果である。
\begin{table}[H]
    \begin{center}
        \caption{分散分析と多重比較の検定条件}
        \begin{tabular}{|c|c|c|c|c|}
        \hline
            & 全ての戦略表間       & A-B間     & A-C間     & B-C間     \\ \hline
        帰無仮説 & 各戦略表の平均点に差がない & \multicolumn{3}{c|}{戦略表間に差がない} \\ \hline
        対立仮説 & 各戦略間に差がある     & \multicolumn{3}{c|}{戦略間に差がある}  \\ \hline
        有意水準 & \multicolumn{4}{c|}{0.05}                      \\ \hline
        P値   & 0.000         & 0.000    & 0.000    & 0.007    \\ \hline
        \end{tabular}
    \end{center}
\end{table}
以上より、全ての戦略表の成績に差があり、かつ全ての戦略表間の成績に差があるという結果になった。
\bunseki{※鳥谷航大}

\subsection{考察}

以上の結果を全てまとめると、以下のようになる
\begin{itemize}
    \item 結果1:複雑性とテストの成績には強い負の相関がある
    \item 結果2:CRTの正答率とテストの成績には相関がない
    \item 結果3:リスク回避性とテストの成績には相関がない
\end{itemize}
以上の実験結果より、以下のようにまとめることが出来る。
\begin{itemize}
    \item 考察1:戦略表の複雑性が高いほどテストの成績は悪くなるため、複雑性は評価指標として適している
    \item 考察2:ブラックジャックの戦略を用いる能力は一般的な認知的判断能力とは異なる
    \item 考察3:個人のリスクを好むかどうかなどの正確はブラックジャックをプレイする上で大きな影響は与えない
\end{itemize}
以上より、本プロジェクトで定義した複雑性が適していることが確認されたため、実験は概ね成功だった。しかし、ブラックジャックをプレイするのに必要な能力が明確になっていないため、今後の実験で検証していきたい。
\bunseki{※鳥谷航大}
\chapter{シミュレータの作成について}
\section{概要}
今回ブラックジャックの戦略を検証するためにデック数を無限に設定したものとデック数を1に設定したものとで
2種類の条件を用意しシミュレーションを実行した。その際に勝率、デック数、複雑性のそれぞれの観点から仮
説を設定した。また、シミュレーションを実行するためのシミュレータをPython3を用いて作成した。その際に
Python Software Foundation(2018)のドキュメントを参照した。以下の部分ではシミュレーションの条件設定、
デック数についての説明とデック数の違いが生む影響、設定した仮説、シミュレータの詳細のそれぞれについて順
に述べていく。
\bunseki{尾崎拓海}

\section{シミュレーションの条件設定}
今回7つの戦略で実験を行う。その際、2つの条件を設定した。
\begin{itemize}
\item 条件1:デック数が無限であるとする
\item 条件2:デック数が1つであるとする
\end{itemize}
条件1は13種類のカードすべてを常に同じ確率で引くという状況に相当する。カードを何十枚、何百枚引いたと
しても次に引くカードの確率は一定であり、場に出たカードに影響を受けないという状況である。\\
条件2は次に引くカードが場にあるカードに影響を受けるという状況に相当する。次に引くカードの確率は既に
ゲームに使用されたカードや場に出ているカードによって変動するという状況である。\\
2つの条件で比較する際に共通している条件は以下の通りである。
\begin{itemize}
\item プレイヤーの人数とディーラーの人数はそれぞれ一人とする。
\item ベーシックストラテジー、ベーシックストラテジー改変1、ベーシックストラテジー改変2、15以上に
なるまでヒットする戦略、16以上になるまでヒットする戦略、17以上になるまでヒットする戦略、18以上に
なるまでヒットする戦略の、それぞれの戦略でシミュレータを実行する。
\item ゲームの実行回数は10万回とする。
\end{itemize}
\bunseki{尾崎拓海}

\subsection{デック数について}
今回デックはジョーカーを抜いた52枚1組を1デックと表す。例えば、8デックを使用するとした場合、1デック
52枚である事から、$52×8=416$、合計416枚のカードがゲームに使用される。また、一般的にはブラックジャック
で使用されるデック数は1,2,4,6,8デックのいずれかである場合が多い。\\
デック数によってデックの枚数や含まれるそれぞれのカードの枚数などが変化し、それがカードを引く確率に影響を
与える。そのためブラックジャックの勝敗にも影響を及ぼす。

\bunseki{尾崎拓海}

\subsection{デック数の違いによる影響}
今回のシミュレーションで比較するデック数無限とデック数1の双方において、10を1枚引いた後で更に続けて10を
引く確率を考えてみる。
\subsubsection{デック数無限の場合}
デック数が無限である事からデックに含まれる10の枚数も同様に無限となっている。この状態のデックから10を1枚
引いたとしてもデックに残っている10の枚数は変わらず無限であり、次に10を引く確率は一定となる。この事から次にデックから10を引く確率は\begin{equation}\frac{4}{52} \fallingdotseq 7.69%\end{equation}となる。\\
\subsubsection{デック数1の場合}
デック数1の場合使用するカードは52枚でありその中に10は4枚含まれている。この状態のデックから10を1枚引くと
デックに含まれる10の枚数は1枚少なくなる。この事から次にデックから10を引く確率は\begin{equation}\frac{3}{51}\fallingdotseq5.88%\end{equation}となる。\\
デック数無限とデック数1の条件とで10を引いた後に続けて10を引く確率を比較してみるとデック数無限のときは
およそ7.69%、デック数1の時はおよそ5.88%となり、デック数により次に引くカードの確率が変化していることが分かる。
これがデック数の違いによる影響である。以下の図\ref{hogehoge}はデック数による勝率の違いを表したもの
である。

\begin{figure}[H]
\begin{center}

 \includegraphics[width=0.7\linewidth]{./figure/DeckDiff.PNG}
 \caption{デック数による勝率の違い \label{hogehoge}}
\end{center}
\end{figure}

\bunseki{尾崎拓海}
\section{ブラックジャックシミュレータ}
ブラックジャックのシミュレータをプログラミング言語(Python3)を用いて作成した。このシミュレータを使用して、
ベーシックストラテジー、ベーシックストラテジー改変1、ベーシックストラテジー改変2、15以上になるまでヒットする
戦略、16以上になるまでヒットする戦略、17以上になるまでヒットする戦略、18以上になるまでヒットする戦略のそれぞれ
について勝ち、負け、引き分けの3つを調べた。また、ブラックジャックを行った際の資金の推移や勝率の推移
といったものを調査するためにシミュレータに所持金の概念を導入したり、実際の人間のように戦略を間違えるという処理を
追加し、ダブルダウンやスプリット、サレンダーといったプレイヤー側の選択肢を追加した。このシミュレータを使用して
様々な実験を実施した。ここではシミュレータ内部の詳細について記述していく。
\bunseki{尾崎拓海}

\subsubsection{基本設計}
まず初めに、シミュレータの基本設計について説明する。今回作成したシミュレータではブラックジャックを行う際に必要
となる要素をクラスとして表現した。具体的にはトランプのカードを表現するカードクラスとそれを一纏めにするデッククラス、
ゲーム参加者を表すクラスとそれを継承したプレイヤークラスとディーラークラス、ゲームの勝敗を判定するマネージャークラス
のそれぞれを定義した。これらのクラスを用いてブラックジャックのゲームを再現し、ベーシックストラテジーとその他の戦略を
実行する。次に各クラスの詳細を記述していく。
\bunseki{尾崎拓海}

\subsubsection{トランプのカードを表現するクラス}
このクラスでは実際のトランプのカードを表現するためにrankという変数にA~Kというトランプのランクを、suitという変数に
スペード、ハート、ダイヤ、クラブのスートを定義した。また、J,Q,K,Aの絵札カードは10や11と数える必要があったので、
ランクを数字に変換する処理もこのクラスに書き、valueという変数に入力した。
\bunseki{尾崎拓海}

\subsubsection{デックを表現するクラス}
このクラスでは先程定義したカードクラスを利用してデックを定義した。具体的には先程のカードクラスの配列を作成し、
その中にジョーカーを除く52種類のトランプカードを作成した。このクラスの初期化時に使用するデックの数を指定する。
また、デックのシャッフルには独自に作成した関数を使用した。このシャッフル関数はPython3のrandom関数を用いて
独自に設計したものであり、引数にシャッフルを行う回数を指定する。カードの配列の長さが仮に52だった場合には、
1~26番目のカードからランダムに取り出したカードと、27~52番目のからランダムに取り出したカードを交換するという
処理を(デック数×指定されたシャッフル回数)繰り返すという処理でシャッフル関数を作成した。
\bunseki{尾崎拓海}

\subsubsection{ゲーム参加者を表すスーパークラス}
このクラスでは自身の手札とその手札の合計値、手札に含まれるAの枚数、バーストしているかどうかのフラグ、
手札がブラックジャックとなっているかどうかのフラグのそれぞれを定義している。手札に含まれるAの枚数は
自身の手札の合計値を計算する時と、ブラックジャックの条件を満たしているかどうかを判別する際に使用した。
また手札の合計値を返す関数を定義し、その内側で自身がバーストしているかどうかの判定も行っている。
\bunseki{尾崎拓海}

\subsubsection{プレイヤークラス}
このクラスは先のゲーム参加者を表すスーパークラスを継承しており、ゲームに参加しているプレイヤーを表現
している。プレイヤークラスでは新たに自身の名前を表す変数と自身の勝利回数、敗北回数を記録する変数を
定義した。またこのクラスでは新しく、カードを受け取る関数とヒットを行う関数、スタンドを行う関数、勝利
回数と敗北回数を増加させる関数を作成した。また、ヒット、スタンド、ダブルダウン、サレンダーの処理を行う
関数を作成した。
\bunseki{尾崎拓海}

\subsubsection{ディーラークラス}
このクラスは先のゲーム参加者を表すスーパークラスを継承しており、ゲームのディーラーを表現しているクラス
となっている。ディーラークラスの中でデックをインスタンス化してディーラー側がデックを所持している事を表
現している。このクラスでは新しく、デックのシャッフル回数という変数を定義した。また、このクラスではカード
を配る関数、ディーラーの手札合計が17を超えるまでカードを引き続ける関数を作成した。カードを配る関数に
ついてはデック数有限の時とデック数無限の時とで処理を変更している。また、カウンティングを行う際に使用
されたカードを数える必要があったので、使用されたカードに対応してカウントを増減させる処理をカウンティング
手法ごとに記述した。
\bunseki{尾崎拓海}

\subsubsection{ゲームマネージャークラス}
このクラスは主にゲームの勝敗判定に使用している。プレイヤーとディーラーの手札の合計値を比較し勝敗を判定する
関数と、手札がブラックジャックになっているかどうかを判定する関数を作成した。また、プレイヤーの勝敗に応じて
資金を移動させる処理を記述した。勝敗判定のタイミングでプレイヤーの勝利回数、敗北回数のそれぞれを記録している。
\bunseki{尾崎拓海}

\subsubsection{メイン関数}
以上のクラスを用いてメイン関数にブラックジャックのゲームを記述した。以下にプログラムの実行手順を示す。
\begin{enumerate}
    \item ゲームに参加するプレイヤーを作成。今回はプレイヤーを一人のみ作成した。
    \item ディーラーを作成。
    \item カットカードを定義。カットカードを挟む位置はデックの半分の位置とした。
    \item ゲーム全体の実行回数を定義。今回は10万回とした。
    \item プレイヤーの戦略を配列形式で定義した。
    \item ゲームを繰り返すwhile文を作成し、ループ回数を10万回とした。
    \begin{enumerate}
        \item デックからカットカードが出てきたかを確認する。もし出てきていればデックをシャッフルする。
	  \item ディーラーが自身を含む各プレイヤーに初期カードを配る。
	  \item プレイヤーは自身の戦略に応じて掛け金を決定する。
	  \item プレイヤーは自身の戦略に沿った行動を選択する。
	  \item プレイヤーは戦略ごとに一定の確率で間違えた行動を選択する。
	  \item すべてのプレイヤーの行動が終了したことを確認後にディーラーが行動を開始する。
	  \item ディーラーの行動終了後に、勝敗判定を行う。
    \end{enumerate}
\end{enumerate}
\bunseki{尾崎拓海}

\subsubsection{エラー時の行動}
戦略の複雑性に応じてエラーのしやすさを定義した。エラーした際の行動は以下のようになる。
\begin{itemize}
    \item ヒットとスタンドのみで構成された戦略の場合
  \begin{itemize}
        \item ヒットでエラー:スタンド
        \item スタンドでエラー:ヒット
    \end{itemize}
    \item ヒット、スタンド、ダブルダウン、スプリットで構成された戦略の場合
    \begin{itemize}
        \item ヒットでエラー:80%スタンド、20%ダブルダウン
        \item スタンドでエラー:80%ヒット、20%ダブルダウン
        \item ダブルダウンでエラー:50%ヒット、50%スタンド
        \item スプリットでエラー:スプリットを行わない
    \end{itemize}
\end{itemize}
\bunseki{尾崎拓海}
\input{simulator/random.tex}
\chapter{ベーシックストラテジーの検証}
\input{verification/verification.tex}
\section{仮説}
今回は仮説を以下のように設定した。
\begin{itemize}
\item 勝率に関して
    \begin{itemize}
        \item 仮説1.デック数が無限の時にはベーシックストラテジーの方が勝率が高い
        \item 仮説2.デック数が1の時にはベーシックストラテジー以外の勝率が高い
    \end{itemize}
\item 複雑性を考慮した場合
    \begin{itemize}
        \item 仮説3.プレイヤーの合計値が15,16,17,18以上になるまでヒットする戦略の方が性能が高い
    \end{itemize}
\item デック数を考慮した場合
    \begin{itemize}
        \item 仮説4.デック数1とデック数無限では勝率に有意な差が出る
    \end{itemize}
\end{itemize}
ベーシックストラテジーの表はデック数が無限であることを前提として導出されている。我々はこの点に着目し、デック数が有限になった際にはベーシックストラテジーよりも優れた戦略が存在するのではないか、あるいは、ベーシックストラテジーはデック数有限には対応しきれないのではないかと考えた。こうした考えから仮説1、仮説2のそれぞれを設定した。また、基準値以上になるまでヒットする戦略の方が複雑性が低くなり、性能の評価がよくなるのではないかという考えから仮説3を設定した。デック数1とデック数無限では,カードを引く確率が変化する事から、デック数が違えば勝率に有意な差が出るのではないかと考え、仮説4を設定した。
\bunseki{※尾崎拓海}

\section{検証手順}
設定した仮説を以下の手順で検証した。
\begin{enumerate}
\item ブラックジャックのシミュレータを作成
\item デック数が1の場合と無限の場合でシミュレーションを10万回実施
\item 勝った割合、負けた割合、引き分けた割合の3つを調べた
\item 得られた結果から基本戦略とその他の戦略との間の勝率に有意な差があるかどうかをカイ二乗検定を用いて調べた
\end{enumerate}
\bunseki{※尾崎拓海}
\input{verification/statistics.tex}
\input{verification/complexity.tex}
\chapter{遺伝的アルゴリズム}
\input{geneticalgorithm/geneticalgorithm.tex}
\section{GA戦略の戦績}
\subsection{シミュレーション結果}
ここでは得られたGA戦略とダブルダウンやスプリットなどが入ったベーシックストラテジー(BS)とヒットとスタンドのみのベーシックストラテジー(BS-HS)の3つをシミュレータを用いてデック無限の条件下でそれぞれ10万回ブラックジャックを行った。その結果を表\ref{tab:score}に示す

\begin{table}[H]
 \caption{デック数無限と各戦略での勝ち、負け、引き分けの比率}
 \begin{center}
  \begin{tabular}{|c|c|c|c|}
  \hline   & 勝ち(\%) & 負け(\%)  & 引き分け(\%) \\
  \hline BS & 43.762 & 47.566 & 8.671 \\
  \hline BS-HS & 42.746 & 48.635 & 8.619 \\
  \hline GA戦略 & 42.858 & 49.502 & 7.64 \\
  \hline
  \end{tabular}
  \label{tab:score}
 \end{center}
\end{table}

今回は勝ち、負け、引き分けの3種類に分けて、比率で表したものが表\ref{tab:score}となっている。勝ちの比率が最も高いのはBSとなった。そしてGA戦略は勝率では BS-HSを上回っていたが、負けの比率が最も高いのはGA戦略となった。
\bunseki{※葛西隼人}

\subsection{カイ2乗検定}
先程のシミュレーションを行った結果から、勝率が最も高いのはBSとなったが、勝ち負け引き分けに優位な差があるのではないか。これを確かめるために、カイ2乗検定を行い、勝ち負け引き分けに優位な差があるかどうかを確かめる。
 帰無仮説は勝ち負け引き分けに有意な差がない。対立仮説は勝ち負け引き分けに有意な差があるとし、カイ2乗検定を行う前に必要な条件を表\ref{score-x}で必要な条件をまとめた。
\begin{table}[H]
 \caption{勝ち負け引き分けのカイ2乗検定条件\label{score-x}}
 \begin{center}
  \begin{tabular}{|c|c|}
  \hline 帰無仮説 & 勝ち負け引き分けに有意な差がない \\
  \hline 対立仮説 & 勝ち負け引き分けに有意な差がある \\
  \hline 有意水準 & 5\% \\
  \hline 自由度 & 2 \\
  \hline 棄却値 & 5.99 \\
  \hline
  \end{tabular}
 \end{center}
\end{table}

棄却値は5.99である。この棄却値よりもカイ2乗値が大きい場合、帰無仮説を棄却して対立仮説が採択される。カイ2乗検定を行うとカイ2乗値は66.754となった。この値をp値に変換すると0.1よりちいさくなった。よって帰無仮説を棄却して対立仮説を採択する。つまり、勝ち負け引き分けに優位な差が存在する。
\bunseki{※葛西隼人}

\subsection{残差分析}
カイ2乗検定によって各戦略間の勝ち負け引き分けに優位な差が存在することが判明した。しかしどこに優位な差が存在するかまでは明らかになっていない。そのためさらに検定を行いどの戦略に優位な差があるかを探す。そのために残差分析を行う。残差分析では調整済み標準化残差を算出し、調整済み標準化残差が1.96より大きい場合と-1.96より小さい場合に優位な差があると分かる。調整済み標準化残差をそれぞれ出したものを表\ref{score-zansa}でまとめる。

\begin{table}[H]
 \caption{調整済み標準化残差}
 \begin{center}
  \begin{tabular}{|c|c|c|c|}
  \hline   & 勝ち(\%) & 負け(\%)  & 引き分け(\%) \\
  \hline BS & 5.051 & -7.83 & 5.113 \\
  \hline BS-HS & -2.98 & 0.59 & 4.277 \\
  \hline GA戦略 & -2.016 & 7.293 & -9.425 \\
  \hline
  \end{tabular}
  \label{tab:score-zansa}
 \end{center}
\end{table}

表\ref{score-zansa}ではそれぞれ戦略での勝ち、負け、引き分けの3つの調整済み標準化残差を示している。例えば、BSの勝ちの項目を見ると値が5.051となっており、1.96以上なのでBSの勝ちの比率は他の戦略より優位に多かったということがわかる。

GA戦略を見ると負けの比率が1.96を上回っており他の戦略より優位に多かったということがわかる。逆に、勝ちと引き分けの比率がどちらも-1.96を下回っており優位に少ないことがわかった。

以上の結果からGA戦略は勝ちの比率はBS-HSと同じだが、引き分けの比率が少なくなり、負けの比率が多くなってしまった事がわかる。
\bunseki{※葛西隼人}







\subsection{SGAの概要と問題点}
この項では遺伝的アルゴリズムの問題点について説明していく。佐藤ら(1997)は代表的な遺伝的アルゴリズムでの世代交代モデルとしてSimple GA(以下SGAと表記する)を挙げ、そのモデルの問題点と改善案を示している。ただし、ここでは適応度をもとにした選択処理についての問題点と改善案を挙げており、交叉アルゴリズムや遺伝子のコード設計、各種パラメータの設定に関しては扱っていない。

世代交代のモデルには、次世代の個体を生成するための親を選択する複製選択と、全ての個体の中から次世代に残す個体を選択する生存選択の2種類の処理が存在する。SGAの場合この2つの選択処理は次のようになっている。

\begin{itemize}
\item{複製選択}\\
適応度に比例した選択確率を用いたルーレット選択方式によって、集団から個体を復元抽出する。
復元抽出とは一度選択された個体も次以降の選択対象に含める、つまり同じ個体が複数回選ばれることを許している選択方法である。
\item{生存選択}\\
無条件で親集団と生成された子集団のすべてを入れ替える。
\end{itemize}

佐藤ら(1997)によるとSGAには3つの問題点があることが指摘されている。
\begin{enumerate}
\item{高い選択圧下での早期収束}
\begin{itemize}
\item{SGAのように適応度を用いてルーレット選択を行っている場合、探索の初期に適応度が突出した個体が存在するとその個体が複製選択において選ばれる可能性が高くなりすぎてしまう。そうなった場合、探索の序盤からその個体に遺伝子全体が収束してしまう現象が起きてしまい、最適な解にたどりつきづらくなってしまう。この現象は初期収束と呼ばれている。}
\end{itemize}
\item{低い選択圧下での停滞}
\begin{itemize}
\item{遺伝的アルゴリズムの探索が進み、各世代の個体間の適応度に差が見られなくなってきた時に複製選択においてルーレット選択の効果が弱くなることがある。そうなると個体が最適な解に向かって進みづらくなってしまう現象が起こることがある。ここでは進化的停滞と呼んでいる。}
\end{itemize}
\item{優秀な遺伝子の破壊}
\begin{itemize}
\item{SGAでは親から子に世代が移る時は無条件で全ての遺伝子を入れ替えてしまう。そのため親個体に適応度の高い個体が存在する場合でもその個体は次の世代では失われてしまう。そのため集団から適応度の高い個体が失われてしまう可能性がある。}
\end{itemize}
\end{enumerate}
\bunseki{伊藤晋之介}

\subsection{SGAの改善案の紹介}
前項で挙げられたようにSGAの世代交代モデルには、初期収束、進化的停滞などの様々な問題点が存在している。これらの問題点を解決するため、佐藤ら(1997)ではSGAの複製選択アルゴリズムと、生存選択アルゴリズムを変更したいくつかの世代交代モデルが紹介されている。

\subsubsection{Iterated Genetic Search(IGS)}
\begin{itemize}
\item{複製選択}\\
適応度を無視して集団から個体をランダムに非復元抽出する。非復元抽出とは、同じ個体を1度しか選択しない選び方のこと。
\item{生存選択}\\
適応度の平均以下の個体をランダムに選び、生成された子個体のうち数体と入れ替える。
\end{itemize}
このモデルでは親個体の選択に適応度を用いず、ランダムに選択している。そのため上で挙げられた初期収束に陥る可能性を避けることができる。生存選択の時には適応度が平均以下の個体を子世代に混ぜることで、適応度が高い個体同士からは作り出せない新たな個体が生成されることを期待できる。そのため効率的に探索を行える可能性がある。

\subsubsection{Steady State(SS)}
\begin{itemize}
\item{複製選択}\\
ランキング選択法を用いて集団から個体を復元抽出する。
\item{生存選択}\\
親集団から最悪個体を選び、生成された子個体と入れ替える。
\end{itemize}
このモデルでは複製選択にランキング選択法を用いている。ランキング選択法とは親個体を選ぶ際に適応度が高い順に個体に順位をつけていく。そして順位に応じてその個体が選択される数を決める方法。ルーレット選択方式に比べ、極端に適応度が高い個体が存在する場合でもその個体以外の個体も確実に選ばれることを保証できる。そのため初期収束してしまう可能性を抑えることができる。生存選択ではIGSと同じように適応度の低い個体をいくつか混ぜることで、新たな個体が生成されることを期待できる。

\subsubsection{CHC}
\begin{itemize}
\item{複製選択}\\
適応度を無視して集団から個体をランダムに非復元抽出する。
\item{生存選択}\\
親集団と子集団を合わせた2世代の中から、適応度の高い順に集団サイズ分の個体を次世代に残す。
\end{itemize}
複製選択については個体をランダムに選択し初期収束を回避している。生存選択では親個体をすべて子個体と入れ替えるのではなく、親世代と子世代を合わせたものの中から適応度順に集団サイズ分だけ選んでいる。この処理を行うことで、親世代の個体が次世代以降も生き残る可能性が出てくる。そうすることで親個体に存在した優秀な個体が世代交代で失われるのを防ぐことができる。

\subsubsection{Elitist Recombination(ER)}
\begin{itemize}
\item{複製選択}\\
適応度を無視して集団から個体をランダムに非復元抽出する。
\item{生存選択}\\
各家族、すなわち親として選ばれた2個体とそこから生成された子の2個体の中から適応度の高い2個体を次世代に残す。
\end{itemize}
複製選択では初期収束を回避するため個体をランダムに選択している。生存選択では各家族間で適応度を比べ優秀なものが生き残るようになっている。そのため親世代の優秀な個体が失われることを防ぐことができる。またCHCとは異なり適応度の比較が家族間で行われることで、優秀な個体が急激に集団に広まることを防ぐ効果もある。
\bunseki{伊藤晋之介}

\subsection{今回設計したGAのまとめと改善案}
\subsubsection{まとめ}
今回我々が作成した遺伝的アルゴリズムでは、親世代で最も優秀な個体を次世代に残すエリート保存方式を採用している。そのため上記の問題点で挙げている優秀な遺伝子の破壊が起きる可能性は低いと考えられる。図\ref{gaprocess}からもわかる通り最優秀個体の適応度は減少してはいない。\\
問題点としては、初期個体にブラックジャックの基本戦略、14以上でスタンドする戦略、15以上でスタンドする戦略、16以上でスタンドする戦略をそれぞれ5個体ずつ設定したことが挙げられる。今回は遺伝的アルゴリズムの実行時間を短縮するため、また最適な解を早く発見するため、戦略として優秀な基本戦略とそれを改変した戦略を初期個体に混ぜた。そのため個体全体がそれらの戦略に初期収束してしまった可能性がある。これは遺伝的アルゴリズムから出力された戦略表からもわかる。遺伝的アルゴリズムから出力された戦略表はハードハンドでは14以上でスタンドする戦略がそのまま出力され、ソフトハンドでは基本戦略から1つの遺伝子を変化させたものになっている。


\subsubsection{改善案}
改善案としては、1つ目に初期個体に含める基本戦略などの数を減らし、初期値の補正を緩くすることが考えられる。しかし初期値に補正を加えない場合、各個体のHとSが完全にランダムになり、複雑性が高くなりすぎてしまう可能性が考えられる。そのため解の探索が上手くいかず、優秀な戦略が得られない危険性がある。\\
2つ目に、作成した遺伝的アルゴリズムの個体数を増やすことが考えられる。今回の探索では実行時間が短かったこと、コンピュータの性能などの理由から個体数を各世代200体に設定した。しかしこの個体数を増やすことで一度に探索できる数が増えより良い解を見つけられる可能性がある。
3つ目にエリート保存方式の調節が挙げられる。今回エリート保存方式で保存する個体の数は1体のみだった。保存する個体数を増やすことで優秀な遺伝子をより多く次世代に生き残らせることができ、探索を効率的に行える効果があると考えられる。
その他細かい改善点としては交叉確率、突然変異確率など各種パラメータの調整、選択アルゴリズムをルーレット方式からランキング方式に変更することなどが挙げられる。
\bunseki{伊藤晋之介}
\chapter{賭け金ありのシミュレーション}
ブラックジャックにはダブルダウンやスプリットがあるので勝率が低くても利得を増やす手段がある。
また、賭け額を操作することで利得を増やすことができる。例えば、10戦して1勝9敗だとしても1勝の時にだけ
多額を賭けていれば勝率が低くても高い利得を得ることができる。なのでここからは賭け金を導入して検証を行う。
\bunseki{轟木文弥}
\section{この章で扱うシミュレーションに用いる戦略表について}
この章で扱うシミュレーションを行うにあたって、3つの戦略を用意した。
ヒットとスタンドの他に、ダブルダウンやスプリットといったルールを含めたベーシックストラテジー、ヒットとスタンドのみのベーシックストラテジー、前述のGA戦略の3つである。
なお、「ヒットとスタンドのみのベーシックストラテジー」は前述していたベーシックストラテジーの事を指す。

これから、3つの戦略の内ダブルダウンやスプリットといったルールを含めたベーシックストラテジーについて説明する。ヒットとスタンドのみのベーシックストラテジー、GA戦略については前述している為簡単に説明する。

これから説明する表の文字はそれぞれ以下のように対応している。

(H:ヒット, \ S:スタンド, \ D:ダブルダウン, \ P:スプリット)  
\\

ルールの項でも説明しているが、改めて説明する。

ヒットは山札からカードを1枚手札に追加することである。21を超えない限り、何度でもヒットすることができる。

スタンドはカードを引かずに自分の行動を終了することである。

スプリットは最初に配られた2枚のカードが同じ数字だった場合、最初の賭け金と同じ金額を追加して、それらのカードを2つに分割し、それぞれの手札で勝負するルールである。

ダブルダウンは賭け金を2倍にして1度のみヒットをし、その後強制的にスタンドするルールである。

\bunseki{渡邊凛}


\subsection{ダブルダウンやスプリットを含めたベーシックストラテジーについて}

この戦略は、Thorp(1962)の1デック用のベーシックストラテジーではなく、Jensen(2014)の4~8デック用のものを参考にしたものである。
この戦略表では、スプリット、ダブルダウンが追加されている。

これらの選択肢を戦略に追加することにより、同じ賭け金でもより期待値が高くなる。
ただし、選択肢が増える分表も大きくなり複雑性が増す為に人間にとって扱いづらく、誤る可能性が高いと言える。

\begin{table}[H]
    \begin{center}
    \caption{ダブルダウンやスプリットを含めたベーシックストラテジー}
    \begin{tabular}{|c|c|c|c|c|c|c|c|c|c|c|c|c|}
    \hline
    \multicolumn{3}{|c|}{\multirow{2}{*}{}}                     & \multicolumn{10}{c|}{ディーラーのアップカード}     \\ \cline{4-13} 
    \multicolumn{3}{|c|}{}                                      & 2 & 3 & 4 & 5 & 6 & 7 & 8 & 9 & 10 & A \\ \hline
    \multirow{28}{*}{手札の合計} & \multirow{9}{*}{ハードハンド}   & 9     & H & H & H & H & H & H & H & H & H  & H \\ \cline{3-13} 
                            &                           & 10    & H & D & D & D & D & H & H & H & H  & H \\ \cline{3-13} 
                            &                           & 11    & D & D & D & D & D & D & D & D & H  & H \\ \cline{3-13} 
                            &                           & 12    & D & D & D & D & D & D & D & D & D  & D \\ \cline{3-13} 
                            &                           & 13    & H & H & S & S & S & H & H & H & H  & H \\ \cline{3-13} 
                            &                           & 14    & S & S & S & S & S & H & H & H & H  & H \\ \cline{3-13} 
                            &                           & 15    & S & S & S & S & S & H & H & H & H  & H \\ \cline{3-13} 
                            &                           & 16    & S & S & S & S & S & H & H & H & H  & H \\ \cline{3-13} 
                            &                           & 17以上  & S & S & S & S & S & S & S & S & S  & S \\ \cline{2-13} 
                            & \multirow{9}{*}{ソフトハンド}   & A,2   & H & H & H & D & D & H & H & H & H  & H \\ \cline{3-13} 
                            &                           & A,3   & H & H & H & D & D & H & H & H & H  & H \\ \cline{3-13} 
                            &                           & A,4   & H & H & D & D & D & H & H & H & H  & H \\ \cline{3-13} 
                            &                           & A,5   & H & H & D & D & D & H & H & H & H  & H \\ \cline{3-13} 
                            &                           & A,6   & H & D & D & D & D & H & H & H & H  & H \\ \cline{3-13} 
                            &                           & A,7   & S & D & D & D & D & S & S & H & H  & H \\ \cline{3-13} 
                            &                           & A,8   & S & S & S & S & S & S & S & S & S  & S \\ \cline{3-13} 
                            &                           & A,9   & S & S & S & S & S & S & S & S & S  & S \\ \cline{3-13} 
                            &                           & A,10  & S & S & S & S & S & S & S & S & S  & S \\ \cline{2-13} 
                            & \multirow{10}{*}{スプリット可能} & A,A   & P & P & P & P & P & P & P & P & P  & P \\ \cline{3-13} 
                            &                           & 2,2   & P & P & P & P & P & P & H & H & H  & H \\ \cline{3-13} 
                            &                           & 3,3   & P & P & P & P & P & P & H & H & H  & H \\ \cline{3-13} 
                            &                           & 4,4   & H & H & H & P & P & H & H & H & H  & H \\ \cline{3-13} 
                            &                           & 5,5   & D & D & D & D & D & D & D & D & H  & H \\ \cline{3-13} 
                            &                           & 6,6   & P & P & P & P & P & H & H & H & H  & H \\ \cline{3-13} 
                            &                           & 7,7   & P & P & P & P & P & P & H & H & H  & H \\ \cline{3-13} 
                            &                           & 8,8   & P & P & P & P & P & P & P & P & P  & P \\ \cline{3-13} 
                            &                           & 9,9   & P & P & P & P & P & S & P & P & S  & S \\ \cline{3-13} 
                            &                           & 10,10 & S & S & S & S & S & S & S & S & S  & S \\ \hline
    \end{tabular}
    \end{center}
\end{table}


\bunseki{渡邊凛}


\subsection{ヒットとスタンドのみのベーシックストラテジーとGA戦略について}

ヒットとスタンドのみのベーシックストラテジーは表\ref{kihonsenryaku}をそのまま使用している。GA戦略は遺伝的アルゴリズムによって生成された戦略表であり、表\ref{gastrategyhard}、\ref{gastrategysoft}をそのまま使用している。これらの戦略にはダブルダウンやスプリットはなく、ヒットとスタンドのみである。そのため、戦略としては単純になり、人間にとって扱いやすい戦略表となる。

\bunseki{渡邊凛}
\section{一定ベット}
まずは、賭け金を常に一定として、所持金がどのように変化するのかをシミュレーションで調べる。シミュレーション結果を図\ref{betdife}と表\ref{bet}で示す。
\begin{figure}[H]
 \begin{center} 
  \includegraphics[width=0.7\linewidth]{./figure/betsimulation-bet-defineite-ver5}
  \caption{一定ベット\label{betdife}}
 \end{center}
\end{figure}

\begin{table}[H]
 \caption{一定ベットの所持金\label{bet}}
 \begin{center}
  \begin{tabular}{|c|c|c|}
  \hline  & 40回目の平均所持金 & 標準偏差 \\
  \hline BS & 998.328 & 73.538\\
  \hline BS-HS & 990.499 & 62.013 \\
  \hline GA戦略 & 983.470 & 62.401\\
  \hline
  \end{tabular}
 \end{center}
\end{table}

このシミュレーションでは最初の所持金を1000とし、常に10賭け、ブラックジャックの勝負を連続で40回行い、それを5万回行った。また戦略はダブルダウンやスプリットなどが入ったベーシックストラテジー(BS)とヒットとスタンドのみのベーシックストラテジー(BS-HS)、GAで作成した戦略(GA戦略)の3つを使用した。
 図\ref{betdife}では所持金は勝負を重ねるたびに少しずつ少なくなっていて、最後の40回目では最初の所持金である1000より少し少なくなっているように見える。そこで表\ref{bet}を見ると40回目のそれぞれの所持金が分かるが、ダブルダウンやスプリットが入ったベーシックストラテジー、ヒットとスタンドのみのベーシックストラテジー、GA戦略の順で多いことが分かる。どの戦略でも最初の所持金である1000を超えることはないことが分かった。\\
 賭け金が一定では最初の所持金を超えることができなかった。そのため賭け金を変動させることが必要だと考えた。まず、この勝負40回の中で勝率が変動するところがないかを確認した。勝負のどこかで勝率が高くなれば賭け金をあげ、低くなれば賭け金を下げることで所持金を増やそうと考えたためである。また、カウンティングという手法を用いてみるという2つの手法を考えた。
\bunseki{柿崎大輝}

\subsection{勝率の推移}
ブラックジャックを連続で行った場合で勝率は変化を調べる。シミュレーション結果を図\ref{win}で示す。
\begin{figure}[H]
 \begin{center} 
  \includegraphics[width=0.7\linewidth]{./figure/betsimulation-win}
  \caption{勝率の推移\label{win}}
 \end{center}
\end{figure}
ブラックジャックの勝負を連続で40回行い、それを5万回繰り返して、勝負ごとでの勝率を調べた。デック数は6で、使用した戦略はヒットとスタンドのみのベーシックストラテジーであった。
図\ref{win}を見ると勝率は42\%~44\%の間で増減を繰り返しており、だいたい43\%の部分に集中していると分かる。勝率は特に目立った規則性は見られず、ランダムに増減を繰り返しているように見られる。勝率は最高で43.87\%で最低は42.868\%であった。
 シミュレーション結果から勝率は勝負40回の中であまり変化しないと考えられる。また、勝率の変動に規則性は見つからなかった。そのため、勝負のどこかで賭け金を増やすという戦略はできないということが分かった。
\bunseki{柿崎大輝}

\subsubsection{カイ2乗検定}
40回の勝負で1つ1つを比べてみて、勝率の差は小さいものであった。しかし、本当にその差には意味がないのか、小さくても本当は意味がある差なのではないか。それを確かめるために、カイ2乗検定を行い、勝率の差が有意な差であるかどうかを確かめる。
 カイ2乗検定を行う前に必要な条件を表\ref{win-x}でまとめる。
\begin{table}[H]
 \caption{勝率の差のカイ2乗検定条件\label{win-x}}
 \begin{center}
  \begin{tabular}{|c|c|}
  \hline 帰無仮説 & 勝率の差に有意な差がない \\
  \hline 対立仮説 & 勝率の差に有意な差がある \\
  \hline 有意水準 & 5\% \\
  \hline 自由度 & 39 \\
  \hline 棄却値 & 54.57\\
  \hline
  \end{tabular}
 \end{center}
\end{table}
有意水準5\%で、棄却値は54.57とした。カイ2乗値がこの棄却値54.57より大きければ帰無仮説を棄却し、対立仮説を採択する。
 ここで、カイ2乗検定を行うとカイ2乗値は40.207となった。カイ2乗値は棄却値よりも小さくなったので、帰無仮説を採択する。よって、勝率の差には有意な差がないという結果になった。つまり、勝負40回で勝率の変化はないという結果となった。
\bunseki{柿崎大輝}

\section{カウンティング}
カウンティングで賭け金を変動させ、所持金を調べる。カウンティングは既存のカウンティング手法としてKO法とHigh-Low法の2種類を用いる。そのほかの条件は一定ベットの時のシミュレーションと同じとした。シミュレーション結果を図\ref{KO}と図\ref{Hi-Lo}と表\ref{countting}で示す。
\begin{figure}[H]
 \begin{center} 
  \includegraphics[width=0.7\linewidth]{./figure/betsimulation-KO}
  \caption{カウンティングKO法\label{KO}}
 \end{center}
\end{figure}

\begin{figure}[H]
 \begin{center} 
  \includegraphics[width=0.7\linewidth]{./figure/betsimulation-Hi-Lo}
  \caption{カウンティングHigh-Low法\label{Hi-Lo}}
 \end{center}
\end{figure}

\begin{table}[H]
 \caption{カウンティングの所持金\label{countting}}
 \begin{center}
  \begin{tabular}{|c|c|c|}
  \hline  & 40回目の平均所持金 & 標準偏差 \\
  \hline KO法BS & 1000.237 & 172.770\\
  \hline KO法BS-HS & 987.207 & 148.615 \\
  \hline KO法GA戦略 & 976.380 & 145.821\\
  \hline High-Low法BS & 1027.961 & 298.197\\
  \hline High-Low法BS-HS  & 997.561 & 262.546\\
  \hline High-Low法GA戦略 & 981.315 & 263.162\\
  \hline
  \end{tabular}
 \end{center}
\end{table}

KO法は図\ref{KO}で見ると一定ベットの時とあまり変化がないように見えるが、表\ref{countting}で見るとダブルダウンやスプリットが入ったベーシックストラテジーの所持金が増加していることが分かる。
 High-Low法は図\ref{Hi-Lo}ではで増えていることがはっきりと分かる。表\ref{countting}で見てもすべての戦略で一定ベットの時よりも増えていることが分かる。しかし、標準偏差が一定ベットやKO法の2つより大きく、安定していないことが分かる。
 既存のカウンティング手法を使うことでダブルダウンやスプリットが入ったベーシックストラテジーでは1000を超えることができた。ただ、ヒットとスタンドのみのベーシックストラテジーやGA戦略では1000を超すことはできなかった。
\bunseki{柿崎大輝}

\section{新しいカウンティング手法の提案}
KO法は安定しているが変化が小さく、High-Low法は変化が大きいが安定しなかった。本プロジェクトはこの2つの問題を解消するべく新たに2つのカウンティング手法を考えた。それがRUKO法と5-6-A法である。今回はKO法をベースにして、KO法の変化を大きくする手法を考えた。ここではその2つのカウンティング手法について詳しく説明する。
\bunseki{轟木文弥}

\subsection{RUKO法}
RUKO法とはRisk Up KO法の略で、KO法のリスクを上げ利得を上げようとしたカウンティング手法である。KO法は6デックの場合初期ランニングカウントが-20で、ランニングカウントが-5以下の時は1ユニットしか賭けない。そのため賭け金を上げる段階が遅くなっている。そこでRUKO法は、初期ランニングカウントをKO法の半分とした。すなわち6デックの場合は-10である。これによりKO法より早い段階から賭け金を大きくし、利得を上げようとした。カードカウントとベットシステムはKO法と同じである。
\bunseki{轟木文弥}

\subsection{5-6-A法}
5-6-A法とはカードの有利さの検証結果から有意な差がある部分の5,6,Aを考慮したカウンティング手法である。先述のように、KO法は賭け金を上げる段階が遅くなっている。そこで5-6-A法では以下の\ref{56Acount}のように特に不利なカードである5と6がでた時にプラス2を加え、特に有利なカードであるAが出た時にランニングカウントをマイナス2を加えた。ただし、10は特に有利なカードであるが、10はデックに存在する枚数が多く-2にするとカウントが大きくならず賭け金があまり上がらなくなるため-1とした。

\begin{table}[H]
    \centering
    \label{56Acount}
    \caption{5-6-A法でのカードカウント}
    \begin{tabular}{|c|c|c|c|c|c|c|c|c|c|} \hline
      2&3&4&5&6&7&8&9&10&A \\ \hline
      +1&+1&+1&+2&+2&+1&0&0&-1&-2 \\ \hline
    \end{tabular}
\end{table}

初期ランニングカウントとベットシステムはKO法と同じである。この手法によりKO法よりランニングカウントを大きくし賭け金を上げ利得を増やそうとした。また、このカードカウントにより正確にデックの状態を判断しようとした。
{轟木文弥}

\section{RUKO法、5-6-A法のシミュレーション}
カウンティングで賭け金を変動させ、所持金を調べる。カウンティングは本プロジェクトで作成したカウンティング手法のRUKO法と5-6-A法の2種類を用いる。そのほかの条件は一定ベットの時のシミュレーションと同じとした。シミュレーション結果を図\ref{RUKO}と図\ref{56A}と表\ref{counttingB}で示す。


\begin{figure}[H]
 \begin{center} 
  \includegraphics[width=0.7\linewidth]{./figure/betsimulation-RUKO}
  \caption{カウンティングRUKO法\label{RUKO}}
 \end{center}
\end{figure}

\begin{figure}[H]
 \begin{center} 
  \includegraphics[width=0.7\linewidth]{./figure/betsimulation-56A}
  \caption{カウンティング5-6-A法\label{56A}}
 \end{center}
\end{figure}

\begin{table}[H]
 \caption{カウンティングの所持金\label{countting}}
 \begin{center}
  \begin{tabular}{|c|c|c|}
  \hline  & 40回目の平均所持金 & 標準偏差 \\
  \hline KO法BS & 1000.237 & 172.770\\
  \hline KO法BS-HS & 987.207 & 148.615 \\
  \hline KO法GA戦略 & 976.380 & 145.821\\
  \hline High-Low法BS & 1027.961 & 298.197\\
  \hline High-Low法BS-HS  & 997.561 & 262.546\\
  \hline High-Low法GA戦略 & 981.315 & 263.162\\
  \hline RUKO法BS & 1000.237 & 172.770\\
  \hline RUKO法BS-HS & 987.207 & 148.615 \\
  \hline RUKO法GA戦略 & 976.380 & 145.821\\
  \hline 5-6-A法BS & 995.918 & 211.884\\
  \hline 5-6-A法BS-HS & 982.640 & 179.584 \\
  \hline 5-6-A法GA戦略 & 968.303 & 180.464\\
  \hline
  \end{tabular}
 \end{center}
\end{table}
{轟木文弥}

\\RUKO法の特徴は、6デックの場合のKO法の初期ランニングカウントが-20であるのに対し-10に変更することで、より早い段階から賭け金を大きくすることにより最終的な利得を増やすことを目的として作成したものである。RUKO法を使用した場合の40回目の所持金の推移のグラフ\ref{RUKO}を見てみると、BS、BS_HS、GAの全ての戦略が最初の所持金額1000を下回る結果となった。これはオリジナルのKO法と比較してみても、より平均所持金を下回っていたことがわかった。また、標準偏差もKO法と比べて増加しており、安定していないことがわかった。
\bunseki{薩田凱斗}
\\5-6-A法の特徴は、KO法のカードカウントを変更することで、より早い段階から賭け金を大きくすることにより最終的な利得を増やすことを目的として作成したものである。5-6-A法を使用した場合の40回目の所持金の推移のグラフ\ref{5-6-A}を見てみると、BS、BS_HS、GAの全ての戦略が最初の所持金額1000を下回る結果となった。これはオリジナルのKO法と比較してみても、より平均所持金を下回っていたことがわかった。また、標準偏差もKO法と比べて増加しており、安定していないことがわかった。
\bunseki{轟木文弥}
\\ 既存のカウンティング手法であるKO法とHigh-Low法の一部を変更して作成したRUKOと5-6-Aの結果をまとめると、どちらもぞれぞれのオリジナルの手法よりも安定性が下がり平均所持金が減るという悪い結果となってしまった。今後の課題としては、それぞれの手法で改変した部分がなぜ平均所持金を下回った結果となってしまったかを究明するとともに、目的である最終的な利得を増やすことに重点を置いて改良していきたい。
\bunseki{薩田凱斗}
\\ カウンティング手法全部のまとめ\\
 ここで注目するのはGA戦略である。一定ベットやカウンティングを使用しても所持金がすべての戦略で一番少なかった。GA戦略は遺伝子アルゴリズムで探索し、発見した戦略であるのになぜ他の戦略より悪い結果となったのか。それは複雑性ということを考慮していないからではないかと考えた。GA戦略は勝率と複雑性からなる性能で探索したもので、行ったシミュレーションでは複雑性を考慮する部分がなく、その結果一番悪い結果となったと考えた。そこで複雑性を確かめる実験の結果を用いて、"エラー率"という指標を作成し、シミュレーションに導入して行うこととした。
\bunseki{柿崎大輝}

\section{エラー率の導入}
GA戦略が結果として最も低い所持金となった原因は、シミュレーションを行う条件にあった。GA戦略は複雑性が低い、つまりシンプルであることが最大の特徴である。そこで、我々はGA戦略の特徴がどのくらい有用かを調べるために複雑性を考慮したシミュレーションを行う必要があった。そして、そのシミュレーションを行うため、複雑性に関する実験の結果をもとにエラー率という概念を作成した。エラー率の定義は、ある戦略を使用する際にプレイヤーが行動を間違えてしまう確率を表したものである。このエラー率をシミュレーションに適用することにより、GA戦略を使用した際のプレイヤーの所持金が良い方向に改善される可能性があるのではないかと考えた。このエラー率の導出方法として、複雑性に関する実験データの正答率をもとに線形回帰分析を行った。また、使用した実験データについては以下の表\ref{hoge}で示す通りである。

  \begin{table}[H]
    \centering
    \caption{複雑性と誤答率}
    \begin{tabular}{|c|c|c|c|c|} \hline
      複雑性&0.2&0.2&0.407&0.448 \\ \hline
      エラー率&0.002&0.009&0.125&0.162 \\ \hline
    \end{tabular}
    \label{hoge}
  \end{table}
また、以上の実験データから線形回帰分析により得られた式は、

  \begin{equation}
    y=0.611x - 0.1172
  \end{equation}

となった。この式を横軸を複雑性、縦軸をエラー率としてグラフにしたものが図\ref{hoge2}である。

  \begin{figure}[H]
    \label{hoge2}
    \begin{center}
      \includegraphics[width=10cm]{figure/betsimulation-error_rate.png}
      \caption{線形回帰分析によるエラー率のグラフ}
    \end{center}
  \end{figure}
\bunseki{薩田凱斗}

\section{エラー率を導入したシミュレーション}
エラー率を導入して、シミュレーションを行う。エラー率は戦略に従って行動するときにミスをする確率で、ミスをすると戦略表とは異なる行動を実行するようにした。またカウンティング手法はKO法とHigh-Low法の2種類を用いた。それ以外の部分は前のシミュレーションと同じ条件とした。表\ref{err}が戦略ごとのエラー率で、図\ref{errKO}と図\ref{errHi-Lo}、表\ref{money-err}がシミュレーション結果である。
\begin{table}[H]
 \caption{戦略ごとのエラー率\label{err}}
 \begin{center}
  \begin{tabular}{|c|c|}
  \hline 戦略 & エラー率(\%) \\
  \hline BS & 16.2\\
  \hline BS-HS & 4.8 \\
  \hline GA戦略 & 0.5\\
  \hline
  \end{tabular}
 \end{center}
\end{table}

\begin{figure}[H]
 \begin{center} 
  \includegraphics[width=0.7\linewidth]{./figure/betsimulation-errKO}
  \caption{エラー率ありのKO法\label{errKO}}
 \end{center}
\end{figure}

\begin{figure}[H]
 \begin{center} 
  \includegraphics[width=0.7\linewidth]{./figure/betsimulation-errHi-Lo}
  \caption{エラー率ありのHigh-Low法\label{errHi-Lo}}
 \end{center}
\end{figure}

\begin{table}[H]
 \caption{エラー率を使用した際の所持金\label{money-err}}
 \begin{center}
  \begin{tabular}{|c|c|c|}
  \hline  & 40回目の平均所持金 & 標準偏差 \\
  \hline KO法BS & 923.303 & 174.255\\
  \hline KO法BS-HS & 963.799 & 150.135 \\
  \hline KO法GA戦略 & 974.817 & 144.821\\
  \hline High-Low法BS & 948.394 & 79.764\\
  \hline High-Low法BS-HS  & 975.440 & 67.263\\
  \hline High-Low法GA戦略 & 981.474 & 67.747\\
  \hline
  \end{tabular}
 \end{center}
\end{table}
エラー率を導入した結果ではGA戦略、ヒットとスタンドのみのベーシックストラテジー、ダブルダウンやスプリットが入ったベーシックストラテジーの順となり、所持金が1番多かったダブルダウンやスプリットが入ったベーシックストラテジーと1番少なかったGA戦略が入れ替わった。特にダブルダウンやスプリットが入ったベーシックストラテジーはエラー率を導入することで大きく所持金が少なくなった。またエラー率を導入したシミュレーションでは1000を超える戦略はないことが分かった\\
 エラー率を導入した場合、GA戦略が1番優れていることが分かった。しかし、GA戦略では1000を超えることができないことも分かった。この点にGA戦略は改善がする必要があると考えられる。
\bunseki{柿崎大輝}

\subsection{分散分析}
エラー率を導入したシミュレーションの時、3つの戦略間で40回目の平均所持金に有意な差があるかどうかを調べる。まず、3つの戦略において有意な差があるかどうかを調べるため、分散分析を行った。優位水準は5\%、KO法とHigh-Low法の2種類で行った。表\ref{conditions-b}で分散分析の条件をまとめた。
\begin{table}[H]
 \caption{分散分析の条件\label{conditions-b}}
 \begin{center}
  \begin{tabular}{|c|c|}
  \hline 帰無仮説 & 3つの戦略で平均所持金に有意な差はない \\
  \hline 対立仮説 & 3つの戦略で平均所持金に有意な差はある \\
  \hline 有意水準 & 5\% \\
  \hline
  \end{tabular}
 \end{center}
\end{table}
分散分析を行うと、KO法、High-Low法のp値がとても小さくなり、0.05以下となる。よって帰無仮説を棄却し、対立仮説を採択する。つまり、3つの戦略で有な差が存在することが確認できた。次は3つの戦略のどこに有意な差が存在するのかを調べるためにここから多重比較を行う。
\bunseki{柿崎大輝}

\subsection{多重比較}
それぞれの戦略で有意な差があるかを調べため、多重比較を行った。有意水準を5\%として行った。結果は表\ref{multiKO}と表\ref{multiHigh-Low}に示す。
\begin{table}[H]
 \caption{KO法での多重比較\label{multiKO}}
 \begin{center}
  \begin{tabular}{|c|c|c|}
  \hline  & BS & BS-HS  \\
  \hline  BS-HS 所持金の差 & 40.496 & \\
	               p値 & 0 & \\
  \hline GA 所持金の差 & 51.515 & 11.019\\
                p値 & 0 & 0\\
  \hline
  \end{tabular}
 \end{center}
\end{table}
\begin{table}[H]
 \caption{KO法での多重比較\label{multiHigh-Low}}
 \begin{center}
  \begin{tabular}{|c|c|c|}
  \hline  & BS & BS-HS  \\
  \hline  BS-HS 所持金の差 & 27.046 & \\
	               p値 & 0 & \\
  \hline GA 所持金の差 & 33.080 & 6.034\\
                p値 & 0 & 0\\
  \hline
  \end{tabular}
 \end{center}
\end{table}
 表\ref{multiKO}と表\ref{multiHigh-Low}ではそれぞれの戦略を比較した場合の所持金の差とp値を表示している。所持金がプラスの場合は列側の戦略のほうが所持金が多く、逆にマイナスなら行側の戦略のほうが所持金が多い。また、p値が0.05より小さい場合、その2つの戦略で有意な差が存在するということが分かる。\\
 KO法ではすべての戦略間でp値が0.05以下となり、すべての戦略間で有意な差が存在することが分かった。High-Low法でもすべての戦略間でp値が0.05以下となり、すべての戦略間で有意な差が存在することが分かった。これでKO法High-Low法の両方ですべて戦略間で有意な差が存在する。
\bunseki{柿崎大輝}

\chapter{まとめ}
\section{まとめ}
まず、本プロジェクトは、複雑性の実験により次の結果が得られた。
\begin{itemize}
    \item 結果1:複雑性とテストの成績には強い負の相関がある
    \item 結果2:CRTの正答率とテストの成績には相関がない
    \item 結果3:リスク回避性とテストの成績には相関がない
\end{itemize}

次に、ブラックジャックのシミュレータを作成した。作成にあたって、使用した疑似乱数を生成するメルセンヌツイスタが適切かどうか検証した。結果として、メルセンヌツイスタは周期が大きく、全て等しい確率で乱数生成していることが確認された。

次に、ベーシックストラテジーについて比較対象となる6つの戦略を考案し、検証を行った。これにより次の結果が得られた。
勝率のみを考慮した場合
\begin{itemize}
\item 結果1:一定の数字以上になるあでヒットする戦略よりも、ベーシックストラテジーとベーシックストラテジー改変1、ベーシックストラテジー改変2のほうが有意に高い勝率だった
\item 結果2:ベーシックストラテジーとベーシックストラテジー改変1、ベーシックストラテジー改変2のそれぞれの戦略間に有意な差はみられなかった
\item 結果3:ベーシックストラテジー改変1と18以上までヒットする戦略にはデック数無限とデック数1で勝率に有意な差があった
\end{itemize}

勝率を考慮すると、ベーシックストラテジーとベーシックストラテジー改変1、ベーシックストラテジー改変2の3つが勝率が高く、優秀な戦略であることが分かった。ただ、ベーシックストラテジー改変1はデック数の違いによって勝率に差がある。

複雑性を考慮して性能を評価した場合
\begin{itemize}
\item 結果4:15以上になるまでヒットする戦略が1番優秀であることが判明した。
\end{itemize}

複雑性を考慮すると、ベーシックストラテジーには改善の余地があることが判明した。

次に、遺伝的アルゴリズムを用いて新しい戦略を得た。得られた戦略とベーシックストラテジー、ダブルダウンやスプリットを含めたベーシックストラテジーと勝率を比較した。複雑性を考慮して新しい戦略を作成したが、比較の際には複雑性を考慮しなかったため、最も成績が悪いという結果となった。

次に、賭け金を導入してベーシックストラテジー、ダブルダウンやスプリットを含めたベーシックストラテジー、遺伝的アルゴリズムでえられた戦略で検証を行った。まずは賭け金を一定にした場合の所持金を調査した。結果として全ての戦略が初期所持金を下回った。次にカウンティングを用いることで利得を増やすことを試みた。結果として、既存のカウンティング手法を使うことで初期所持金を上回ることに成功した。次に新しいカウンティング手法を考案し検証した。結果として既存の手法より良い結果を得られることができなかった。

次に、複雑性を考慮してシミュレーションをするために、複雑性の実験データをもとにエラー率という概念を作成した。エラー率を適用してシミュレーションを行った結果、本プロジェクトが作成した、遺伝的アルゴリズムを用いて得られた戦略が最も優秀であるという結果が得られた。
\bunseki{轟木文弥}

\chapter{今後の課題}
我々が一年を通して活動した結果、今後の課題は以下の2点であると判断した。

\begin{itemize}
\item 利得をプラスにできる戦略の探索
\item ブラックジャックをプレイする際に重要な能力の調査
\end{itemize}

このプロジェクトで最終的に一番優秀と判断された戦略を用いてシミュレーションをしたところ、最終的なプレイヤーの利得はマイナスであった。また、賭け金の賭け方からアプローチするため、High-Low法とKO法というカウンティング手法を、基本戦略、ヒットとスタンドのみの基本戦略、今回導出されたGA戦略の3つの戦略を用いて賭け金のシミュレーションを行ったが、いずれも利得がプラスになることはなかった。

我々のプロジェクトの目的は、ディーラーをやっつける、すなわちカジノ側から利益を得ることが大本の目的であるため、まだ目的を達成したとは言えない。

なぜこのような結果となったのか、我々の予想としては、今回の遺伝的アルゴリズムのシミュレーションにはエラー率という概念が導入されていないためと思われる。

複雑性は単にその戦略の複雑さを数値にしただけであるが、エラー率はそれに加え、人間がどれだけ戦略を間違えやすいかという要素が含まれている。

戦略の勝率を導出するために作成したシミュレータには、エラー率の概念が導入されているが、その戦略を探索する遺伝的アルゴリズムのプログラムには導入されていなかった。つまり、戦略を作成する段階と、その戦略を評価する段階で、評価指標が異なるものであったのである。そのため、作成段階では優秀とされた戦略が、評価するにあたって性能が良くないと判断されてしまったのである。

以上のことより、利得をプラスにするためには、勝率が50%以上の戦略を見つけだしゲームを行うか、カウンティングの手法を改変して利益を出すようにしなければならない。

そこで今後の課題として、遺伝的アルゴリズムにエラー率の概念を導入し、より勝率の高い戦略を見つけ出すことと、カウンティングの手法の中で優秀なものを、今回使用した遺伝的アルゴリズム、もしくは他の技術を用いて探索する事が挙げられた。

また、今回の複雑性の検証実験により、ブラックジャックをプレイする際、戦略表を記憶する上で、一般的な認知的判断能力は関係がないことがわかった。
つまり、別の指標を用いて、戦略表を記憶する能力を測ることができる可能性があるということである。
その能力とは何であるのかを調査することが2つ目の課題として挙げられた。

\bunseki{菱田美紗紀}

\chapter{前期活動}
本プロジェクトで題材としているブラックジャックというゲームを
実際のプレイも交えて学習した。前章までで説明したベーシックストラテジーを
試すということもその中で行ったが、実際に使用してみると表を覚え、かつ
素早いゲームの進行に合わせながら実行するのは容易ではないという実感が得られた。
もちろんディーラー側に戦略の実行をさとられないようにするためには、ゲームの進行を
止めるなど違和感を持たせるような行動はできるだけ無くす必要がある。以上のことから、
戦略の単純化の必要性を再確認した。

シミュレータはPython3で作成を行った。前期までで得られたブラックジャックの戦略比較に用いた
数値はこのシミュレータによって得られた。また開発の効率化のためにバージョン管理システムである
Gitを導入し、Gitによるバージョン管理について学習した。導入の理由としては、プロジェクト学習で
プログラム開発を行う場合複数人で作業を行うためである。複数人の作業では、最新版のソースコードを
作業者同士で共有する必要がある。単純にプログラムファイル群をひとつにまとめて送付し合うという
方式もできるが、これではどの部分が変更されたのかの確認を行いにくい。Gitというシステムでは、誰が
どのようなタイミングでどのような変更を行ったかが記録されるため、エラーの解消を効率的に行うことができる。

また、中間報告書の作成作業にもGitを用いた。中間報告書の記述では、プログラムのソースコード以上に
文責者を明確にすることがプロジェクト学習として要求されている。また、報告書の記述はTeXであり、テキストファイル
であるため、Gitで管理することができる。単純にファイルをまとめて送付する場合よりも円滑な管理が行えたと思う。

以上のように、導入したバージョン管理システムGitはプロジェクト学習の多くの作業で有効だった。

シミュレータの正しさ、シミュレーション結果の分析のために統計学を学習した。

本プロジェクトにおいて、戦略の複雑性を評価することはとても重要な事項である。
複雑性の定義付けのためにChaitin(1969)によって定義されたコルモゴロフ複雑性の定義を参考にし、コルモゴロフ複雑性の定義と使われ方について調査し学習した。

本プロジェクトでは最適な戦略の探索を行うための技術の一つとしてニューラルネットワークを挙げ、斎藤(2016) の書籍を教科書とし、勉強会を行った。

\bunseki{米村祥裕}


%\chapter{後期目標}
%\input{./lategoal.tex}
\chapter{後期活動}
\input{./lateactivity.tex}
\chapter{中間発表の評価}
 本章では中間発表で記入してもらった評価シートの集計結果とコメントを参考にして今後の改善点を記述する。
 評価シートの評価項目は「発表技術」と「発表内容」の2つと「発表内容」の細部に「ブラックジャックのルール説明の評価」、「検証の評価」の2つ合わせて計4つ用意した。そして、それぞれについて1(非常に悪い)から10(非常に優秀)までの間で評価点を付け、それぞれについてのコメント(評価理由)やアドバイスを記入する欄を用意した。
\bunseki{※葛西隼人}
\section{中間発表}
\subsection{評価点数の集計}
中間発表で記入してもらった評価シートは計42枚だった。シートを記入した人の所属の分布の表\ref{tab:dist} のようになった。

\begin{table}[htb]
  \begin{center}
    \caption{評価人数集計}
    \begin{tabular}{|c|c|c|} \hline 
      所属 & 学年 & 人数  \\ \hline \hline
      教員 &  & 6  \\
      一般 &  & 0 \\
      学生 & 院2年 & 0 \\
     学生 & 院1年 & 0 \\
             & 学部4年 & 1 \\
       & 学部3年 & 34 \\
             & 学部2年 & 1 \\
             & 学部1年 & 0 \\ \hline \hline
      合計 &  & 42 \\ \hline
    \end{tabular}
    \label{tab:dist}
  \end{center}
\end{table}

評価人数の構成としては学部3年が大半を占めていた。その他には教員、学部4年と学部1年から1
人ずつであった。次はそれぞれの評価項目についての平均点を表\ref{tab:point}に記す。

\begin{figure}
\begin{center}
\caption{評価点数集計}
\begin{tabular}{|c|c|c|c|c|c|} \hline
  所属 & 学年 & 発表技術 & 発表内容 & ルール説明 & 検証説明  \\ \hline \hline
  教員 &        & 8 & 7.5 & 7.83 & 7.67 \\
  学生 &        & 6.47 & 7.03 & 7.56 & 6.22 \\
         & 学部2,4年 & 6 & 7 & 5.5 & 7 \\
         & 学部3年 & 6.38 & 7.01 & 7.68 & 6.18 \\ \hline \hline
  全体 &        & 6.69 & 7.1 & 7.6 & 6.43 \\ \hline
\end{tabular}
\label{tab:point}
\end{center}
\end{figure}

全体の平均は「発表技術」については6.69、「発表内容」については7.1、「ルール説明」については7.6、「検証説明」については6.43となった。それぞれの項目について高く評価したのは「教員」だった。
次に、それぞれの結果を図\ref{gizyutu}、図\ref{naiyou}、図\ref{ru-ru}、図\ref{kensyou}に示す。

\begin{figure}[h]
 \begin{tabular}{cc}
  \begin{minipage}[h]{0.45\hsize}
  \centering
 \includegraphics[width=0.7\linewidth]{./figure/gizyutu.jpg}
\caption{発表技術の評価グラフ}
\label{gizyutu}
 \end{minipage} &

\begin{minipage}[h]{0.45\hsize}
  \centering
 \includegraphics[width=0.7\linewidth]{./figure/naiyou.jpg}
 \caption{発表内容の評価グラフ}
\label{naiyou}
\end{minipage} 
\end{tabular}
\end{figure}

\begin{figure}[h]
 \begin{tabular}{cc}
  \begin{minipage}[h]{0.45\hsize}
  \centering
 \includegraphics[width=0.7\linewidth]{./figure/ru-ru.jpg}
\caption{ルール説明の評価グラフ}
\label{ru-ru}
 \end{minipage} &

\begin{minipage}[h]{0.45\hsize}
  \centering
 \includegraphics[width=0.7\linewidth]{./figure/kensyou.jpg}
 \caption{検証説明の評価グラフ}
\label{kensyou}
\end{minipage} 
\end{tabular}
\end{figure}
「発表技術」と「発表内容」については評価点が共に7、8点と高い評価点が多かった。ルール説明についても同様に高い評価点が多かった、一方で検証説明については4、5点が多い結果となった。

また、「発表技術」と「発表内容」の評点の相関係数は0.71となった。これは2つの評価項目がかなり関連してると言える数値である。次にコメントについて解析する。
\bunseki{※葛西隼人}

\subsection{コメント解析と改善点}
まず、「発表技術」について、肯定的なコメントと否定的なコメントに分けて並べる。

肯定的なコメント
\begin{itemize}
\item 複雑な内容をうまく説明してくれた
\item スライド内で図を多用されていて分かりやすかった
\end{itemize}

否定的なコメント
\begin{itemize}
\item 前を見て話してもらえないと聞き取りにくい
\item 説明が少し早い
\item プレゼンの内容が構造的でないために、全体に対して今どの部分を話しているのかわかりにくい
\end{itemize}

今回はスライドの分量が多く、どうしても早口になってしまったのでこのような意見が多かった。そしてスライドの完成が遅かったために構成までについて詳しく話し合うことができなかった。また発表の練習時間が取れなかったために技術面で足りない点もいくつかコメントで述べられていた。
次に、「発表内容」についても同様に、コメントを肯定的なコメントと否定的なコメントに分けて並べる。
肯定的なコメント
\begin{itemize}
\item 活動内容が明確でよかった
\item 実験・検証が多く、説得力をもたせている
\item 評価基準や比較対象が明確で理解しやすかった
\end{itemize}

否定的なコメント
\begin{itemize}
\item 用語についてより詳しく説明すべき
\item 聞く人の知識が必要になるのでもう少しわかりやすく
\item 検証結果のみせ方にもう少し工夫があると良かった。(専門的な知識がない人にもわかりやすく)
\end{itemize}

「発表内容」に関しては、ブラックジャックのルール説明はわかりやすいとのコメントが多かった。しかし検証の説明はあまり理解出来ないとのコメントも見受けられた。最後に「発表内容」と「発表技術」の2つのコメント欄から重要なアドバイスがいくつかあったため、これらについても述べていく。 
\begin{itemize}
\item なぜあのような式を利用することで性能が表されているかの説明がもう少し欲しかった
\item 文字数を減らし簡潔な内容の方がいいと思う
\item スライドに色を使って見やすくしたほうが良い
\item 基本的に文字がたくさんのスライドで、太字や下線などもなかったので、どこに注目してよいかわからなかった。後期の発表ではもっとスライドを効果的に見せてほしい
\end{itemize}
これらの意見に関しても、後期の活動に反映することとする。
以上より中間発表の評価コメントは賛否両論であり、とても参考になった。
\bunseki{※葛西隼人}
\chapter{最終成果発表の評価}
 本章では最終成果発表で記入してもらった評価シートの集計結果とコメントを参考にして今後の改善点を記述する。
 評価シートの評価項目は「発表技術」と「発表内容」の2つと「発表内容」の細部に「複雑性にちての説明の評価」、「カウンティングについての説明の評価」の2つ合わせて計4つ用意した。そして、それぞれについて1(非常に悪い)から10(非常に優秀)までの間で評価点を記入する欄、それぞれについてのコメント(評価理由)やアドバイスを記入する欄をそれぞれ用意した。
\bunseki{葛西隼人}
\section{最終成果発表}
\subsection{評価点数の集計}
最終成果発表で記入してもらった評価シートは計50枚だった。シートを記入した人の所属の分布は表\ref{tab:dist2} のようになった。

\begin{table}[htb]
  \begin{center}
    \caption{評価人数集計}
    \begin{tabular}{|c|c|c|} \hline 
      所属 & 学年 & 人数  \\ \hline \hline
      教員 &  & 7  \\
      一般 &  & 8 \\
      学生 & 院2年 & 0 \\
     学生 & 院1年 & 0 \\
             & 学部4年 & 0 \\
       & 学部3年 & 31 \\
             & 学部2年 & 2 \\
             & 学部1年 & 2 \\ \hline \hline
      合計 &  & 50 \\ \hline
    \end{tabular}
    \label{tab:dist2}
  \end{center}
\end{table}

中間発表では一般の評価人数が0人だったが、今回は函館市内の高校生が来ていたため一般の評価人数は増えて8人だった。学生と教員に関しては学部4年が0人となった代わりに、学部1年が0人となっていた点以外は中間発表とほとんど差異のない構成人数であった。次はそれぞれの評価項目についての平均点を表\ref{tab:point2}に記す。
\begin{table}[H]
\begin{center}
\caption{評価点数集計}
\begin{tabular}{|c|c|c|c|c|c|} \hline
  所属 & 学年 & 発表技術 & 発表内容 & 複雑性説明 & カウンティング説明  \\ \hline \hline
  教員 &        & 7.42 & 7.14 & 7.14 & 6.71 \\ 
  一般 &        & 7.38 & 7 & 6.12 & 6.75 \\
  学生 &        & 7 & 7.2 & 6.66 & 7.02 \\
         & 学部1,2年 & 6.5 & 6.25 & 6.5 & 6.5 \\
         & 学部3年 & 6.87 & 7.39 & 6.7 & 7.26 \\ \hline \hline
  全体 &        & 7 & 7.2 & 6.66 & 7.02 \\ \hline
\end{tabular}
\label{tab:point2}
\end{center}
\end{table}

全体の平均は「発表技術」については7点、「発表内容」については7.2点、「複雑性説明」については6.66点、「カウンティング説明」については7.02点となった。中間発表と最終成果発表で「発表技術」のみ比較を行う。中間発表では6.69点、最終成果発表では7点となった。点数のみの比較でも0.31点高くなっている。これは中間発表からの技術改善が見られていると考えられる。「発表内容」については中間発表と異なるので比較は行わない。次に、それぞれの結果を図\ref{gizyutu2}、図\ref{naiyou2}、図\ref{hukuzatusei}、図\ref{counting}に示す。

\begin{figure}[h]
 \begin{tabular}{cc}
  \begin{minipage}[h]{0.45\hsize}
  \centering
 \includegraphics[width=0.7\linewidth]{./figure/gizyutu2.jpg}
\caption{発表技術の評価グラフ}
\label{gizyutu2}
 \end{minipage} &

\begin{minipage}[h]{0.45\hsize}
  \centering
 \includegraphics[width=0.7\linewidth]{./figure/naiyou2.jpg}
 \caption{発表内容の評価グラフ}
\label{naiyou2}
\end{minipage} 
\end{tabular}
\end{figure}

\begin{figure}[h]
 \begin{tabular}{cc}
  \begin{minipage}[h]{0.45\hsize}
  \centering
 \includegraphics[width=0.7\linewidth]{./figure/hukuzatusei.jpg}
\caption{複雑性説明の評価グラフ}
\label{hukuzatusei}
 \end{minipage} &

\begin{minipage}[h]{0.45\hsize}
  \centering
 \includegraphics[width=0.7\linewidth]{./figure/counting.jpg}
 \caption{カウンティング説明の評価グラフ}
\label{counting}
\end{minipage} 
\end{tabular}
\end{figure}

「発表技術」と「発表内容」については評価点が共に7、8点と高い評価点が多かった。そして、「発表内容」に関しては最も低くても5点と全体的には評価はまとまっていたが、他のグラフは1,3
点など低い点数があり全体的な評価は散らばっていた結果となった。「複雑性説明」では5~8点の間にヒストグラムが集中していて理解度に差あることがわかった。

また、「発表技術」と「発表内容」の評価点数の相関係数は0.73となった。これは強い正の相関があり、2つの評価項目がかなり関連してると言える数値である。次にコメントについて解析する。
\bunseki{葛西隼人}

\subsection{コメント解析と改善点}
まず、「発表技術」について、肯定的なコメントと否定的なコメントに分けて並べる。

肯定的なコメント
\begin{itemize}
\item アニメーションがついてて見やすいスライドだと思った
\item アジェンダがあり、長いプレゼンがわかりやすかった
\item スライドの内容はよく整理されておりわかりやすかったです
\end{itemize}

否定的なコメント
\begin{itemize}
\item 発表場所が悪かったのもあり声がきこえないところがあった
\item 発表者が原稿を読んでいる感じがしたのが残念だった
\item 発表が単調で何が大事なポイントが分かりづらかった
\end{itemize}

スライド構成については評価するコメントはあったが、声が小さい、スライドの文字ばかり見ているとの発表者の技術面についての否定的な意見が多かった。そして発表の時間配分が他のプロジェクトと異なっていたため発表を全て聞けなかった人が少しいた。次に、「発表内容」についても同様に、肯定的なコメントと否定的なコメントに分けて並べる。

肯定的なコメント
\begin{itemize}
\item 仮説・実験・検証が適切に行われていた
\item 実験・検証が多く、説得力をもたせている
\item プロジェクト目標の設定が明確であった
\item 人間が使うことも含めた戦略の考察が面白いと思った
\end{itemize}

否定的なコメント
\begin{itemize}
\item エラー率は単純に複雑性から出すのではなく、似た状況がある時エラーするほうが正確なデータが出そうだと思った。
\item ポスターにもスライドにもチームとしてどう取り組んだのか全く示されていないのは、プロジェクト学習の発表として不十分に思えます
\end{itemize}

そして要望としてのあったコメントをいくつかあったので記述する。
\begin{itemize}
\item 勝率を100パーセントにしてほしい
\item 他のプロジェクトの発表もあるので、時間配分は考えてほしい
\end{itemize}

中間発表では、用語の説明が足りないというコメントが多く見受けられたが、最終成果発表ではカウンティングと複雑性の両方でわかりやすいというコメントがあった。今回の発表を通して得られた発表での留意点を以下に記述する。
\begin{itemize}
\item スライドは色を用いて見やすくする
\item 太線や下線などの重要な箇所を視覚的に理解できるようにする
\item 発表者は声は大きく、はっきり聞き取りやすく来ている人に目を向ける
\item 用語の説明を入れる、構成はアジェンダを用いてわかりやすくする
\end{itemize}

本プロジェクトでの中間発表と最終成果発表は今後の研究活動にも参考になる点が数多くあり非常に有益なものであった。
\bunseki{葛西隼人}

\input{reference/reference.tex}

\chapter{付録}
\lstset{ 
   basicstyle={\ttfamily\small}, %書体の指定 
   frame=tRBl, %フレームの指定 
   framesep=10pt, %フレームと中身(コード)の間隔 
   breaklines=true, %行が長くなった場合の改行 
   linewidth=12cm, %フレームの横幅 
   lineskip=-0.5ex, %行間の調整 
   tabsize=2 %Tabを何文字幅にするかの指定 
}

\begin{itemize}
\item{シミュレータに使われるクラス}
\input{appendix/crass_Card.tex}
\newpage
\input{appendix/crass_Deck.tex}
\newpage
\input{appendix/crass_Dealer.tex}
\newpage
\begin{itemize}
\item ゲームの勝敗や掛け金の受け渡しを管理するクラス
\begin{lstlisting}
class GameManager:
    def __init__(self, players, dealer):
        self.players = players
        self.dealer = dealer
        self.checkdeal = True

    # 各プレイヤーとディーラーとの間で勝敗を決める
    def judge(self):
        for x in self.players:
            self.checkblackjack(x)
        self.checkblackjack(self.dealer)
        for player in self.players:
            if not player.surrendeflg:
                # プレイヤーがバーストした場合
                if player.burst == True:
                    if player.tag == "clone":
                        for i, x in enumerate(self.players):
                            if x.name == player.name:
                                self.players[i].addtotallose(player.betMoney)
                                break
                    player.addtotallose(player.betMoney)

                # プレイヤーがバーストせずにディーラーがバーストした場合
                elif player.burst == False and self.dealer.burst == True:
                    # スプリットしているかどうかのフラグ
                    spflg = False
                    for x in self.players:
                        if x.tag == "clone":
                            spflg = True

                    if player.tag == "clone":
                        for i, x in enumerate(self.players):
                            if x.name == player.name:
                                if player.naturalbj and not spflg:
                                    self.players[i].addtotalwin(player.betMoney*1.5)
                                    break
                                else:
                                    self.players[i].addtotalwin(player.betMoney)
                                    break
                    if player.naturalbj and not spflg:
                        player.addtotalwin(player.betMoney*1.5)
                    else:
                        player.addtotalwin(player.betMoney)

                # プレイヤーのトータルがディーラーのトータルよりも多い場合
                elif player.total > self.dealer.total:
                    spflg = False
                    for x in self.players:
                        if player.tag=="clone":
                            spflg = True

                    if player.tag == "clone":
                        for i, x in enumerate(self.players):
                            if x.name == player.name:
                                if player.naturalbj and not spflg:
                                    self.players[i].addtotalwin(player.betMoney*1.5)
                                    break
                                else:
                                    self.players[i].addtotalwin(player.betMoney)
                                    break
                    if player.naturalbj and not spflg:
                        player.addtotalwin(player.betMoney*1.5)
                    else:
                        player.addtotalwin(player.betMoney)

                # プレイヤーのトータルがディーラーのトータルよりも少ない場合
                elif player.total < self.dealer.total:
                    if player.tag == "clone":
                        for i, x in enumerate(self.players):
                            if x.name == player.name:
                                self.players[i].addtotallose(player.betMoney)
                                break
                    player.addtotallose(player.betMoney)

                # プレイヤーのトータルとディーラーのトータルが同じ場合
                elif player.total == self.dealer.total:
                    # プレイヤーがナチュラルブラックジャックかつディーラーがナチュラルブラックジャック
                    if player.naturalbj and self.dealer.naturalbj:
                        if player.tag == "clone":
                            for i, x in enumerate(self.players):
                                if x.name == player.name:
                                    self.players[i].addtotaldraw()
                                    break
                        player.addtotaldraw()
                    # プレイヤーがナチュラルブラックジャックかつディーラーがノーマルブラックジャック
                    elif player.naturalbj and self.dealer.normalbj:
                        if player.tag == "clone":
                            for i, x in enumerate(self.players):
                                if x.name == player.name:
                                    self.players[i].addtotalwin(player.betMoney * 1.5)
                                    break
                        player.addtotalwin(player.betMoney * 1.5)
                    # プレイヤーがノーマルブラックジャックかつディーラーがナチュラルブラックジャック
                    elif player.normalbj and self.dealer.naturalbj:
                        if player.tag == "clone":
                            for i, x in enumerate(self.players):
                                if x.name == player.name:
                                    self.players[i].addtotallose(player.betMoney)
                                    break
                        player.addtotallose(player.betMoney)
                    # プレイヤーがノーマルブラックジャックかつディーラーがノーマルブラックジャック
                    elif player.normalbj and self.dealer.normalbj:
                        if player.tag == "clone":
                            for i, x in enumerate(self.players):
                                if x.name == player.name:
                                    self.players[i].addtotaldraw()
                                    break
                        player.addtotaldraw()
                    else:
                        if player.tag == "clone":
                            for i, x in enumerate(self.players):
                                if x.name == player.name:
                                    self.players[i].addtotaldraw()
                                    break
                        player.addtotaldraw()

    # ナチュラルブラックジャックとノーマルブラックジャックを判別する関数
    # 入力にプレイヤー個人またはディーラ-個人を与える
    def checkblackjack(self, player):
        if player.total == 21:
            if len(player.cards) == 2:
                player.naturalbj = True
            else:
                player.normalbj = True

\end{lstlisting}
\end{itemize}
\newpage
\newpage
\begin{itemize}
\item ゲーム参加者を表すスーパークラス
\begin{lstlisting}
class GamePlayer:

    # 初期化関数
    def __init__(self):
        self.cards = []
        self.total = 0
        self.acetotal = 0
        self.usedace = 0
        self.burst = False
        self.naturalbj = False
        self.normalbj = False
        self.surrendeflg = False

    #  子オブジェクトから呼び出せる初期化関数
    def initialize(self):
        self.cards = []
        self.total = 0
        self.acetotal = 0
        self.usedace = 0
        self.burst = False
        self.naturalbj = False
        self.normalbj = False
        self.surrendeflg = False

    # ゲームプレイヤーの手札の合計値を返す関数
    def totalvalue(self):
        i = 0
        self.total = 0
        self.acetotal = 0
        cardnum = len(self.cards)

        while i < cardnum:
            if self.cards[i].rank == 'A':
                self.acetotal += 1
            self.total += self.cards[i].value
            i += 1
        self.total -= 10 * self.usedace

        # プレイヤーのバースト判定の処理
        if self.total > 21:
            if self.acetotal - self.usedace > 0:
                self.total -= 10
                self.usedace += 1
                if self.total > 21:
                    self.burst = True
            else:
                self.burst = True

\end{lstlisting}
\end{itemize}
\newpage
\input{appendix/crass_Player.tex}
\newpage
\input{appendix/main_program.tex}
\end{itemize}
\lstinputlisting[caption=geneticalgorithm.py,label=gasource]{appendix/geneticalgorithm/src/geneticalgorithm.py}
\lstinputlisting[caption=searchbyga.py,label=searchbyga]{appendix/geneticalgorithm/src/searchbyga.py}

%\input{appendix/crass_Card.tex}
%\input{appendix/crass_Deck.tex}
%\begin{itemize}
\item ゲーム参加者を表すスーパークラス
\begin{lstlisting}
class GamePlayer:

    # 初期化関数
    def __init__(self):
        self.cards = []
        self.total = 0
        self.acetotal = 0
        self.usedace = 0
        self.burst = False
        self.naturalbj = False
        self.normalbj = False
        self.surrendeflg = False

    #  子オブジェクトから呼び出せる初期化関数
    def initialize(self):
        self.cards = []
        self.total = 0
        self.acetotal = 0
        self.usedace = 0
        self.burst = False
        self.naturalbj = False
        self.normalbj = False
        self.surrendeflg = False

    # ゲームプレイヤーの手札の合計値を返す関数
    def totalvalue(self):
        i = 0
        self.total = 0
        self.acetotal = 0
        cardnum = len(self.cards)

        while i < cardnum:
            if self.cards[i].rank == 'A':
                self.acetotal += 1
            self.total += self.cards[i].value
            i += 1
        self.total -= 10 * self.usedace

        # プレイヤーのバースト判定の処理
        if self.total > 21:
            if self.acetotal - self.usedace > 0:
                self.total -= 10
                self.usedace += 1
                if self.total > 21:
                    self.burst = True
            else:
                self.burst = True

\end{lstlisting}
\end{itemize}

\end{document}
