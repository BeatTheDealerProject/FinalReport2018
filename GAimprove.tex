
\subsection{SGAの概要とその問題点}
この項ではGAの問題点について説明していく。佐藤ら(1997)は代表的なGAでの世代交代モデルとしてSimple GA(以下SGAと表記する)を挙げ、そのモデルの問題点と改善案を示している。ただし、ここでは適応度をもとにした選択処理についての問題点と改善案を挙げており、交叉アルゴリズムや遺伝子のコード設計、各種パラメータの設定に関しては扱っていない。

世代交代のモデルには、次世代の個体を生成するための親を選択する複製選択と、全ての個体の中から次世代に残す個体を選択する生存選択の2種類の処理が存在する。SGAの場合この2つの選択処理は次のようになっている。

\begin{itemize}
\item{複製選択}\\
適応度に比例した選択確率を用いたルーレット選択方式によって、集団から個体を復元抽出する。
復元抽出とは一度選択された個体も次以降の選択対象に含める、つまり同じ個体が複数回選ばれることを許している選択方法である。
\item{生存選択}\\
無条件で親集団と生成された子集団のすべてを入れ替える。
\end{itemize}

佐藤ら(1997)によるとSGAには3つの問題点があることが指摘されている。
\begin{enumerate}
\item{高い選択圧下での早期収束}
\begin{itemize}
\item{SGAのように適応度を用いてルーレット選択を行っている場合、探索の初期に適応度が突出した個体が存在するとその個体が複製選択において選ばれる可能性が高くなりすぎてしまう。そうなった場合、探索の序盤からその個体に遺伝子全体が収束してしまう現象が起きてしまい、最適な解にたどりつきづらくなってしまう。この現象は初期収束と呼ばれている。}
\end{itemize}
\item{低い選択圧下での停滞}
\begin{itemize}
\item{GAの探索が進み、各世代の個体間の適応度に差が見られなくなってきた時に複製選択においてルーレット選択の効果が弱くなることがある。そうなると個体が最適な解に向かって進みづらくなってしまう現象が起こることがある。ここでは進化的停滞と呼んでいる。}
\end{itemize}
\item{優秀な遺伝子の破壊}
\begin{itemize}
\item{SGAでは親から子に世代が移る時は無条件で全ての遺伝子を入れ替えてしまう。そのため親個体に適応度の高い個体が存在する場合でもその個体は次の世代では失われてしまう。そのため集団から適応度の高い個体が失われてしまう可能性がある。}
\end{itemize}
\end{enumerate}


\subsection{今回設計したGAのまとめと改善案}
\subsubsection{まとめ}
今回我々が作成したGAではエリート保存方式を採用している。そのため上記の問題点で挙げている優秀な遺伝子の破壊が起きる可能性は低いと考えられる。図\ref{gaprocess}からもわかる通り最優秀個体の適応度は徐々に高くなっている。\\
問題点としては、初期個体にブラックジャックの基本戦略、14以上でスタンドする戦略、15以上でスタンドする戦略、16以上でスタンドする戦略をそれぞれ5個体ずつ設定したことが挙げられる。今回はGAの実行時間を短縮するため、また最適な解を早く発見するため、戦略として優秀な基本戦略とそれを改変した戦略を初期個体に混ぜた。そのため個体全体がそれらの戦略に初期収束してしまった可能性がある。これはGAから出力された戦略表からもわかる。GAから出力された戦略表はハードハンドでは14以上でスタンドする戦略がそのまま出力され、ソフトハンドでは基本戦略から1つの遺伝子を変化させたものになっている。


\subsubsection{改善案}
改善案としては、初期個体に基本戦略などを加えず完全にランダムにする、もしくは初期個体に含める基本戦略などの数を減らすことが考えられる。しかし初期値に補正を加えない場合、HとSが完全にランダムになり、複雑性が高くなりすぎてしまう可能性が考えられる。そのため解の探索が上手くいかず、優秀な戦略が得られないことも考えられる。\\
もう一つの案として、作成したGAの世代数を増やすことも考えられる。図\ref{gaprocess}の適応度推移をみると最優秀個体の適応度は段々と増加している。そのため世代数を増やし、長く実行することで今回得られた戦略より適応度の高い戦略が得られる可能性も考えられる。
その他細かい改善点としては各種パラメータの調整、選択アルゴリズムを変更することなどが挙げられる。
\bunseki{伊藤晋之介}