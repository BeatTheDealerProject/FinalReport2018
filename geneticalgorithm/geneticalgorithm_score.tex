\section{GA戦略の戦績}
\subsection{シミュレーション結果}
ここでは、遺伝的アルゴリズムから得られた戦略(以下、GA戦略)、ダブルダウンやスプリットが考慮されているベーシックストラテジー(以下、BS)、ヒットとスタンドのみで構成されたベーシックストラテジー(以下、BS-HS)の3つの戦略のシミュレーションの結果(※の比較)について述べる。なお、今回のシミュレーションではデック数は無限に設定し、10万回ゲームを行った。その結果を表\ref{tab:score}に示す

\begin{table}[H]
 \caption{デック数無限と各戦略での勝ち、負け、引き分けの比率}
 \begin{center}
  \begin{tabular}{|c|c|c|c|}
  \hline   & 勝ち(\%) & 負け(\%)  & 引き分け(\%) \\
  \hline BS & 43.762 & 47.566 & 8.671 \\
  \hline BS-HS & 42.746 & 48.635 & 8.619 \\
  \hline GA戦略 & 42.858 & 49.502 & 7.64 \\
  \hline
  \end{tabular}
  \label{tab:score}
 \end{center}
\end{table}

今回は勝ち、負け、引き分けの3種類に分けて、比率で表したものが表\ref{tab:score}となっている。勝ちの比率が最も高いのはBSとなった。そしてGA戦略は勝率では BS-HSを上回っていたが、負けの比率が最も高いのはGA戦略となった。
\bunseki{※葛西隼人}

\subsection{カイ2乗検定}
先程のシミュレーションを行った結果から、勝率が最も高いのはBSとなったが、勝ち負け引き分けに優位な差があるのではないかと考えられる。これを確かめるために、カイ2乗検定を行い、勝ち負け引き分けに優位な差があるかどうかを確かめる。
 帰無仮説は勝ち負け引き分けに有意な差がない。対立仮説は勝ち負け引き分けに有意な差があるとし、カイ2乗検定を行う前に必要な条件を表\ref{score-x}で必要な条件をまとめた。
\begin{table}[H]
 \caption{勝ち負け引き分けのカイ2乗検定条件\label{score-x}}
 \begin{center}
  \begin{tabular}{|c|c|}
  \hline 帰無仮説 & 勝ち負け引き分けに有意な差がない \\
  \hline 対立仮説 & 勝ち負け引き分けに有意な差がある \\
  \hline 有意水準 & 5\% \\
  \hline 自由度 & 2 \\
  \hline 棄却値 & 5.99 \\
  \hline
  \end{tabular}
 \end{center}
\end{table}

棄却値は5.99である。この棄却値よりもカイ2乗値が大きい場合、帰無仮説を棄却して対立仮説が採択される。カイ2乗検定を行うとカイ2乗値は66.754となった。この値をp値に変換すると0.1よりちいさくなった。よって帰無仮説を棄却して対立仮説を採択する。つまり、勝ち負け引き分けに優位な差が存在する。
\bunseki{※葛西隼人}

\subsection{残差分析}
カイ2乗検定によって各戦略間の勝ち負け引き分けに優位な差が存在することが判明した。しかしどこに優位な差が存在するかまでは明らかになっていない。そのためさらに検定を行いどの戦略に優位な差があるかを探す。そのために残差分析を行う。残差分析では調整済み標準化残差を算出し、調整済み標準化残差が1.96より大きい場合と-1.96より小さい場合に優位な差があると分かる。調整済み標準化残差をそれぞれ出したものを表\ref{score-zansa}でまとめる。

\begin{table}[H]
 \caption{調整済み標準化残差}
 \begin{center}
  \begin{tabular}{|c|c|c|c|}
  \hline   & 勝ち(\%) & 負け(\%)  & 引き分け(\%) \\
  \hline BS & 5.051 & -7.83 & 5.113 \\
  \hline BS-HS & -2.98 & 0.59 & 4.277 \\
  \hline GA戦略 & -2.016 & 7.293 & -9.425 \\
  \hline
  \end{tabular}
  \label{tab:score-zansa}
 \end{center}
\end{table}

表\ref{score-zansa}ではそれぞれ戦略での勝ち、負け、引き分けの3つの調整済み標準化残差を示している。例えば、BSの勝ちの項目を見ると値が5.051となっており、1.96以上なのでBSの勝ちの比率は他の戦略より優位に多かったということがわかる。

GA戦略を見ると負けの比率が1.96を上回っており他の戦略より優位に多かったということがわかる。逆に、勝ちと引き分けの比率がどちらも-1.96を下回っており優位に少ないことがわかった。

以上の結果からGA戦略は勝ちの比率はBS-HSと同じだが、引き分けの比率が少なくなり、負けの比率が多くなってしまった事がわかる。
\bunseki{※葛西隼人}