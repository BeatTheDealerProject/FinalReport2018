\section{カウンティングについて}
カウンティングとは、ブラックジャックにおいてゲーム内で使用されているデックの状態がプレイヤーにとって有利な状態か、あるいは不利な状況であるかを見極めるための高等テクニックである。プレイヤーにとって有利な時には、リスクを最小限に抑えつつ、より多くの金額を賭けることができる。また、カウンティングを使用することによってデック内に残っているカードの構成に基づいて、プレイヤーは自身の行動選択を変更させることも可能である。エドワード・ソープ氏が最初に発明して以来そのカウンティングの手法ついては数多くの種類が誕生してきたが、本プロジェクトではHigh-Low法とKO法と呼ばれる2種類のカウンティング手法を使用してシミュレーションなどを行った。ここでは、そのHigh-Low法とKO法の2つについて説明する。
\bunseki{※薩田凱斗}

\subsection{High-Low法}
まずはHigh-Low法について説明する。High-Low法の特徴はランニングカウントと同じ分のユニットを賭けるということである。ランニングカウントとは、各カードに割り当てられた値の暫定的な合計値のことである。自分自身の手札の値以外にも、他のプレイヤーのカードの値とディーラーのアップカードの値も含む。そして、ユニットとは自分が決めるベット額の単位のことである。一般的には、ミニマムベットを基準にする場合が多い。例えば、あるカジノのブラックジャックのテーブルでのミニマムベットが5ドルだとすると、ユニットも5ドルに設定する。ランニングカウントを複数デックで使用する場合には、トゥルーカウントというものを算出する必要がある。トゥルーカウントは、ランニングカウントを残りデック数で割ったものである。例えば、ランニングカウントが10で残りデック数が5の時、トゥルーカウントは10割る2で5となる。では、なぜトゥルーカウントを使用する必要があるのかを説明する。仮にランニングカウントが+5で残りデック数が6の場合とランニングカウントが+5で残りデック数が1の場合は全く違った状況となる。前者はデックごとに1枚のハイカードが残っているため、プレイヤーにとっては有利な状況とは言えない。後者は、残り1デック、つまり52枚のうち10とAが5枚残っておりプレイヤーにとって有利な状況であることがわかる。High-Low法のカウント方法は以下の\ref{hlc}のようにカードの値が2から6の時にはランニングカウントをプラス1、7から9の時はランニングカウントを0、そして10、Aの時にはマイナス1をランニングカウントに加えていくというものである。また、High-Low法においては、初期ランニングカウントを0から開始する。
\begin{table}[H]
\centering
\begin{tabular}{|c|c|c|c|c|c|c|c|c|c|}
\caption{High-Low法でのカードカウント\label{hlc}}
\hline
2&3&4&5&6&7&8&9&10&A \\ \hline
+1&+1&+1&+1&+1&0&0&0&-1&-1 \\ \hline
\end{tabular}
\end{table}
このランニングカウントがプラスの場合は、10やAなどのカードがデック内に残っているためプレイヤーにとって手札の合計が21となるブラックジャックとなる可能性が高く、非常に有利になるためベッド額を増やしていき、反対にランニングカウントがマイナスの時は不利な状況なためベッド額を抑えるといったものである。また、以下の\ref{hlb}はHigh-Low法を使用した際のベットシステムである。
\begin{table}[H]
\centering
\caption{High-Low法でのベットシステム\label{hlb}}
\begin{tabular}{|c|c|c|c|c|c|c|c|c|c|c|}
\hline
ランニングカウント&-5以下&-4&-3&-2&-1&0&+1&+2&+3&+4以上 \\ \hline
ベット(ユニット)&1&2&3&4&5&6&7&8&9&10 \\ \hline
\end{tabular}
\end{table}
先述した通り、High-Low法はカウント値と同じユニットを賭けていく。
\bunseki{※薩田凱斗}

\subsection{KO法}
次にKO法についてである。KO法は以下の\ref{koc}のようにカードの値が2から7の時にはランニングカウントをプラス1、8と9の時はランニングカウントを0、そして10、Aの時にはマイナス1をランニングカウントに加えていくというものである。High-Low法との唯一の違いは7をローカードとして扱っているため、カウント値が0ではなく+1となっている。そのため、それぞれのカウント値を足し合わせると+4となる。これはアンバランスシステムとも呼ばれ、High-Low法のようにトゥルーカウントを算出する必要がない。
\begin{table}[H]
\centering
\begin{tabular}{|c|c|c|c|c|c|c|c|c|c|}
\caption{KO法でのカードカウント\label{koc}}
\hline
2&3&4&5&6&7&8&9&10&A \\ \hline
+1&+1&+1&+1&+1&+1&0&0&-1&-1 \\ \hline
\end{tabular}
\end{table}

KO法の大きな特徴としてはデック数に応じて、初期ランニングカウントを変えていくということである。カジノで使用されるデック数で一般的な6デックを例にとると、初期ランニングカウントは-20である。仮に6デックの場合には合計で24枚のAがあり、そこから1デック分のA、つまり4枚を抜いた数を初期ランニングカウントとして扱う。なので1デックの場合の初期ランニングカウントは0、同様に2デックの場合は-4、6デックの場合は-20が初期ランニングカウントとなる。以下の\ref{kob}はKO法を使用した際のベットシステムである。

\begin{table}[H]
\begin{tabular}{|c|c|c|c|c|c|c|c|c|c|c|}
\caption{KO法でのベットシステム\label{kob}}
\hline
ランニングカウント&-5以下&-4&-3&-2&-1&0&+1&+2&+3&+4以上 \\ \hline
ベット(ユニット)&1&2&2&3&4&5&6&8&9&10 \\ \hline
\end{tabular}
\end{table}
High-Low法との違いはランニングカウントが-3の時に2ユニットを賭け、ランニングカウントが+1になるまではHigh-Low法と1ユニット分減らしたベットシステムとなっている。
\bunseki{※薩田凱斗}