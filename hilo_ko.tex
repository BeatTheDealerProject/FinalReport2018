\section{カウンティングについて}
カウンティングとは、ブラックジャックにおいてゲーム内で使用されているデックの状態がプレイヤーにとって有利な状態か、あるいは不利な状況であるかを見極めるための高等テクニックである。プレイヤーにとって有利な時には、リスクを最小限に抑えつつ、より多くの賭けをすることができる。また、カウンティングを使用することによってデックに残っているカードの構成に基づいてプレイヤーの選択を変更させることも可能である。そのカウンティングの手法については数多く存在するが、本プロジェクトではHigh-Low法とKO法と呼ばれる2種類のカウンティング手法を使用してシミュレーションなどを行った。ここでは、そのHigh-Low法とKO法の2つについて説明する。まずはHigh-Low法について説明する。High-Low法の特徴はランニングカウントと同じ分のユニットを賭けるということである。ランニングカウントとは、各カードに割り当てられた値の暫定的な合計値のことである。そして、ユニットとはミニマムベットのことである。例えば、あるカジノのブラックジャックのテーブルでのミニマムベットが5ドルだとすると、その場合のユニットは5ドルとなる。また、以下の\ref{hlc}のようにカードの値が2から6の時にはランニングカウントをプラス1、7から9の時はランニングカウントを0、そして10、Aの時にはマイナス1をランニングカウントに加えていくというものである。また、High-Low法においては、初期ランニングカウントを0から開始する。

  \begin{table}[H]
    \centering
    \begin{tabular}{|c|c|c|c|c|c|c|c|c|c|} \hline
    2&3&4&5&6&7&8&9&10&A \\ \hline
    +1&+1&+1&+1&+1&0&0&0&-1&-1 \\ \hline
    \end{tabular}
    \caption{High-Low法でのカードカウント}
    \label{hlc}
  \end{table}

ランニングカウントがプラスの場合は、10やAなどのカードがデック内に残っているためプレイヤーにとって手札の合計が21となるブラックジャックとなる可能性が高く、非常に有利になるためベッド額を増やしていき、反対にランニングカウントがマイナスの時はベッド額を抑えるといったものである。以下の\ref{hlb}はHigh-Low法を使用した際のベットシステムである。

  \begin{table}[H]
    \centering
    \begin{tabular}{|c|c|c|c|c|c|c|c|c|c|c|} \hline
      ランニングカウント&-5以下&-4&-3&-2&-1&0&+1&+2&+3&+4以上 \\ \hline
      ベット(ユニット)&1&2&3&4&5&6&7&8&9&10 \\ \hline
    \end{tabular}
    \caption{High-Low法でのベットシステム}
    \label{hlb}
  \end{table}

次にKO法についてである。KO法は以下の\ref{koc}のようにカードの値が2から7の時にはランニングカウントをプラス1、8と9の時はランニングカウントを0、そして10、Aの時にはマイナス1をランニングカウントに加えていくというものである。

  \begin{table}[H]
    \centering
    \begin{tabular}{|c|c|c|c|c|c|c|c|c|c|} \hline
      2&3&4&5&6&7&8&9&10&A \\ \hline
      +1&+1&+1&+1&+1&+1&0&0&-1&-1 \\ \hline
    \end{tabular}
    \caption{KO法でのカードカウント}
    \label{koc}
  \end{table}

KO法の大きな特徴としてはデック数に応じて、初期ランニングカウントを変えていくということである。カジノで使用されるデック数で一般的な6デックを例にとると、初期ランニングカウントは-20である。仮に6デックの場合には合計で24枚のAがあり、そこから1デック分のA、つまり4枚を抜いた数を初期ランニングカウントとして扱う。なので1デックの場合の初期ランニングカウントは0、同様に2デックの場合は-4、6デックの場合は-20が初期ランニングカウントとなる。以下の\ref{kob}はKO法を使用した際のベットシステムである。

\begin{table}[H]
  \centering
  \begin{tabular}{|c|c|c|c|c|c|c|c|c|c|c|} \hline
    ランニングカウント&-5以下&-4&-3&-2&-1&0&+1&+2&+3&+4以上 \\ \hline
    ベット(ユニット)&1&2&2&3&4&5&6&8&9&10 \\ \hline
  \end{tabular}
  \caption{KO法でのベットシステム}
  \label{kob}
\end{table}

\bunseki{*薩田凱斗}
