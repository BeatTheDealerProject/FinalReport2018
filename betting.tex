\section{ベッティングシステムの検討}
 カウンティング時の適切なベッティング手法が存在しないため、カウンティングに適用可能な手法を調査した。その結果ベッティング方法は以下のようなグループに分けることができた。
\begin{itemize}
\item 勝ち負けに応じて賭け額を変化させる手法
\item 数列を用意し、その数列を操作しながら賭け額を変化させる手法
\item 資金の何割かを賭ける手法
\item 勝率に応じて賭け額を変化させる手法
\end{itemize}
以上の4つについてこれから説明する。
\bunseki{※柏田輝}

\subsection{勝ち負けに応じて賭け額を変化させる手法}
勝ち負けに応じて賭け額を変化させる手法では、代表的なものとしてマーチンゲール法というものが挙げられる。この手法は、基準となる賭け額を決めて、負けるたびに賭け額を倍に増やすという手法である。例えば最初の賭け額を100と設定した場合、1度負けると200、2度負けると400...のように賭け額が倍になるのが特徴である。この手法の特徴としては、1度勝てば負け額をすべて取り戻し、基準となる賭け額だけ利益を出すことが出来る。しかし、負け続けると賭け額が指数関数的に増えてしまい、破産しやすいというデメリットがある。
\bunseki{※柏田輝}

\subsection{数列を用意し、その数列を操作しながら賭け額を変化させる手法}
 数列を用意し、その数列を操作しながら賭け額を変化させる手法では、代表的なものとして2in1法というものが挙げられる。この手法は、賭け額の単位を設定する。2連敗後に適用し、負ける度にその直前に賭けた単位を右端に記録していき、勝つ度に記録の両端の数字を1つずつ削除していきます。毎ゲーム時に賭ける単位は、勝ち負けに関わらず記録の両端の数字を合計した金額とする。例えば、2連続で負けた後の数列は{1,1}となり、ここから適用する。両端数字が1,1なので次は2単位賭ける。ここで負けた場合数列が{1,1,2}となり、次は3単位賭けることになる。更ににここで負けた場合数列が{1,1,2,3}となり次は4単位賭けることになる。ここで勝つと数列の両端の数字である1,3を削除し数列が{1,2}に変化するので、次は3単位賭けることになる。そこでまた勝つことにより、数列の両端の数字がなくなり、負けた分を全て回収できたことになる。この手法の特徴は、2回分の負け額を1度の賭けで回収のすることができる。また、マーチンゲール法ほど賭け額が増えないので破産率が低いのも特徴である。しかし、この手法ではマーチンゲール法のように1度勝っただけでは利益が増えない。つまり、連勝しなければ利益を得ることができない手法となっている。また実際のカジノでは、メモを見たりメモを取ることができないため、数列を暗記し、ややこしい計算を頭の中で行わなければならないといったデメリットもある。
\bunseki{※柏田輝}

\subsection{資金の何割かを賭ける手法}
資金の何割かを賭ける手法では、代表的なものとして10\%投資法というものが挙げられる。この手法は、自分がゲーム使える額全体のうちから一度のゲームにつき、その10\%を賭けるという手法である。この手法の特徴としては、絶対に破産しない上に計算が単純なので覚える必要がない。また、連敗すると賭け金が低くなるので、負けやすい間の不利益を抑えることができ、連勝すると賭け金が高くなるので、勝ちやすい間の利益が高くなる。しかし、今までのベッティング手法とは違い、負け額を確実に回収したり、利益を確実に出す手法ではないので、利益を得にくいといったデメリットがある。また、勝率が50\%を下回ってしまうと持ち金が0に収束していくというデメリットもある。
\bunseki{※柏田輝}

\subsection{勝率に応じて賭け額を変化させる手法}
勝率に応じて賭け額を変化させる手法では、代表的なものとしてケリー基準というものが挙げられる。この手法は、以下のような公式から賭け額を決定する。
\begin{center} (AP-Q)/A 
\end{center}
Aはデシマルオッズから1引いた値、つまり勝った場合の純粋な利益であり、Pは勝つ確率であり、Qは負ける確率である。これらを計算し出た値に1を足した値を賭け額の単位とする。例えば、基準となる賭け額が100、勝つ確率が52\%、負ける確率が48\%、デシマルオッズが2だった場合は、(1*0.52-0.48)/1=0.04となるので、賭け額は104となる。この手法の特徴としては、他のベッティングシステムとは異なり勝率を参照するので、カウンティングを併用しやすいといったメリットがある。一方で、毎回勝率を計算しなければならないといったデメリットがある。
\bunseki{※柏田輝}
\subsection{まとめ}
 ベッティングシステムについて調査した結果、ベッティング手法は、勝ち負けに応じて賭け額を変化させる手法、数列を用意し、その数列を操作しながら賭け額を変化させる手法、資金の何割かを賭ける手法、勝率に応じて賭け額を変化させる手法の4種類のグループに分けることができた。そのうち、勝ち負けに応じて賭け額を変化させる手法と数列を用意し、その数列を操作しながら賭け額を変化させる手法は、カウンティング値を参照するのが最初の一度のみで、その後はカウンティング値が低くなっても、途中で賭け額を決めることができない。また、資金の何割かを賭ける手法についても、賭け額が資金によって決定されているのでカウンティング値を参照しない。これらの点から、勝率に応じて賭け額を変化させる手法がカウンティングに適用しやすい手法だと考えられる。
\bunseki{※柏田輝}

