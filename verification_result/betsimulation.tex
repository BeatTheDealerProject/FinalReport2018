\section{一定ベット}
まずは、賭け金を常に一定として、所持金がどのように変化するのかをシミュレーションで調べる。シミュレーション結果を図\ref{betdife}と表\ref{bet}で示す。
\begin{figure}[H]
 \begin{center} 
  \includegraphics[width=0.7\linewidth]{./figure/betsimulation-bet-defineite}
  \caption{一定ベット\label{betdife}}
 \end{center}
\end{figure}

\begin{table}[H]
 \caption{一定ベットの所持金\label{bet}}
 \begin{center}
  \begin{tabular}{|c|c|c|}
  \hline  & 40回目の平均所持金 & 標準偏差 \\
  \hline BS & 998.328 & 73.538\\
  \hline BS-HS & 990.499 & 62.013 \\
  \hline GA戦略 & 983.470 & 62.401\\
  \hline
  \end{tabular}
 \end{center}
\end{table}
このシミュレーションでは最初の所持金を1000とし、常に10賭け、ブラックジャックの勝負を連続で40回行い、それを5万回行った。また戦略はダブルダウンやスプリットなどが入ったベーシックストラテジー(BS)とヒットとスタンドのみのベーシックストラテジー(BS-HS)、GAで作成した戦略(GA戦略)の3つを使用した。\\
 図\ref{betdife}で所持金は勝負を重ねるたびに減少していて、最後の40回目では最初の所持金である1000より少し少なくなっているように見える。そこで表\ref{bet}を見ると40回目のそれぞれの所持金が分かるが、ダブルダウンやスプリットが入ったベーシックストラテジー、ヒットとスタンドのみのベーシックストラテジー、GA戦略の順で多いことが分かる。どの戦略でも最初の所持金である1000を超えることはないことが分かった。\\
 賭け金が一定では最初の所持金を超えることができなかった。そのため本プロジェクトでは賭け金を変動させて解決を目指した。そこで賭け金を変動させる手法を2つ考えた。1つ目は勝率が高くなる部分で賭け金を増加させる手法である。ブラックジャックを連続で行ったときに勝率が変化するタイミングが存在すると考え、そのタイミングで賭け金を増加させようとした。そのため、まずはブラックジャックの勝負40回の中で勝率が変動するところがないかを確認した。2つ目はカウンティングという手法を用いてみる。
\bunseki{柿崎大輝}

\subsection{勝率の推移}
ブラックジャックを連続で行った場合の勝率の変化を調べる。シミュレーション結果を図\ref{win}で示す。
\begin{figure}[H]
 \begin{center} 
  \includegraphics[width=0.7\linewidth]{./figure/betsimulation-win}
  \caption{勝率の推移\label{win}}
 \end{center}
\end{figure}
ブラックジャックの勝負を連続で40回行い、それを5万回繰り返し、勝負ごとでの勝率を調べた。デック数は6で、使用した戦略はヒットとスタンドのみのベーシックストラテジーであった。\\
図\ref{win}を見ると勝率は42\%\verb|~|44\%の間で増減を繰り返しており、だいたい43\%の部分に集中していると分かる。勝率は特に目立った規則性は見られず、ランダムに増減を繰り返しているように見られる。勝率は最高で43.87\%で最低は42.868\%であった。\\
 シミュレーション結果から勝率は勝負40回の中であまり変化しないと考えられる。また、勝率の変動に規則性は見つからなかった。そのため、勝負のどこかで賭け金を増やすという戦略はできないということが分かった。
\bunseki{柿崎大輝}

\subsubsection{カイ2乗検定}
40回の勝負で1つ1つを比べてみて、勝率の差は小さいものであった。しかし、本当にその差には意味がないのか、小さくても本当は意味がある差なのではないか。それを確かめるために、カイ2乗検定を行い、勝率の差が有意な差であるかどうかを確かめる。\\
 カイ2乗検定を行う前に必要な条件を表\ref{win-x}でまとめる。
\begin{table}[H]
 \caption{勝率の差のカイ2乗検定条件\label{win-x}}
 \begin{center}
  \begin{tabular}{|c|c|}
  \hline 帰無仮説 & 勝率の差に有意な差がない \\
  \hline 対立仮説 & 勝率の差に有意な差がある \\
  \hline 有意水準 & 5\% \\
  \hline 自由度 & 39 \\
  \hline 棄却値 & 54.57\\
  \hline
  \end{tabular}
 \end{center}
\end{table}
有意水準5\%で、棄却値は54.57とした。カイ2乗値がこの棄却値54.57より大きければ帰無仮説を棄却し、対立仮説を採択する。\\
 ここで、カイ2乗検定を行うとカイ2乗値は40.207となった。カイ2乗値は棄却値よりも小さくなったので、帰無仮説を採択する。よって、勝率の差には有意な差がないという結果になった。つまり、勝負40回で勝率の変化はないという結果となった。
\bunseki{柿崎大輝}

\section{カウンティング}
カウンティングで賭け金を変動させ、所持金を調べる。カウンティングは既存のカウンティング手法としてKO法とHigh-Low法の2種類を用いる。そのほかの条件は一定ベットの時のシミュレーションと同じとした。シミュレーション結果を図\ref{KO}と図\ref{Hi-Lo}と表\ref{countting}で示す。
\begin{figure}[H]
 \begin{center} 
  \includegraphics[width=0.7\linewidth]{./figure/betsimulation-KO}
  \caption{カウンティングKO法\label{KO}}
 \end{center}
\end{figure}

\begin{figure}[H]
 \begin{center} 
  \includegraphics[width=0.7\linewidth]{./figure/betsimulation-Hi-Lo}
  \caption{カウンティングHigh-Low法\label{Hi-Lo}}
 \end{center}
\end{figure}

\begin{table}[H]
 \caption{カウンティングの所持金\label{countting}}
 \begin{center}
  \begin{tabular}{|c|c|c|}
  \hline  & 40回目の平均所持金 & 標準偏差 \\
  \hline KO法BS & 1000.237 & 172.770\\
  \hline KO法BS-HS & 987.207 & 148.615 \\
  \hline KO法GA戦略 & 976.380 & 145.821\\
  \hline High-Low法BS & 1027.961 & 298.197\\
  \hline High-Low法BS-HS  & 997.561 & 262.546\\
  \hline High-Low法GA戦略 & 981.315 & 263.162\\
  \hline
  \end{tabular}
 \end{center}
\end{table}

KO法は図\ref{KO}で見ると一定ベットの時とあまり変化がないように見えるが、表\ref{countting}で見るとダブルダウンやスプリットが入ったベーシックストラテジーの所持金が増加していることが分かる。\\
 High-Low法は図\ref{Hi-Lo}ではで増えていることがはっきりと分かる。表\ref{countting}で見てもすべての戦略で一定ベットの時よりも増えていることが分かる。しかし、標準偏差が一定ベットやKO法の2つより大きく、安定していないことが分かる。\\
 既存のカウンティング手法を使うことでダブルダウンやスプリットが入ったベーシックストラテジーでは1000を超えることができた。ただ、ヒットとスタンドのみのベーシックストラテジーやGA戦略では1000を超すことはできなかった。
\bunseki{柿崎大輝}

\section{カウンティングの根拠}
\ \ カウンティングが有効であるという理由は2つある。1つ目に、プレイヤーは行動を選択でき、ディーラーは16以下の時ヒットをしなければならないというルールによるものである。例えば、デックの残りが全て8であり、プレイヤーはカウンティングでその状況を把握しているとする。この時、プレイヤーのハンドは8が2枚の16であり、ディーラーのアップカードも8となる。プレイヤーはここでヒットをすると8を引いてバーストになってしまうことが分かっているのでスタンドを選択する。一方、ディーラーの2枚目のカードも8であるため、ディーラーのハンドも8が2枚の16である。しかしディーラーは16以下のためヒットを選択する。すると8を引くので24となりディーラーはバーストになってしまう。これは極端な例だがこの場合プレイヤーの勝利が確定している。なので行動を変えられるプレイヤーにとってカウンティングは有効である。
\ \ 2つ目に、ブラックジャックにはプレイヤーに有利なカード、ディーラーに有利なカード、プレイヤーに不利なカード、ディーラーに不利なカードが存在するという点である。例えば、10はプレイヤーに有利でディーラーに不利なカードである。ここで、10以外のカードが多く使用されデックに10が多く残っている状況を想定する。この時、プレイヤーは11以下の時ダブルダウンを行うと10を引く確率が高いので20や21など21に近く強いハンドになり勝率が上がる。また、プレイヤーは行動を変えられるので12以上の時はスタンドをすることでバーストを避けることができる。さらに、ナチュラルブラックジャックの確率も上がるため利得が高くなる。反対に、ディーラーはダブルダウンの選択肢がないので利得は上がらない。また、ディーラーは16以下の時ヒットをするというルールがあるため、12,13,14,15,16の時もヒットをするのでバースト率が上がる。さらにディーラーはナチュラルブラックジャックであっても配当が1.5倍にはならないため、利得が上がらない。以上のことから、10以外のカードが多く使用されデックに10が多く残っていると、プレイヤーの利得が上がりディーラーのバースト率が上がるため10はプレイヤーに有利でディーラーに不利なカードであるということが言える。次に、10が多く使用され2,3,4がデックに多く残っている状況を想定する。この時、プレイヤーはダブルダウンをすると13や14など21から遠く弱いハンドになるため勝率が下がる。また、ベーシックストラテジーではプレイヤーのハンドが17の時にはディーラーのハンドに関わらず全ての場合でスタンドとなっているのでプレイヤーはスタンドを選択するが、この場合ではヒットをしてもバースト率が低いためスタンドによって勝率を下げている。さらに、ナチュラルブラックジャックの確率が低いため利得が下がる。反対に、ディーラーは元々ダブルダウンをしないため利得が下がらない。また、バースト率が下がるためディーラーの勝率が上がる。さらにディーラーはナチュラルブラックジャックであっても配当が1.5倍にはならないため、利得が上がらない。
\ \ 以上のことから、2,3,4はプレイヤーに不利でディーラーに有利なカードであるということが言える。また、ブラックジャックにはプレイヤーに有利なカード、ディーラーに有利なカード、プレイヤーに不利なカード、ディーラーに不利なカードが存在するということがいえる。この2つの理由により、カウンティングは有効であるということが言える。
\bunseki{轟木文弥}

\subsection{カードの重要度の検証}
ブラックジャックにはプレイヤーに有利なカード、ディーラーに有利なカード、プレイヤーに不利なカード、ディーラーに不利なカードが存在する。そこで本プロジェクトは、どのカードがどれほどプレイヤーにとって有利なカードかを検証した。検証方法として、普通のデ1ック、1デックから1種類カードを抜いたデックをそれぞれ用意し、シミュレーションした結果を比較した。比較対象となるデックは以下の通りである。

\begin{itemize}
 \item 普通の1デック
 \item Aを抜いたデック
 \item 2を抜いたデック
 \item 3を抜いたデック
 \item 4を抜いたデック
 \item 5を抜いたデック
 \item 6を抜いたデック
 \item 7を抜いたデック
 \item 8を抜いたデック
 \item 9を抜いたデック
 \item 10を抜いたデック
\end{itemize}

\subsection{仮説}
10,Aが残っていると勝率が上がる。逆に2,3,4が残っていると勝率が下がる。5,6,7,8,9は影響がないという仮説を設定する。これは次の3つに注目したからである。1つ目に、数値の小さいカードはダブルダウンの勝率が下がり、ディーラーのバースト率が下がることからプレイヤーにとって不利である。2つ目に、10はダブルダウンの勝率が上がり、ディーラーのバースト率が上がることからプレイヤーに有利である。3つ目に、Aはプレイヤーの方が1と11を自由に選択でき、ナチュラルブラックジャックの可能性もあるため有利である。この3つの考えから、仮説を設定した。
\bunseki{轟木文弥}

\section{シミュレーション}
設定した仮説をそれぞれの数字をデックから抜いた場合のシミュレータを回して勝率を確認した。これによりカードの重要度を測った。デック数は6とした。戦略はダブルダウン、スプリット、サレンダーを含むベーシックストラテジーを使用した。シミュレーション回数は10万回とした。ただし、今回は賭け金を導入しなかったためダブルダウンは利得が2倍になるという点を勝利数2とすることで表現した。シミュレーション結果を\ref{pullcard}に示す。

\begin{table}[H]
 \caption{カードを抜いたデックの勝率\label{pullcard}}
 \begin{center}
  \begin{tabular}{|c|c|c|}
  \hline & 勝ち(\%) & 勝ち以外(引き分けと負け、\%) \\
  \hline 普通の1デック & 42.653 & 57.347\\
  \hline Aを抜いたデック & 41.576 & 58.424\\
  \hline 2を抜いたデック & 42.697 & 57.303\\
  \hline 3を抜いたデック & 43.023 & 56.977\\
  \hline 4を抜いたデック & 42.891 & 57.109\\
  \hline 5を抜いたデック & 43.940 & 56.060\\
  \hline 6を抜いたデック & 43.756 & 56.244\\
  \hline 7を抜いたデック & 42.851 & 57.149\\
  \hline 8を抜いたデック & 42.047 & 57.953\\
  \hline 9を抜いたデック & 42.127 & 57.873\\
  \hline 10を抜いたデック & 41.576 & 58.424\\
  \hline
  \end{tabular}
 \end{center}
\end{table}

8,9,10,Aを抜いたデックは普通のデックより勝率が下がり、2,3,4,5,6,7を抜いたデックは勝率が上がった。
得られたデータをカイ2乗検定と多重比較を用いてどこに差が存在するかを検証した。
結果として、5,6,10,Aには有意な差が存在することが確認された。
つまり、5,6はプレイヤーに不利なカードであり、10,Aはプレイヤーに有利なカードであるという結果が得られた。
\bunseki{轟木文弥}

\section{新しいカウンティング手法の提案}
KO法は安定しているが変化が小さく、High-Low法は変化が大きいが安定しなかった。本プロジェクトは安定していて変化が大きいカウンティング手法を見つけるべく新たに2つのカウンティング手法を考えた。それがRUKO法と5-6-A法である。今回はKO法をベースにして、安定しているKO法の変化を大きくする手法を考えた。ここではその2つのカウンティング手法について詳しく説明する。
\bunseki{轟木文弥}

\subsection{RUKO法}
まずRUKO法について説明する。RUKO法とはRisk Up KO法の略で、KO法のリスクを上げることで利得を上げようとしたカウンティング手法である。KO法は6デックの場合初期ランニングカウントが-20で、ランニングカウントが-5以下の時は1ユニットしか賭けない。そのため賭け金を上げる段階が遅くなっている。そこでRUKO法は、初期ランニングカウントをKO法の半分とした。すなわち6デックの場合は-10である。これによりKO法より早い段階からランニングカウントが-5を超えることで賭け金を大きくし、利得を上げようとした。カードカウントとベットシステムはKO法と同じである。
\bunseki{轟木文弥}

\subsection{5-6-A法}
次に5-6-A法について説明する。5-6-A法とは、カードの有利さの検証結果から有意な差があった部分の5,6,Aを考慮したカウンティング手法である。先述のように、KO法は賭け金を上げる段階が遅くなっている。そこで5-6-A法では以下の\ref{56Acount}のように、特に不利なカードである5と6が出た時ランニングカウントを+2し、特に有利なカードであるAが出た時ランニングカウントを-2した。ただし、10は特に有利なカードであるが、10はデックに存在する枚数が多くランニングカウントを-2するとカウントが大きくならず賭け金があまり上がらなくなる。これは賭け金を上げるという意図に反しているため10が出てもランニングカウントを-1とした。

\begin{table}[H]
    \centering
    \label{56Acount}
    \caption{5-6-A法でのカードカウント}
    \begin{tabular}{|c|c|c|c|c|c|c|c|c|c|} \hline
      2&3&4&5&6&7&8&9&10&A \\ \hline
      +1&+1&+1&+2&+2&+1&0&0&-1&-2 \\ \hline
    \end{tabular}
\end{table}

初期ランニングカウントとベットシステムはKO法と同じである。この手法によりKO法よりランニングカウントを大きくし賭け金を上げ利得を増やそうとした。また、このカードカウントにより正確にデックの状態を判断しようとした。
\bunseki{轟木文弥}

\section{RUKO法、5-6-A法のシミュレーション}
カウンティングで賭け金を変動させ、所持金を調べる。カウンティングは本プロジェクトで作成したカウンティング手法のRUKO法と5-6-A法の2種類を用いる。そのほかの条件は一定ベットの時のシミュレーションと同じとした。シミュレーション結果を図\ref{RUKO}と図\ref{56A}と表\ref{countting}で示す。


\begin{figure}[H]
 \begin{center} 
  \includegraphics[width=0.7\linewidth]{./figure/betsimulation-RUKO}
  \caption{カウンティングRUKO法\label{RUKO}}
 \end{center}
\end{figure}

\begin{figure}[H]
 \begin{center} 
  \includegraphics[width=0.7\linewidth]{./figure/betsimulation-56A}
  \caption{カウンティング5-6-A法\label{56A}}
 \end{center}
\end{figure}

\begin{table}[H]
 \caption{カウンティングの所持金\label{countting}}
 \begin{center}
  \begin{tabular}{|c|c|c|}
  \hline  & 40回目の平均所持金 & 標準偏差 \\
  \hline KO法BS & 1000.237 & 172.770\\
  \hline KO法BS-HS & 987.207 & 148.615 \\
  \hline KO法GA戦略 & 976.380 & 145.821\\
  \hline High-Low法BS & 1027.961 & 298.197\\
  \hline High-Low法BS-HS  & 997.561 & 262.546\\
  \hline High-Low法GA戦略 & 981.315 & 263.162\\
  \hline RUKO法BS & 1000.237 & 172.770\\
  \hline RUKO法BS-HS & 987.207 & 148.615 \\
  \hline RUKO法GA戦略 & 976.380 & 145.821\\
  \hline 5-6-A法BS & 995.918 & 211.884\\
  \hline 5-6-A法BS-HS & 982.640 & 179.584 \\
  \hline 5-6-A法GA戦略 & 968.303 & 180.464\\
  \hline
  \end{tabular}
 \end{center}
\end{table}
\bunseki{轟木文弥}

\section{RUKO法のまとめ}
RUKO法の特徴は、6デックの場合のKO法の初期ランニングカウントが-20であるのに対し-10に変更することで、より早い段階から賭け金を大きくすることにより最終的な利得を増やすことを目的として作成したものである。RUKO法を使用した場合の40回目の所持金の推移のグラフ\ref{RUKO}を見てみると、ベーシックストラテジー、ヒットとスタンドのみのベーシックストラテジー、GA戦略の全ての戦略が最初の所持金額1000を下回る結果となった。これはオリジナルのKO法と比較してみても、より平均所持金を下回っていたことがわかった。また、標準偏差もKO法と比べて増加しており、安定していないことがわかった。
\bunseki{薩田凱斗}
\section{5-6-A法のまとめ}
5-6-A法の特徴は、KO法のカードカウントを変更することで、より早い段階から賭け金を大きくすることにより最終的な利得を増やすことを目的として作成したものである。5-6-A法を使用した場合の40回目の所持金の推移のグラフ\ref{56A}を見てみると、ベーシックストラテジー、ヒットとスタンドのみのベーシックストラテジー、GA戦略の全ての戦略が最初の所持金額1000を下回る結果となった。これはオリジナルのKO法と比較してみても、より平均所持金を下回っていたことがわかった。また、標準偏差もKO法と比べて増加しており、安定していないことがわかった。
\bunseki{轟木文弥}
\section{RUKO法と5-6-A法のまとめ}
既存のカウンティング手法であるKO法とHigh-Low法の一部を変更して作成したRUKOと5-6-Aの結果をまとめると、どちらもぞれぞれのオリジナルの手法よりも安定性が下がり平均所持金が減るという悪い結果となってしまった。今後の課題としては、それぞれの手法で改変した部分がなぜ平均所持金を下回った結果となってしまったかを究明するとともに、目的である最終的な利得を増やすことに重点を置いて改良していきたい。
\bunseki{薩田凱斗}
\section{カウンティング手法全部のまとめ}
ここで注目するのはGA戦略である。一定ベットやカウンティングを使用しても所持金がすべての戦略で一番少なかった。GA戦略は遺伝的アルゴリズムで探索し、発見した戦略であるのになぜ他の戦略より悪い結果となったのか。それは複雑性ということを考慮していないからではないかと考えた。GA戦略は勝率と複雑性からなる性能で探索したもので、行ったシミュレーションでは複雑性を考慮する部分がなく、その結果一番悪い結果となったと考えた。そこで複雑性を確かめる実験の結果を用いて、"エラー率"という指標を作成し、シミュレーションに導入して行うこととした。
\bunseki{柿崎大輝}

\section{エラー率の導入}
GA戦略が結果として最も低い所持金となった原因は、シミュレーションを行う条件にあった。GA戦略は複雑性が低い、つまりシンプルであることが最大の特徴である。そこで、我々はGA戦略の特徴がどのくらい有用かを調べるために複雑性を考慮したシミュレーションを行う必要があった。そして、そのシミュレーションを行うため、複雑性に関する実験の結果をもとにエラー率という概念を作成した。エラー率の定義は、ある戦略を使用する際にプレイヤーが行動を間違えてしまう確率を表したものである。このエラー率をシミュレーションに適用することにより、GA戦略を使用した際のプレイヤーの所持金が増加する可能性があるのではないかと考えた。このエラー率の導出方法として、複雑性に関する実験データの正答率をもとに線形回帰分析を行った。また、使用した実験データについては以下の表\ref{hoge}で示す通りである。

  \begin{table}[H]
    \centering
    \caption{複雑性と誤答率}
    \begin{tabular}{|c|c|c|c|c|} \hline
      複雑性&0.2&0.2&0.407&0.448 \\ \hline
      エラー率&0.002&0.009&0.125&0.162 \\ \hline
    \end{tabular}
    \label{hoge}
  \end{table}
また、以上の実験データから線形回帰分析により得られた式は、

  \begin{equation}
    y=0.611x - 0.1172
  \end{equation}

となった。この式を横軸を複雑性、縦軸をエラー率としてグラフにしたものが図\ref{hoge2}である。

  \begin{figure}[H]
    \label{hoge2}
    \begin{center}
      \includegraphics[width=10cm]{figure/betsimulation-error_rate.png}
      \caption{線形回帰分析によるエラー率のグラフ}
    \end{center}
  \end{figure}
\bunseki{薩田凱斗}

\section{エラー率を導入したシミュレーション}
エラー率を導入して、シミュレーションを行う。エラー率は戦略に従って行動するときにミスをする確率で、ミスをすると戦略表とは異なる行動を実行するようにした。またカウンティング手法はKO法とHigh-Low法の2種類を用いた。それ以外の部分は前のシミュレーションと同じ条件とした。表\ref{err}が戦略ごとのエラー率で、図\ref{errKO}と図\ref{errHi-Lo}、表\ref{money-err}がシミュレーション結果である。
\begin{table}[H]
 \caption{戦略ごとのエラー率\label{err}}
 \begin{center}
  \begin{tabular}{|c|c|}
  \hline 戦略 & エラー率(\%) \\
  \hline BS & 16.2\\
  \hline BS-HS & 4.8 \\
  \hline GA戦略 & 0.5\\
  \hline
  \end{tabular}
 \end{center}
\end{table}

\begin{figure}[H]
 \begin{center} 
  \includegraphics[width=0.7\linewidth]{./figure/betsimulation-errKO}
  \caption{エラー率ありのKO法\label{errKO}}
 \end{center}
\end{figure}

\begin{figure}[H]
 \begin{center} 
  \includegraphics[width=0.7\linewidth]{./figure/betsimulation-errHi-Lo}
  \caption{エラー率ありのHigh-Low法\label{errHi-Lo}}
 \end{center}
\end{figure}

\begin{table}[H]
 \caption{エラー率を使用した際の所持金\label{money-err}}
 \begin{center}
  \begin{tabular}{|c|c|c|}
  \hline  & 40回目の平均所持金 & 標準偏差 \\
  \hline KO法BS & 923.303 & 174.255\\
  \hline KO法BS-HS & 963.799 & 150.135 \\
  \hline KO法GA戦略 & 974.817 & 144.821\\
  \hline High-Low法BS & 948.394 & 79.764\\
  \hline High-Low法BS-HS  & 975.440 & 67.263\\
  \hline High-Low法GA戦略 & 981.474 & 67.747\\
  \hline
  \end{tabular}
 \end{center}
\end{table}
エラー率を導入した結果ではGA戦略、ヒットとスタンドのみのベーシックストラテジー、ダブルダウンやスプリットが入ったベーシックストラテジーの順となり、所持金が1番多かったダブルダウンやスプリットが入ったベーシックストラテジーと1番少なかったGA戦略が入れ替わった。特にダブルダウンやスプリットが入ったベーシックストラテジーはエラー率を導入することで大きく所持金が少なくなった。またエラー率を導入したシミュレーションでは1000を超える戦略はないことが分かった\\
 エラー率を導入した場合、GA戦略が1番優れていることが分かった。しかし、GA戦略では1000を超えることができないことも分かった。この点に関してGA戦略は改善がする必要があると考えられる。
\bunseki{柿崎大輝}

\subsection{分散分析}
エラー率を導入したシミュレーションの時、3つの戦略間で40回目の平均所持金に有意な差があるかどうかを調べる。まず、3つの戦略において有意な差があるかどうかを調べるため、分散分析を行った。優位水準は5\%、KO法とHigh-Low法の2種類で行った。表\ref{conditions-b}で分散分析の条件をまとめた。
\begin{table}[H]
 \caption{分散分析の条件\label{conditions-b}}
 \begin{center}
  \begin{tabular}{|c|c|}
  \hline 帰無仮説 & 3つの戦略で平均所持金に有意な差はない \\
  \hline 対立仮説 & 3つの戦略で平均所持金に有意な差はある \\
  \hline 有意水準 & 5\% \\
  \hline
  \end{tabular}
 \end{center}
\end{table}
分散分析を行うと、KO法、High-Low法のp値がとても小さくなり、0.05以下となる。よって帰無仮説を棄却し、対立仮説を採択する。つまり、3つの戦略で有な差が存在することが確認できた。次は3つの戦略のどこに有意な差が存在するのかを調べるためにここから多重比較を行う。
\bunseki{柿崎大輝}

\subsection{多重比較}
それぞれの戦略で有意な差があるかを調べため、多重比較を行った。有意水準を5\%として行った。結果は表\ref{multiKO}と表\ref{multiHigh-Low}に示す。
\begin{table}[H]
 \caption{KO法での多重比較\label{multiKO}}
 \begin{center}
  \begin{tabular}{|c|c|c|}
  \hline  & BS & BS-HS  \\
  \hline  BS-HS 所持金の差 & 40.496 & \\
	               p値 & 0 & \\
  \hline GA 所持金の差 & 51.515 & 11.019\\
                p値 & 0 & 0\\
  \hline
  \end{tabular}
 \end{center}
\end{table}
\begin{table}[H]
 \caption{KO法での多重比較\label{multiHigh-Low}}
 \begin{center}
  \begin{tabular}{|c|c|c|}
  \hline  & BS & BS-HS  \\
  \hline  BS-HS 所持金の差 & 27.046 & \\
	               p値 & 0 & \\
  \hline GA 所持金の差 & 33.080 & 6.034\\
                p値 & 0 & 0\\
  \hline
  \end{tabular}
 \end{center}
\end{table}
 表\ref{multiKO}と表\ref{multiHigh-Low}ではそれぞれの戦略を比較した場合の所持金の差とp値を表示している。所持金がプラスの場合は列側の戦略のほうが所持金が多く、逆にマイナスなら行側の戦略のほうが所持金が多い。また、p値が0.05より小さい場合、その2つの戦略で有意な差が存在するということが分かる。\\
 KO法ではすべての戦略間でp値が0.05以下となり、すべての戦略間で有意な差が存在することが分かった。High-Low法でもすべての戦略間でp値が0.05以下となり、すべての戦略間で有意な差が存在することが分かった。これでKO法High-Low法の両方ですべて戦略間で有意な差が存在する。
\bunseki{柿崎大輝}
