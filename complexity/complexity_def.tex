\subsection{複雑性の定義について}

プレイヤーに扱いやすい戦略とは何かについて戦略の複雑性を定義し、複雑性の低い戦略をプレイヤーに扱いやすい
戦略であるとした。戦略の複雑性はChaitin(1969)によって定義されたコルモゴロフ複雑性を参考に新たに定義した。
コルモゴロフ複雑性とは、ある文字列があったときに、その文字列を生成するためのプログラムの内、最小の命令長を
その文字列の複雑性であると定義したものである。

新たに定義した複雑性を算出するためには、まず戦略の表を文字列として文字列を圧縮する。
圧縮の方法としては、元の表を「連続する文字+連続して文字が出た回数」と変換することで文字列を圧縮した。
例として、「HHSSSHHHHH」という10文字からなる文字列を圧縮すると、「H2S3H5」となり、圧縮した後の文字列は6文字
となる。また「HHHHHHHHHH」という文字列を「H10」と圧縮したときのように、連続して同じ文字が出た回数が2桁の場合には
数字部分を1文字として数える。つまり「H10」の文字列長は2文字である。

圧縮した後の文字列長を元の表の大きさ(80)で割った数値を複雑性の定義とした。
\bunseki{※米村祥裕}