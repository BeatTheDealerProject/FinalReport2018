後期は、まず主に3つの課題を設定し、それぞれの課題解決に向かってグループを作成し活動した。

1つ目の課題は、遺伝的アルゴリズムを使用して新しいブラックジャックの戦略を探索することである。以下、この課題解決を行ったグループ名を、GA班とする。

2つ目の課題は、前期活動で設定した複雑性という要素が、人間が戦略表を記憶する能力と一致しているのかを、実験を通して検証することである。以下、この課題解決を行ったグループ名を、複雑性班とする。

3つ目の課題は、最適な戦略の探索に、賭け金の概念を導入し、それに伴いカウンティングについても既存のものを学習し、賭け方に考慮することである。以下、この課題解決を行ったグループ名を、カウンティング班とする。

以下に、それぞれのグループの活動のあらましを述べる。

\begin{itemize}
\item GA班
\end{itemize}

まずはじめに、遺伝的アルゴリズムを用いて戦略を探索するプログラムを作成した。遺伝的アルゴリズムを使用した背景には、ブラックジャックの戦略の組み合わせは膨大であり、遺伝的アルゴリズムを使用するのが最適であると判断したという理由がある。使用した言語はPythonである。探索は、研究費で購入した計算用PCを使用し、また学内から接続できるようにネットワーク設定も行った。

そのようにして得られた戦略を用いて勝率を導出したが、予想されていたほどの結果は出なかった。そこで原因を解明したところ、交叉手法や突然変異確率、初期個体値の見直しが必要であると判明した。それらを見直したうえで最終的にGA戦略という新たな戦略を導出した。

\begin{itemize}
\item 複雑性班
\end{itemize}

まずはじめに、コルモゴロフ複雑性を参考にし、複雑性とカウンティングなどの戦略の組み合わせで利得が最大になるような戦略の評価方法を考えた。

その結果、戦略の評価方法を知るには実際に人間に戦略表を記憶させる実験を行うことが必要だと判明したため、実験の計画をたてた。実験には認知的判断能力も知る必要があったため、知能テストの作成も行った。

実験から得られた結果をもとに、前期に作成した複雑性との相関を検定した。その結果、2つの間には非常に強い負の相関がみられた。つまり、複雑性が高くなるにつれ、戦略表を間違えて覚えやすいということであり、我々が導出した複雑性は評価指標として適していることが明らかになった。

また、同時に行った認知的判断能力テストの結果から、ブラックジャックの戦略を記憶するうえで一般的な認知能力は関係がないという新しい結果も得られた。

\begin{itemize}
\item カウンティング班
\end{itemize}

まずはじめに、前期に作成したシミュレータを改変し、ダブルダウン、スプリット、サレンダーが行えるようにした。それに伴い、それらのルールを織り込んだカウンティングとベッティングシステムについて調査した。ベッティングシステムについては、それぞれの比較が必要であったため、比較方法の手法の検討も行った。

これらの研究から、ある戦略を使用した際、膨大な数のゲーム数を行った場合の最終的な利得を、カウンティング手法を変更した時でも導出できるシミュレータの開発に成功した。そしてこのシミュレータを使用して、GA班が導出した戦略について検証した。

ところがGA班から導出された戦略を用いた戦略では、基本戦略を超えた結果が出ることはなかった。これについて検討したところ、戦略を間違えて覚える確率(以下、エラー率とする)について考えていなかったためそのような結果になったと考えられた。そこで、複雑性班が行った実験の結果を基にエラー率を導出し、シミュレーションに組み込んだ。その結果、当初の予想の通り、GA班が導出した戦略が一番優秀であるという結果が得られた。

\begin{itemize}
\item 共通して行った事
\end{itemize}

函館市の高校生が参加する見学会があったため、そのためのスライド作成などを行った。
12月には、一年の活動の総まとめとして成果発表会に参加した。

\bunseki{※菱田美紗紀}
