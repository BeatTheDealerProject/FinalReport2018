\section{ベッティングシステムの検討}
カウンティング時の適切なベッティングシステムが存在しないため、カウンティングに適用可能な手法を調査した。その結果ベッティングシステムは以下のようなグループに分けることができた。
\begin{itemize}
\item 勝ち負けに応じて一定の倍率で賭け額を変化させる手法
\item 数列を用意し、数列を操作しながら賭け額を変化させる手法
\item 資金の何割かを賭ける手法
\item 勝率に応じて賭け額を変化させる手法
\item その他の手法
\end{itemize}
以上の5つについてこれから説明する。
\bunseki{※柏田輝}

\subsection{勝ち負けに応じて一定の倍率で賭け額を変化させる手法}
勝ち負けに応じて賭け額を変化させる手法には以下のような手法がある。
\begin{itemize}
\item マーチンゲール法
\item グランマーチンゲール法
\item パーレー法
\item グランパーレー法
\end{itemize}
\subsubsection{マーチンゲール法}
マーチンゲール法とは、基準となる賭け金(ユニット)を決めて、負けるたびにユニットを倍に増やしていき、勝った時にユニット数を1に戻すという手法である。例えば1ユニットを100と設定した場合、1度負けると100×2となり200、2度負けると200×2となり400となる。この手法の特徴としては、1度勝てば負け額をすべて取り戻し、1ユニット分だけ利益を出すことが出来る。しかし、負け続けると賭けるユニット数が指数関数的に増えてしまい、破産しやすいというデメリットや、カジノでは賭け金の上限が設定されているため、上限に達すると使用できないといった問題点がある。
\subsubsection{グランマーチンゲール法}
グランマーチンゲール法とは、負けるたびにユニットを2倍しその値に1を加えて勝つまでユニット数を増やしていき、勝った時にユニット数を1に戻す手法である。例えば1ユニットを100と設定した場合、1度負けると100×2+100となり300、2度負けると300×2+100となり700となる。この手法の特徴として、最初の1ユニット目を賭けてから勝った場合までのゲーム回数×1ユニット分の利益を出すことができる。つまり、マーチンゲール法をハイリスクハイリターンにした手法であるため、勝ち続けることができれば利益を出すことができるが、負け続けるとマーチンゲール法より早く賭け金の上限に達してしまったり、より破産しやすくなってしまう。
\subsubsection{パーレー法}
パーレー法とは、勝つたびにユニット数を2倍にする手法である。例えば、1ユニットを100と設定すると、1度勝つと100×2となり200、2度勝つと200×2で400となる。この手法の特徴として、連勝することで少ない賭け金で大きな利益を出すことができる。しかし、1度でも負けてしまうと利益がマイナスになってしまうのである程度利益をした後、自分でユニット数を1に戻す必要がある。
\subsubsection{グランパーレー法}
グランパーレー法とは、勝つたびにユニット数を2倍にしその値に1を加えていく手法である。例えば1ユニットを100と設定した場合の賭け金は、1度勝つと100×2+100となり300、2度勝つと300×2+100となり700となる。この手法の特徴として、勝ち続けると利益を指数関数的に増やすことができる。しかし、負けた場合に最初の1ユニット目を賭けてからその時までのゲーム回数×1ユニット分負けてしまう。つまり、パーレー法をハイリスクハイリターンにした手法である。
\bunseki{※柏田輝}

\subsection{数列を用意し、その数列を操作しながら賭け額を変化させる手法}
 数列を用意し、その数列を操作しながら賭け額を変化させる手法には以下のような手法がある。
\begin{itemize}
 \item 2in1法
 \item モンテカルロ法
 \item バーネット法
\end{itemize}
\subsubsection{2in1法}
2in1法とは、数列の両端を足した数のユニット数を賭ける手法である。この手法は2連敗した後に適用され、負ける度にその直前に賭けたユニット数を右端に記録していき、勝つ度に記録の両端の数字を1つずつ削除する。毎ゲーム時に賭けるユニット数は、勝ち負けに関わらず記録の両端の数字を合計したユニット数を賭ける。例えば、2連続で負けた後の数列は1,1となり、ここから適用する。両端数字が1,1なので次は2ユニット賭ける。ここで負けた場合数列が1,1,2となり、次は3ユニット賭けることになる。更ににここで負けた場合数列が1,1,2,3となり次は4ユニット賭けることになる。ここで勝つと数列の両端の数字である1,3を削除し数列が1,2に変化するので、次は3ユニット賭けることになる。そこでまた勝つことにより、数列の両端の数字がなくなり、負けた分を全て回収できたことになる。この手法の特徴は、2回分の負け額を1度の賭けで回収のすることができる。また、マーチンゲール法ほど賭け額が増えないので破産率が低いのも特徴である。しかし、この手法ではマーチンゲール法のように1度勝っただけでは利益が増えない。つまり、連勝しなければ利益を得ることができない手法となっている。また実際のカジノでは、メモを見たりメモを取ることができないため、数列を暗記し、ややこしい計算を頭の中で行わなければならないといったデメリットもある。
\subsubsection{モンテカルロ法}
モンテカルロ法とは、これは最初の数列を1,2,3とし、数列の両端を足した数だけユニット数を賭ける手法である。負ける度に、数列に直前にその直前に賭けたユニット数を付け加え、勝つ度に配当が2倍のゲームでは記録の両端の数字を1つずつ削除し、配当が3倍のゲームでは両端の数字を2つずつ削除する。毎ゲーム時に賭けるユニット数は、勝ち負けに関わらず記録の両端の数字を合計したユニット数を賭ける。例えば、配当が2倍のゲームでは、1度目は1,2,3となっているので両端数字が1,3であり、4ユニット賭ける。この後負けた場合に、数列は1,2,3,4となり、両端数字が1,4となるので、次は5ユニット賭ける。ここで勝った場合は、両端数字の1,4を削除するので数列が2,3になるので次に賭けるユニット数は両端数字2,3なので5ユニット賭けることになる。ここで勝つと両端数字である。2,3が削除されるので、次に賭けるときは再び1,2,3の数列を使う。また配当が3倍のゲームでは、1度目は1,2,3となっているので両端数字が1,3であり、4ユニット賭ける。この後負けた場合に、数列は1,2,3,4となり、両端数字が1,4となるので、次は5ユニット賭ける。ここで負けた場合、数列が1,2,3,4,5となりり次に賭けるユニット数は両端数字が1,5なので6になる。ここで勝った場合、両端数字を2つずつ削除するので数列は3となり、数列の要素数が2個以下のため、次に賭けるときは再び1,2,3の数列を使う。特徴としては、2in1法と同じように、数列を利用する関係上カジノでは扱いにくい。また、この手法は配当が3倍のゲームでは儲かるが、2倍ゲームでは必ずしも利益が出て終わる手法ではないのでブラックジャックには向いていない。
\subsubsection{バーネット法}
バーネット法とは、賭けるユニット数を1,3,2,6のように変化させる手法である。最初に1ユニット賭け勝った場合に3,2,6のように賭けていく手法で、6ユニット賭けた後に勝った場合は、負けるまで6ユニットを賭け続け、負けた場合は数列の最初の1からまたかけ始めていく手法である。この手法の特徴として、連勝時には賭け金を上げて利益を得る。連敗時には賭け金を下げてリスクを減らすことができる。連勝や連敗に対しては非常に頼もしいが、勝ち負けが交互となったり、短いスパンで勝ち負けが同数で進行した場合は、効果を発揮することができない。
\bunseki{※柏田輝}

\subsection{資金の何割かを賭ける手法}
資金の何割かを賭ける手法には以下のような手法がある。
\begin{itemize}
\item 10%投資法
\item 全額投資法
\end{itemize}
\subsubsection{10\%投資法}
10\%投資法とは、自分がゲーム使える額全体のうちから1度のゲームにつき、その10\%を賭けるという手法である。この手法の特徴としては、絶対に破産しない上に計算が単純なので覚える必要がない。また、連敗すると賭け金が低くなるので、負けやすい間の不利益を抑えることができ、連勝すると賭け金が高くなるので、勝ちやすい間の利益が高くなる。しかし、今までのベッティング手法とは違い、負け額を確実に回収したり、利益を確実に出す手法ではないので、利益を得にくいといったデメリットがある。また、勝率が50\%を下回ってしまうと持ち金が0に収束していくというデメリットもある。
\subsubsection{全額投資法}
全額投資法とは、自分がゲーム使える額全体を1度のゲームですべて賭ける手法である。この手法の特徴としては、絶対に破産はしないが、1度負けると賭けることのできる金額を全て失うことになる。しかし、カジノでは基本的に長期的に勝負し続けると、少しずつプレイヤー側の利益がマイナスになることが多いのでその裏をかくことができる手法である。
\bunseki{※柏田輝}

\subsection{勝率に応じて賭け額を変化させる手法}
勝率に応じて賭け額を変化させる手法には以下のような手法がある。
\begin{itemize}
\item ケリー基準
\item ハーフケリー
\end{itemize}
\subsubsection{ケリー基準}
ケリー基準とは、以下のような公式から賭け額を決定する。
\begin{center}
(AP-Q)/A
\end{center}
Aは勝った場合に帰ってくる配当(デシマルオッズ)から1引いた値、つまり勝った場合の純粋な利益であり、Pは勝つ確率であり、Qは負ける確率である。これらを計算し出た値に1を足した値を賭け額のユニットとなる。例えば、基準となる賭け額が100、勝つ確率が52\%、負ける確率が48\%、デシマルオッズが2だった場合は、(1*0.52-0.48)/1=0.04となるので、賭け額は104となる。
\subsubsection{ハーフケリー}
ハーフケリーとは、以下のような公式をから賭け額を決定する。
\begin{center}
(AP-Q)/A/2
\end{center}
つまりケリー基準の出た値を半分にし、賭け額を決定する手法である。例えば、基準となる賭け額が100、勝つ確率が52\%、負ける確率が48\%、デシマルオッズが2だった場合は、(1*0.52-0.48)/1=0.04となり、その値の半分は0.02で、賭け額は102となる。つまり、ハーフケリーはケリー基準をローリスクローリターンにした手法であるといえる。これらの手法の特徴としては、他のベッティングシステムとは異なり勝率を参照するので、カウンティングを併用しやすいといったメリットがある。一方で、毎回勝率を計算しなければならないといったデメリットがある。
\bunseki{※柏田輝}

\subsection{その他の手法}
その他の方法には以下のような手法がある。
\begin{itemize}
\item ダランベール法
\item 逆ダランベール法
\item 31システム
\item ココモ法
\subsubsection{ダランベール法}
ダランベール法とは、勝負に負けたときにユニットを1増やし、勝ったときに1減らすという手法である。例えば、1ユニット100と設定した場合、1度負ければ100+100となり200、2度負ければ200+100となり300、1度勝てば300-100となり200となる。この手法は見た目の通りとても単純であり覚えやすいという特徴がある。また、勝っているときでも負けているときでも、利得の増減がゆるやかである。連敗しても長い目で見ればある程度マイナスになった分の利得を取り戻しやすいというメリットがある。しかし、勝ちと負けがが交互に繰り返された場合には効果がみられないというデメリットがある。この手法は、勝率が約50%であり、勝利した際の払い戻しが賭け金の2倍であるゲームで有効である。つまりブラックジャックで有効である。
\subsubsection{逆ダランベール法}
逆ダランベール法とは、勝負に勝ったときにユニットを1増やし、負けたときに1減らすという、ダランベール法と真逆の手法である。例えば、1ユニットを100と設定した場合、1度勝てば100+100となり200、2度勝てば200+100となり300、1度負ければ300-100となり200となる。ダランベール法と同じで、とても単純であり覚えやすいという特徴がある。また、連勝すれば急激に利得が増えるというメリットがある。逆に、連勝しなければ大きな利得を得ることができず、ダランベール法と同じく、勝ちと負けが繰り返された場合では利得がでにくいというデメリットがある。つまり、大勝ちはできないが逆に大負けもしなく、連敗してもダランベール法のように賭け金が膨大にならず、賭け金の上限に引っかかることもない。
\subsubsection{31システム}
21システムとは、負けた時にユニットを1,1,1,2,2,4,4,8,8の順番で増やし、勝ったときには現在の賭け金を2倍にし、2連勝したときに最初のユニットに戻る、というカウンティング手法である。また8連敗した場合も最初のユニットに戻る。これらの数字を全て足すと31になることから、31システムと呼ばれる。例えば、1ユニットを100と設定した場合、はじめの賭け金は100、1回目の勝負で負けた場合も100、2回目の勝負で負けた場合も100、3回目に勝利した場合には200となる。このカウンティング手法を使用し賭けていき、8敗したとしても、トータルの損失が31ユニットしかないため、資金管理がしやすく、また仕組み的に、2連勝すればどこのタイミングでも利益が出せるというメリットがある。この手法に似ているカウンティング手法としてマーチンゲール法があるが、そちらに比べて賭け金の増え方が緩やかであることに加え、賭け金の上限に引っかかる可能性も低い。
\subsubsection{ココモ法}
ココモ法とは、3回目以降の賭け金を、前回の賭け金+前々回の賭け金で導出するカウンティング手法である。1回目と2回目の賭け金は1ユニットである。例えば、1ユニットを100と設定した場合、1回目と2回目は100、3回目は100+100で200、4回目は200+100で300、5回目は300+200で500、6回目は500+300で800となる。例からわかる通り、ゲーム回数を重ねるにつれて、賭け金は指数関数的に増えてゆき、負けが連続しても一度勝利すればそれまでの損失分もしくはそれ以上の利益を得ることがメリットである。逆に賭け金の上限に引っかかった場合、破綻するというのがデメリットである。この手法に似ているカウンティング手法に、マーチンゲール法があるが、そちらに比べて賭け金の増え方は緩やかである。
この手法は、勝率が約30%であり、勝率した際の払い戻しが3倍のゲームで一番利得が出るようになっており、ブラックジャックのような勝率が約50%であり、勝利した際の払い戻しが2倍のゲームではあまり効果がみられないという特徴もある。
\bunseki{※菱田美紗紀}

\subsection{まとめ}
ベッティングシステムについて調査した結果、ベッティング手法は、勝ち負けに応じて賭け額を変化させる手法、数列を用意し、その数列を操作しながら賭け額を変化させる手法、資金の何割かを賭ける手法、勝率に応じて賭け額を変化させる手法、その他の手法の5種類のグループに分けることができた。そのうち、勝ち負けに応じて賭け額を変化させる手法と数列を用意し、その数列を操作しながら賭け額を変化させる手法、その他の手法は、カウンティング値を参照するのが最初の1度のみで、その後はカウンティング値が低くなっても、途中で賭け額を決めることができない。また、資金の何割かを賭ける手法についても、賭け額が資金によって決定されているのでカウンティング値を参照しない。これらの点から、勝率に応じて賭け額を変化させる手法がカウンティングに適用しやすい手法だと考えられる。
\bunseki{※柏田輝}
