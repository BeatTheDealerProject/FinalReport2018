\section{ベッティングシステムの検討}
 カウンティング時の適切なベッティングシステムが存在しないため、カウンティングに適用可能な手法を調査した。その結果ベッティングシステムは以下のようなグループに分けることができた。
\begin{itemize}
\item 勝ち負けに応じて一定の倍率で賭け額を変化させる手法
\item 数列を操作しながら賭け額を変化させる手法
\item 資金の何割かを賭ける手法
\item 勝率に応じて賭け額を変化させる手法
\end{itemize}
以上の4つについてこれから説明する。
\bunseki{※柏田輝}

\subsection{勝ち負けに応じて一定の倍率で賭け額を変化させる手法}
勝ち負けに応じて賭け額を変化させる手法には以下のような手法がある。
 \begin{itemize}
 \item マーチンゲール法
 \item グランマーチンゲール法
 \item パーレー法
 \item グランパーレー法
 \end{itemize}
\subsubsection{マーチンゲール法}  
 マーチンゲール法とは、基準となる賭け金(単位)を決めて、負けるたびに単位を倍に増やしていき、勝った時に単位数を1に戻すという手法である。例えば1単位を100と設定した場合、1度負けると100×2となり200、2度負けると200×2となり400となる。この手法の特徴としては、1度勝てば負け額をすべて取り戻し、1単位分だけ利益を出すことが出来る。しかし、負け続けると賭ける単位数が指数関数的に増えてしまい、破産しやすいというデメリットがある。
\subsubsection{グランマーチンゲール法} 
 グランマーチンゲール法とは、負けるたびに単位を2倍しその値に1を加えて勝つまで単位数を増やしていき、勝った時に単位数を1に戻す手法である。例えば1単位を100と設定した場合、1度負けると100×2+100となり300、2度負けると300×2+100となり700となる。この手法の特徴として、最初の1単位目を賭けてから勝った場合までのゲーム回数×1単位分の利益を出すことができる。つまり、マーチンゲール法をハイリスクハイリターンにした手法である。
\subsubsection{パーレー法}  
 パーレー法とは、勝つたびに単位数を2倍にする手法である。例えば、1単位を100と設定すると、1度1勝と100×2となり200、2度勝つと200*2で400となる。この手法の特徴として、少ない賭け金で大きな利益を出すことができる。しかし、一度でも負けてしまうと利益がマイナスになってしまうのである程度利益をした後、自分で単位数を1に戻す必要がある。
\subsubsection{グランパーレー法}  
 グランパーレー法とは、勝つたびに単位数を2倍にしその値に1を加えていく手法である。例えば1単位を100と設定した場合の賭け金は、1度勝つと100×2+100となり300、2度勝つと300×2+100となり700となる。この手法の特徴として、利益が指数関数的に増えていきます。しかし、負けた場合に最初の1単位目を賭けてからその時までのゲーム回数×1単位分負けてしまう。つまり、パーレー法をハイリスクハイリターンにした手法である。
\bunseki{※柏田輝}

subsection{数列を用意し、その数列を操作しながら賭け額を変化させる手法}
 数列を用意し、その数列を操作しながら賭け額を変化させる手法には以下のような手法がある。
  \begin{itemize}
 \item 2in1法
 \item モンテカルロ法
 \item バーネット法
 \end{itemize}
\subsubsection{2in1法}  
2in1法とは、数列の両端を足した数の単位数を賭ける手法である。この手法は2連敗した後に適用され、負ける度にその直前に賭けた単位数を右端に記録していき、勝つ度に記録の両端の数字を1つずつ削除する。毎ゲーム時に賭ける単位数は、勝ち負けに関わらず記録の両端の数字を合計した単位数を賭ける。例えば、2連続で負けた後の数列は{1,1}となり、ここから適用する。両端数字が1,1なので次は2単位賭ける。ここで負けた場合数列が{1,1,2}となり、次は3単位賭けることになる。更ににここで負けた場合数列が{1,1,2,3}となり次は4単位賭けることになる。ここで勝つと数列の両端の数字である1,3を削除し数列が{1,2}に変化するので、次は3単位賭けることになる。そこでまた勝つことにより、数列の両端の数字がなくなり、負けた分を全て回収できたことになる。この手法の特徴は、2回分の負け額を1度の賭けで回収のすることができる。また、マーチンゲール法ほど賭け額が増えないので破産率が低いのも特徴である。しかし、この手法ではマーチンゲール法のように1度勝っただけでは利益が増えない。つまり、連勝しなければ利益を得ることができない手法となっている。また実際のカジノでは、メモを見たりメモを取ることができないため、数列を暗記し、ややこしい計算を頭の中で行わなければならないといったデメリットもある。
\subsubsection{モンテカルロ法}
 モンテカルロ法とは、これは最初の数列を{1,2,3}とし、数列の両端を足した数だけ単位数を賭ける手法である。負ける度に、数列に直前にその直前に賭けた単位数を付け加え、勝つ度に配当が2倍のゲームでは記録の両端の数字を1つずつ削除し、配当が3倍のゲームでは両端の数字を2つずつ削除する。毎ゲーム時に賭ける単位数は、勝ち負けに関わらず記録の両端の数字を合計した単位数を賭ける。例えば、配当が2倍のゲームでは、1度目は{1,2,3}となっているので両端数字が1,3であり、4単位賭ける。この後負けた場合に、数列は{1,2,3,4}となり、両端数字が1,4となるので、次は5単位賭ける。ここで勝った場合は、両端数字の1,4を削除するので数列が{2,3}になるので次に賭ける単位数は両端数字2,3なので5単位賭けることになる。ここで勝つと両端数字である。2,3が削除されるので、次に賭けるときは再び{1,2,3}の数列を使う。また配当が3倍のゲームでは、1度目は{1,2,3}となっているので両端数字が1,3であり、4単位賭ける。この後負けた場合に、数列は{1,2,3,4}となり、両端数字が1,4となるので、次は5単位賭ける。ここで負けた場合、数列が{1,2,3,4,5}となりり次に賭ける単位数は両端数字が{1,5}なので6になる。ここで勝った場合、両端数字を2つずつ削除するので数列は{3}となり、数列の要素数が2個以下のため、次に賭けるときは再び{1,2,3}の数列を使う。特徴としては、2in1法と同じように、数列を利用する関係上カジノでは扱いにくい。また、この手法は配当が3倍のゲームでは儲かるが、2倍ゲームでは必ずしも利益が出て終わる手法ではないのでブラックジャックには向いていない。\\
\subsubsection{バーネット法}
 バーネット法とは、賭ける単位数を1,3,2,6のように変化させる手法である。最初に1単位賭け勝った場合に3,2,6のように賭けていく手法で、6単位賭けた後に勝った場合は、負けるまで6単位を賭け続け、負けた場合は数列の最初の1からまたかけ始めていく手法である。この手法の特徴として、連勝時には賭け金を上げて利益を得る。連敗時には賭け金を下げてリスクを減らすことができる。連勝や連敗に対しては非常に頼もしいが、勝ち負けが交互となったり、短いスパンで勝ち負けが同数で進行した場合は、効果を発揮することができない。
\bunseki{※柏田輝}

\subsection{資金の何割かを賭ける手法}
 資金の何割かを賭ける手法には以下のような手法がある。
  \begin{itemize}
 \item 10%投資法
 \item 全額投資法
 \end{itemize}
\subsubsection{10\%投資法}
 10\%投資法とは、自分がゲーム使える額全体のうちから一度のゲームにつき、その10\%を賭けるという手法である。この手法の特徴としては、絶対に破産しない上に計算が単純なので覚える必要がない。また、連敗すると賭け金が低くなるので、負けやすい間の不利益を抑えることができ、連勝すると賭け金が高くなるので、勝ちやすい間の利益が高くなる。しかし、今までのベッティング手法とは違い、負け額を確実に回収したり、利益を確実に出す手法ではないので、利益を得にくいといったデメリットがある。また、勝率が50\%を下回ってしまうと持ち金が0に収束していくというデメリットもある。
\subsubsection{全額投資法}全額投資法とは、自分がゲーム使える額全体を一度のゲームですべて賭ける手法である。この手法の特徴としては、絶対に破産はしないが、一度負けると賭けることのできる金額を全て失うことになる。しかし、カジノでは基本的に長期的に勝負し続けると、少しずつプレイヤー側の利益がマイナスになることが多いのでその裏をかくことができる手法である。
\bunseki{※柏田輝}

\subsection{勝率に応じて賭け額を変化させる手法}
 勝率に応じて賭け額を変化させる手法には以下のような手法がある。
  \begin{itemize}
 \item ケリー基準
 \item ハーフケリー
 \end{itemize}
\subsubsection{ケリー基準}
 ケリー基準とは、以下のような公式から賭け額を決定する。
\begin{center} (AP-Q)/A 
\end{center}
Aは勝った場合に帰ってくる配当(デシマルオッズ)から1引いた値、つまり勝った場合の純粋な利益であり、Pは勝つ確率であり、Qは負ける確率である。これらを計算し出た値に1を足した値を賭け額の単位とします。例えば、基準となる賭け額が100、勝つ確率が52\%、負ける確率が48\%、デシマルオッズが2だった場合は、(1*0.52-0.48)/1=0.04となるので、賭け額は104となる。
\subsubsection{ハーフケリー}
 ハーフケリーとは、以下のような公式をから賭け額を決定する。
\begin{center} (AP-Q)/A/2
\end{center}
つまりケリー基準の出た値を半分にし、賭け額を決定する手法である。例えば、基準となる賭け額が100、勝つ確率が52\%、負ける確率が48\%、デシマルオッズが2だった場合は、(1*0.52-0.48)/1=0.04となり、その値の半分は0.02で、賭け額は102となる。つまり、ハーフケリーはケリー基準をローリスクローリターンにした手法であるといえる。これらの手法の特徴としては、他のベッティングシステムとは異なり勝率を参照するので、カウンティングを併用しやすいといったメリットがある。一方で、毎回勝率を計算しなければならないといったデメリットがある。
\bunseki{※柏田輝}

\subsection{まとめ}
 ベッティングシステムについて調査した結果、ベッティング手法は、勝ち負けに応じて賭け額を変化させる手法、数列を用意し、その数列を操作しながら賭け額を変化させる手法、資金の何割かを賭ける手法、勝率に応じて賭け額を変化させる手法の4種類のグループに分けることができた。そのうち、勝ち負けに応じて賭け額を変化させる手法と数列を用意し、その数列を操作しながら賭け額を変化させる手法は、カウンティング値を参照するのが最初の一度のみで、その後はカウンティング値が低くなっても、途中で賭け額を決めることができない。また、資金の何割かを賭ける手法についても、賭け額が資金によって決定されているのでカウンティング値を参照しない。これらの点から、勝率に応じて賭け額を変化させる手法がカウンティングに適用しやすい手法だと考えられる。
\bunseki{※柏田輝}
