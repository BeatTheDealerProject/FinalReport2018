
\section{完全版のブラックジャックシミュレータ}
今回はブラックジャックを行った際の資金の推移や勝率の推移と言ったものを調査するために前期に作成したシミュレーターを改変した。具体的にはシミュレーターに所持金の概念を適用したり、実際の人間のように戦略を間違えるという処理を追加し、ダブルダウンやスプリット、サレンダーといったプレイヤー側の選択肢を追加した。このシミュレータをもとに様々な実験を行った。ここではシミュレータ内部の詳細について記述していく。
\bunseki{※尾崎拓海}

\subsubsection{基本設計}
まず初めに、シミュレータの基本設計について説明する。今回作成したシミュレータではブラックジャックを行う際に必要となる要素をクラスとして表現した。具体的にはトランプのカードを表現するカードクラスとそれを一纏めにするデッククラス、ゲーム参加者を表すクラスとそれを継承したプレイヤークラスとディーラークラス、ゲームの勝敗を判定するマネージャークラスのそれぞれを定義した。これらのクラスを用いてブラックジャックのゲームを再現し、ベーシックストラテジーとその他の戦略を実行するプログラムを作成した。次に各クラスの詳細を記述していく。
\bunseki{※尾崎拓海}

\subsubsection{トランプのカードを表現するクラス}
このクラスでは実際のトランプのカードを表現するためにrankという変数にA~Kというトランプのランクを、suitという変数にスペード、ハート、ダイヤ、クラブのスートを定義した。また、J,Q,K,Aの絵札カードは10や11と数える必要があったので、ランクを数字に変換する処理もこのクラスに書き、valueという変数に入力した。
\bunseki{※尾崎拓海}

\subsubsection{デックを表現するクラス}
このクラスでは先程定義したカードクラスを利用してデックを定義した。具体的には先程のカードクラスの配列を作成し、その中にジョーカーを除く52種類のトランプカードを作成した。このクラスの初期化時に使用するデックの数を指定する。また、デックのシャッフルには独自に作成した関数を使用した。このシャッフル関数はPython3のrandom関数を用いて独自に設計したものであり、引数にシャッフルを行う回数を指定する。カードの配列の長さが仮に52だった場合には、1~26番目のカードからランダムに取り出したカードと、27~52番目のからランダムに取り出したカードを交換するという処理を(デック数×指定されたシャッフル回数)繰り返すという処理でシャッフル関数を作成した。
\bunseki{※尾崎拓海}

\subsubsection{ゲーム参加者を表すスーパークラス}
このクラスでは自身の手札とその手札の合計値、手札に含まれるAの枚数、バーストしているかどうかのフラグ、手札がブラックジャックとなっているかどうかのフラグのそれぞれを定義している。手札に含まれるAの枚数は自身の手札の合計値を計算する時と、ブラックジャックの条件を満たしているかどうかを判別する際に使用した。また手札の合計値を返す関数を定義し、その内側で自身がバーストしているかどうかの判定も行っている。
\bunseki{※尾崎拓海}

\subsubsection{プレイヤークラス}
このクラスは先のゲーム参加者を表すスーパークラスを継承しており、ゲームに参加しているプレイヤーを表現している。プレイヤークラスでは新たに自身の名前を表す変数と自身の勝利回数、敗北回数を記録する変数を定義した。またこのクラスでは新しく、カードを受け取る関数とヒットを行う関数、スタンドを行う関数、勝利回数と敗北回数を増加させる関数を作成した。また、ヒット、スタンド、ダブルダウン、サレンダーの処理を行う関数を作成した。
\bunseki{※尾崎拓海}

\subsubsection{ディーラークラス}
このクラスは先のゲーム参加者を表すスーパークラスを継承しており、ゲームのディーラーを表現しているクラスとなっている。ディーラークラスの中でデックをインスタンス化してディーラー側がデックを所持している事を表
現している。このクラスでは新しく、デックのシャッフル回数という変数を定義した。また、このクラスではカードを配る関数、ディーラーの手札合計が17を超えるまでカードを引き続ける関数を作成した。カードを配る関数についてはデック数有限の時とデック数無限の時とで処理を変更している。また、カウンティングを行う際に使用されたカードを数える必要があったので、使用されたカードに対応してカウントを増減させる処理をカウンティング手法ごとに記述した。
\bunseki{※尾崎拓海}

\subsubsection{ゲームマネージャークラス}
このクラスは主にゲームの勝敗判定に使用している。プレイヤーとディーラーの手札の合計値を比較し勝敗を判定する関数と、手札がブラックジャックになっているかどうかを判定する関数を作成した。また、プレイヤーの勝敗に応じて資金を移動させる処理を記述した。勝敗判定のタイミングでプレイヤーの勝利回数、敗北回数のそれぞれを記録している。
\bunseki{※尾崎拓海}

\subsubsection{メイン関数}
以上のクラスを用いてメイン関数にブラックジャックのゲームを記述した。以下にプログラムの実行手順を示す。
\begin{enumerate}
    \item ゲームに参加するプレイヤーを作成。今回はプレイヤーを一人のみ作成した。
    \item ディーラーを作成。
    \item カットカードを定義。カットカードを挟む位置はデックの半分の位置とした。
    \item ゲーム全体の実行回数を定義。今回は10万回とした。
    \item プレイヤーの戦略を配列形式で定義した。
    \item ゲームを繰り返すwhile文を作成し、ループ回数を10万回とした。
    \begin{enumerate}
        \item デックからカットカードが出てきたかを確認する。もし出てきていればデックをシャッフルする。
	  \item ディーラーが自身を含む各プレイヤーに初期カードを配る。
	  \item プレイヤーは自身の戦略に応じて掛け金を決定する。
	  \item プレイヤーは自身の戦略に沿った行動を選択する。
	  \item プレイヤーは戦略ごとに一定の確率で間違えた行動を選択する。
	  \item すべてのプレイヤーの行動が終了したことを確認後にディーラーが行動を開始する。
	  \item ディーラーの行動終了後に、勝敗判定を行う。
    \end{enumerate}
\end{enumerate}
\bunseki{※尾崎拓海}

\subsubsection{エラー時の行動}
戦略の複雑性に応じてエラーのしやすさを定義した。エラーした際の行動は以下のようになる。
\begin{itemize}
    \item ヒットとスタンドのみで構成された戦略の場合
  \begin{itemize}
        \item ヒットでエラー:スタンド
        \item スタンドでエラー:ヒット
    \end{itemize}
    \item ヒット、スタンド、ダブルダウン、スプリットで構成された戦略の場合
    \begin{itemize}
        \item ヒットでエラー:80%スタンド、20%ダブルダウン
        \item スタンドでエラー:80%ヒット、20%ダブルダウン
        \item ダブルダウンでエラー:50%ヒット、50%スタンド
        \item スプリットでエラー:スプリットを行わない
    \end{itemize}
\end{itemize}
\bunseki{※尾崎拓海}