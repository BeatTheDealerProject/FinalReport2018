 本章では最終成果発表で記入してもらった評価シートの集計結果とコメントを参考にして今後の改善点を記述する。
 評価シートの評価項目は「発表技術」と「発表内容」の2つと「発表内容」の細部に「複雑性にちての説明の評価」、「カウンティングについての説明の評価」の2つ合わせて計4つ用意した。そして、それぞれについて1(非常に悪い)から10(非常に優秀)までの間で評価点を記入する欄、それぞれについてのコメント(評価理由)やアドバイスを記入する欄をそれぞれ用意した。
\bunseki{※葛西隼人}
\section{最終成果発表}
\subsection{評価点数の集計}
最終成果発表で記入してもらった評価シートは計50枚だった。シートを記入した人の所属の分布は表は\ref{tab:dist2} のようになった。

\begin{table}[htb]
  \begin{center}
    \caption{評価人数集計}
    \begin{tabular}{|c|c|c|} \hline 
      所属 & 学年 & 人数  \\ \hline \hline
      教員 &  & 7  \\
      一般 &  & 8 \\
      学生 & 院2年 & 0 \\
     学生 & 院1年 & 0 \\
             & 学部4年 & 0 \\
       & 学部3年 & 31 \\
             & 学部2年 & 2 \\
             & 学部1年 & 2 \\ \hline \hline
      合計 &  & 50 \\ \hline
    \end{tabular}
    \label{tab:dist2}
  \end{center}
\end{table}
次はそれぞれの評価項目についての平均点を表\ref{tab:point2}に記す。
\begin{table}[H]
\begin{center}
\caption{評価点数集計}
\begin{tabular}{|c|c|c|c|c|c|} \hline
  所属 & 学年 & 発表技術 & 発表内容 & 複雑性説明 & カウンティング説明  \\ \hline \hline
  教員 &        & 7.42 & 7.14 & 7.14 & 6.71 \\ 
  一般 &        & 7.38 & 7 & 6.12 & 6.75 \\
  学生 &        & 7 & 7.2 & 6.66 & 7.02 \\
         & 学部1,2年 & 6.5 & 6.25 & 6.5 & 6.5 \\
         & 学部3年 & 6.87 & 7.39 & 6.7 & 7.26 \\ \hline \hline
  全体 &        & 7 & 7.2 & 6.66 & 7.02 \\ \hline
\end{tabular}
\label{tab:point2}
\end{center}
\end{table}

全体の平均は「発表技術」については7、「発表内容」については7.2、「複雑性説明」については6.66、「カウンティング説明」については7.02となった。

\subsection{コメント解析と改善点}
まず、「発表技術」について、多かったコメントを肯定的なコメントと否定的なコメントに分けて並べる。

肯定的なコメント
\begin{itemize}
\item アニメーションがついてて見やすいスライドだと思った
\item アジェンダがあり、長いプレゼンがわかりやすかった
\item スライドの内容はよく整理されておりわかりやすかったです
\end{itemize}

否定的なコメント
\begin{itemize}
\item 発表場所が悪かったのもあり声がきこえないところがあった
\item 発表者が原稿を読んでいる感じがしたのが残念だった
\item 発表が単調で何が大事なポイントが分かりづらかった
\end{itemize}

スライド構成については評価するコメントはあったが、声が小さい、スライドの文字ばかり見ているとの発表者の技術面についての否定的な意見が多かった。そして発表の時間配分が他のプロジェクトと異なっていたため発表を全て聞けなかった人が少しいた。次に、「発表内容」についても同様に、多かったコメントを肯定的なコメントと否定的なコメントに分けて並べる。

肯定的なコメント
\begin{itemize}
\item 仮説・実験・検証が適切に行われていた
\item 実験・検証が多く、説得力をもたせている
\item 
\end{itemize}

否定的なコメント
\begin{itemize}
\item 用語についてより詳しく説明すべき
\item 聞く人の知識が必要になるのでもう少しわかりやすく
\end{itemize}
