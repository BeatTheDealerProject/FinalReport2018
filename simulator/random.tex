\section{シミュレータの擬似乱数の検証}
今回シミュレータを作成するにあたり、擬似乱数を使用した。この擬似乱数が適切かどうかについて検証する。今回使用した擬似乱数生成方法はPython3のrandom関数である。Python Software Foundation(2018)によればrandomの擬似乱数を生成するアルゴリズムはメルセンヌツイスタを用いている。今回は周期と広井(2007)の等確率性の検定を行う。
\bunseki{柿崎大輝}
\subsection{周期}
擬似乱数には周期が存在する。周期とは同じ数列が出てくるようになるまでの数字の出現回数のことを指す。周期が小さいとよく同じ数列が出てきてしまいランダム性が低い。つまり周期が大きいとランダム性が高いので、性能が良いということになる。松本(2013)ではメルセンヌツイスタの周期は$2^{19937}-1$である。これはほかの擬似乱数に比べ、かなり大きい周期である。そのため、メルセンヌツイスタを擬似乱数として使うのに十分であると考えられる。
\bunseki{柿崎大輝}
\subsection{等確率性の検定}
等確率性とはどの値も等しい確率で出てくるかどうかである。カイ2乗検定を使い等確率性を検証する。random関数を使用して、0~1の範囲
の乱数を生成する。その後、その値を0~0.1、0.1~0.2、0.2~0.3、0.3~0.4、0.4~0.5、0.5~0.6、0.6~0.7、0.7~0.8、0.8~0.9、0.9~1.0の10通りに分類する。それをまとめると表\ref{table:randomresult}になる。
\begin{table}[H]
 \caption{random関数での結果}
 \label{table:randomresult}
 \begin{center}
  \begin{tabular}{|c|c|c|c|c|c|c|c|c|c|}
    \hline    0~0.1 &  0.1~0.2 & 0.2~0.3 & 0.3~0.4 &  0.4~0.5 & 0.5~0.6 & 0.6~0.7 & 0.7~0.8 & 0.8~0.9 & 0.9~1.0 \\
    \hline 95 & 85 & 100 & 102 & 91 & 114 & 87 & 108 & 115 & 103 \\
    \hline
  \end{tabular}
 \end{center}
\end{table}
完璧なランダムなのであれば、この結果はどれも100になることが予想できる。しかし、実際はすべてが100にはならないので、カイ2乗検定を行い検証する。先ほど出た度数を実現度数として使用し、100を理論度数としてカイ2乗検定を行う。この時、自由度は9で有意水準を5%とすると、棄却値は16.92となり、カイ2乗値がこれより小さいと擬似乱数が等しく出てきたといえる。実際に計算すると、カイ2乗値は9.98となった。この値は16.92より小さいので、擬似乱数によって出た値はすべて等しい確率で出てきたといえる。

以上のことから、メルセンヌツイスタは周期が大きいこととカイ2乗検定で値が全て等しい確率で出ていることから十分に使えると判断することができるため、今回のシミュレータにおいて使用した。
\bunseki{柿崎大輝}