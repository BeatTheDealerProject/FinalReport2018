\lstset{ 
   basicstyle={\ttfamily\small}, %書体の指定 
   frame=tRBl, %フレームの指定 
   framesep=10pt, %フレームと中身(コード)の間隔 
   breaklines=true, %行が長くなった場合の改行 
   linewidth=12cm, %フレームの横幅 
   lineskip=-0.5ex, %行間の調整 
   tabsize=2 %Tabを何文字幅にするかの指定 
}

\begin{itemize}
\item{シミュレータに使われるクラス}
\input{appendix/crass_Card.tex}
\newpage
\input{appendix/crass_Deck.tex}
\newpage
\input{appendix/crass_Dealer.tex}
\newpage
\begin{itemize}
\item ゲームの勝敗や掛け金の受け渡しを管理するクラス
\begin{lstlisting}
class GameManager:
    def __init__(self, players, dealer):
        self.players = players
        self.dealer = dealer
        self.checkdeal = True

    # 各プレイヤーとディーラーとの間で勝敗を決める
    def judge(self):
        for x in self.players:
            self.checkblackjack(x)
        self.checkblackjack(self.dealer)
        for player in self.players:
            if not player.surrendeflg:
                # プレイヤーがバーストした場合
                if player.burst == True:
                    if player.tag == "clone":
                        for i, x in enumerate(self.players):
                            if x.name == player.name:
                                self.players[i].addtotallose(player.betMoney)
                                break
                    player.addtotallose(player.betMoney)

                # プレイヤーがバーストせずにディーラーがバーストした場合
                elif player.burst == False and self.dealer.burst == True:
                    # スプリットしているかどうかのフラグ
                    spflg = False
                    for x in self.players:
                        if x.tag == "clone":
                            spflg = True

                    if player.tag == "clone":
                        for i, x in enumerate(self.players):
                            if x.name == player.name:
                                if player.naturalbj and not spflg:
                                    self.players[i].addtotalwin(player.betMoney*1.5)
                                    break
                                else:
                                    self.players[i].addtotalwin(player.betMoney)
                                    break
                    if player.naturalbj and not spflg:
                        player.addtotalwin(player.betMoney*1.5)
                    else:
                        player.addtotalwin(player.betMoney)

                # プレイヤーのトータルがディーラーのトータルよりも多い場合
                elif player.total > self.dealer.total:
                    spflg = False
                    for x in self.players:
                        if player.tag=="clone":
                            spflg = True

                    if player.tag == "clone":
                        for i, x in enumerate(self.players):
                            if x.name == player.name:
                                if player.naturalbj and not spflg:
                                    self.players[i].addtotalwin(player.betMoney*1.5)
                                    break
                                else:
                                    self.players[i].addtotalwin(player.betMoney)
                                    break
                    if player.naturalbj and not spflg:
                        player.addtotalwin(player.betMoney*1.5)
                    else:
                        player.addtotalwin(player.betMoney)

                # プレイヤーのトータルがディーラーのトータルよりも少ない場合
                elif player.total < self.dealer.total:
                    if player.tag == "clone":
                        for i, x in enumerate(self.players):
                            if x.name == player.name:
                                self.players[i].addtotallose(player.betMoney)
                                break
                    player.addtotallose(player.betMoney)

                # プレイヤーのトータルとディーラーのトータルが同じ場合
                elif player.total == self.dealer.total:
                    # プレイヤーがナチュラルブラックジャックかつディーラーがナチュラルブラックジャック
                    if player.naturalbj and self.dealer.naturalbj:
                        if player.tag == "clone":
                            for i, x in enumerate(self.players):
                                if x.name == player.name:
                                    self.players[i].addtotaldraw()
                                    break
                        player.addtotaldraw()
                    # プレイヤーがナチュラルブラックジャックかつディーラーがノーマルブラックジャック
                    elif player.naturalbj and self.dealer.normalbj:
                        if player.tag == "clone":
                            for i, x in enumerate(self.players):
                                if x.name == player.name:
                                    self.players[i].addtotalwin(player.betMoney * 1.5)
                                    break
                        player.addtotalwin(player.betMoney * 1.5)
                    # プレイヤーがノーマルブラックジャックかつディーラーがナチュラルブラックジャック
                    elif player.normalbj and self.dealer.naturalbj:
                        if player.tag == "clone":
                            for i, x in enumerate(self.players):
                                if x.name == player.name:
                                    self.players[i].addtotallose(player.betMoney)
                                    break
                        player.addtotallose(player.betMoney)
                    # プレイヤーがノーマルブラックジャックかつディーラーがノーマルブラックジャック
                    elif player.normalbj and self.dealer.normalbj:
                        if player.tag == "clone":
                            for i, x in enumerate(self.players):
                                if x.name == player.name:
                                    self.players[i].addtotaldraw()
                                    break
                        player.addtotaldraw()
                    else:
                        if player.tag == "clone":
                            for i, x in enumerate(self.players):
                                if x.name == player.name:
                                    self.players[i].addtotaldraw()
                                    break
                        player.addtotaldraw()

    # ナチュラルブラックジャックとノーマルブラックジャックを判別する関数
    # 入力にプレイヤー個人またはディーラ-個人を与える
    def checkblackjack(self, player):
        if player.total == 21:
            if len(player.cards) == 2:
                player.naturalbj = True
            else:
                player.normalbj = True

\end{lstlisting}
\end{itemize}
\newpage
\newpage
\begin{itemize}
\item ゲーム参加者を表すスーパークラス
\begin{lstlisting}
class GamePlayer:

    # 初期化関数
    def __init__(self):
        self.cards = []
        self.total = 0
        self.acetotal = 0
        self.usedace = 0
        self.burst = False
        self.naturalbj = False
        self.normalbj = False
        self.surrendeflg = False

    #  子オブジェクトから呼び出せる初期化関数
    def initialize(self):
        self.cards = []
        self.total = 0
        self.acetotal = 0
        self.usedace = 0
        self.burst = False
        self.naturalbj = False
        self.normalbj = False
        self.surrendeflg = False

    # ゲームプレイヤーの手札の合計値を返す関数
    def totalvalue(self):
        i = 0
        self.total = 0
        self.acetotal = 0
        cardnum = len(self.cards)

        while i < cardnum:
            if self.cards[i].rank == 'A':
                self.acetotal += 1
            self.total += self.cards[i].value
            i += 1
        self.total -= 10 * self.usedace

        # プレイヤーのバースト判定の処理
        if self.total > 21:
            if self.acetotal - self.usedace > 0:
                self.total -= 10
                self.usedace += 1
                if self.total > 21:
                    self.burst = True
            else:
                self.burst = True

\end{lstlisting}
\end{itemize}
\newpage
\input{appendix/crass_Player.tex}
\newpage
\input{appendix/main_program.tex}
\end{itemize}