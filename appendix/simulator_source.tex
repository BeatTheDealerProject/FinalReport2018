\lstset{ 
   basicstyle={\ttfamily\small}, %書体の指定 
   frame=tRBl, %フレームの指定 
   framesep=10pt, %フレームと中身(コード)の間隔 
   breaklines=true, %行が長くなった場合の改行 
   linewidth=12cm, %フレームの横幅 
   lineskip=-0.5ex, %行間の調整 
   tabsize=2 %Tabを何文字幅にするかの指定 
}

\begin{itemize}
\item{シミュレータに使われるクラス}
\begin{itemize}
\item トランプのカードを表現するクラス
\begin{lstlisting}
class Card:
    RANKS = ('A', '2', '3', '4', '5', '6', '7', '8', '9', '10', 'J', 'Q', 'K')
    SUITS = ('Spade', 'Heart', 'Diamond', 'Club')

    # 初期化
    def __init__(self, rank, suit):
        self.rank = rank
        self.suit = suit
        self.value = int(self.getvalue())

    # ランクを数字に変換する
    def getvalue(self):
        if self.rank == 'A':
            return 11
        elif self.rank == 'J' or self.rank == 'Q' or self.rank == 'K':
            return 10
        else:
            return self.rank
\end{lstlisting}
\end{itemize}
\newpage
\newpage
\begin{itemize}
\item デックを表現するクラス
\begin{lstlisting}
class Deck:
    CARDS = [Card(rank, suit) for suit in Card.SUITS for rank in Card.RANKS]
    Cards = []
    BaseDeck = []
    for rank in Card.RANKS:
        for suit in Card.SUITS:
            # オブジェクト共有を回避するための基本となる一デッキ
            BaseDeck.append(Card(rank, suit))  

    # 初期化
    # decNum の数だけデッキを使用する
    def __init__(self, decNum):
        basedec = []
        while (decNum > 0):
            basedec += self.BaseDeck
            decNum -= 1
        self.Cards = basedec
        self.current = 0

    # シャッフルをする関数
    # 引数に入れる数字によりシャッフルの回数を制御
    def shuffle(self, shuffleNum):
        self.current = 0
        while shuffleNum > 0:
            cut1 = random.randrange(0, len(self.Cards) / 2)
            cut2 = random.randrange(len(self.Cards) / 2, len(self.Cards))
            temp = self.Cards[cut1]
            self.Cards[cut1] = self.Cards[cut2]
            self.Cards[cut2] = temp
            shuffleNum -= 1
\end{lstlisting}
\end{itemize}
\newpage
\newpage
\begin{itemize}
\item ディーラーを表現するクラス
\begin{lstlisting}
class Dealer(GamePlayer):
    # ディーラーの初期化
    def __init__(self, deckNum):
        self.deck = Deck(deckNum)
        self.totaldealerhandlist = [0] * 6
        # ディーラーがシャッフルする回数。今回は10000回シャッフルする。
        self.shufflenum = 10000
        self.deck.shuffle(deckNum * self.shufflenum)
        # ランニングカウント
        self.IRC = 0
        super().__init__()

    # カードを配る関数
    def dealcard(self):

        # 無限デック想定の場合
        card = Card(Card.RANKS[random.randrange(13)], "spade")
        return card
        """
        # 有限デック想定の場合
        # HiLow
        card = self.deck.Cards[self.deck.current]
        if 2 <= card.value <= 6:
            self.IRC = self.IRC + 1
        elif 10 <= card.value <= 11:
            self.IRC = self.IRC - 1

        self.deck.current += 1
        if self.deck.current == len(self.deck.Cards):
            self.deck.current = 0
        return card
        """


    # 一番最初にカードを配る際の関数
    def firstdeal(self, player):
        super().__init__()
        for x in player:
            x.initialize()
        firstdeal = 2
        while firstdeal > 0:
            self.cards.append(self.dealcard())
            for x in player:
                x.cards.append(self.dealcard())
            firstdeal -= 1

    # 合計が17を超えるまで続ける処理
    def continuehit(self):
        self.totalvalue()
        while (self.total < 17):
            self.cards.append(self.dealcard())
            self.totalvalue()


\end{lstlisting}
\end{itemize}
\newpage
\newpage
\begin{itemize}
\item ゲームの勝敗や掛け金の受け渡しを管理するクラス
\begin{lstlisting}
class GameManager:
    def __init__(self, players, dealer):
        self.players = players
        self.dealer = dealer
        self.checkdeal = True

    # 各プレイヤーとディーラーとの間で勝敗を決める
    def judge(self):
        for x in self.players:
            self.checkblackjack(x)
        self.checkblackjack(self.dealer)
        for player in self.players:
            if not player.surrendeflg:
                # プレイヤーがバーストした場合
                if player.burst == True:
                    if player.tag == "clone":
                        for i, x in enumerate(self.players):
                            if x.name == player.name:
                                self.players[i].addtotallose(player.betMoney)
                                break
                    player.addtotallose(player.betMoney)

                # プレイヤーがバーストせずにディーラーがバーストした場合
                elif player.burst == False and self.dealer.burst == True:
                    # スプリットしているかどうかのフラグ
                    spflg = False
                    for x in self.players:
                        if x.tag == "clone":
                            spflg = True

                    if player.tag == "clone":
                        for i, x in enumerate(self.players):
                            if x.name == player.name:
                                if player.naturalbj and not spflg:
                                    self.players[i].addtotalwin(player.betMoney*1.5)
                                    break
                                else:
                                    self.players[i].addtotalwin(player.betMoney)
                                    break
                    if player.naturalbj and not spflg:
                        player.addtotalwin(player.betMoney*1.5)
                    else:
                        player.addtotalwin(player.betMoney)

                # プレイヤーのトータルがディーラーのトータルよりも多い場合
                elif player.total > self.dealer.total:
                    spflg = False
                    for x in self.players:
                        if player.tag=="clone":
                            spflg = True

                    if player.tag == "clone":
                        for i, x in enumerate(self.players):
                            if x.name == player.name:
                                if player.naturalbj and not spflg:
                                    self.players[i].addtotalwin(player.betMoney*1.5)
                                    break
                                else:
                                    self.players[i].addtotalwin(player.betMoney)
                                    break
                    if player.naturalbj and not spflg:
                        player.addtotalwin(player.betMoney*1.5)
                    else:
                        player.addtotalwin(player.betMoney)

                # プレイヤーのトータルがディーラーのトータルよりも少ない場合
                elif player.total < self.dealer.total:
                    if player.tag == "clone":
                        for i, x in enumerate(self.players):
                            if x.name == player.name:
                                self.players[i].addtotallose(player.betMoney)
                                break
                    player.addtotallose(player.betMoney)

                # プレイヤーのトータルとディーラーのトータルが同じ場合
                elif player.total == self.dealer.total:
                    # プレイヤーがナチュラルブラックジャックかつディーラーがナチュラルブラックジャック
                    if player.naturalbj and self.dealer.naturalbj:
                        if player.tag == "clone":
                            for i, x in enumerate(self.players):
                                if x.name == player.name:
                                    self.players[i].addtotaldraw()
                                    break
                        player.addtotaldraw()
                    # プレイヤーがナチュラルブラックジャックかつディーラーがノーマルブラックジャック
                    elif player.naturalbj and self.dealer.normalbj:
                        if player.tag == "clone":
                            for i, x in enumerate(self.players):
                                if x.name == player.name:
                                    self.players[i].addtotalwin(player.betMoney * 1.5)
                                    break
                        player.addtotalwin(player.betMoney * 1.5)
                    # プレイヤーがノーマルブラックジャックかつディーラーがナチュラルブラックジャック
                    elif player.normalbj and self.dealer.naturalbj:
                        if player.tag == "clone":
                            for i, x in enumerate(self.players):
                                if x.name == player.name:
                                    self.players[i].addtotallose(player.betMoney)
                                    break
                        player.addtotallose(player.betMoney)
                    # プレイヤーがノーマルブラックジャックかつディーラーがノーマルブラックジャック
                    elif player.normalbj and self.dealer.normalbj:
                        if player.tag == "clone":
                            for i, x in enumerate(self.players):
                                if x.name == player.name:
                                    self.players[i].addtotaldraw()
                                    break
                        player.addtotaldraw()
                    else:
                        if player.tag == "clone":
                            for i, x in enumerate(self.players):
                                if x.name == player.name:
                                    self.players[i].addtotaldraw()
                                    break
                        player.addtotaldraw()

    # ナチュラルブラックジャックとノーマルブラックジャックを判別する関数
    # 入力にプレイヤー個人またはディーラ-個人を与える
    def checkblackjack(self, player):
        if player.total == 21:
            if len(player.cards) == 2:
                player.naturalbj = True
            else:
                player.normalbj = True

\end{lstlisting}
\end{itemize}
\newpage
\newpage
\begin{itemize}
\item ゲーム参加者を表すスーパークラス
\begin{lstlisting}
class GamePlayer:

    # 初期化関数
    def __init__(self):
        # 参加者の手札
        self.cards = []  
        # 参加者の手札の合計値
        self.total = 0  
        # 参加者の手札に含まれるAの枚数
        self.acetotal = 0  
        # 1として数えたAの枚数
        self.usedace = 0  
        # バーストしているかどうか
        self.burst = False  
        # ナチュラルブラックジャックを満たしているかどうか
        self.naturalbj = False 
        # 手札の合計値が21 になっているかどうか
        self.normalbj = False  

    # 子オブジェクトから呼び出せる初期化関数
    def initialize(self):
        self.cards = []
        self.total = 0
        self.acetotal = 0
        self.usedace = 0
        self.burst = False
        self.naturalbj = False
        self.normalbj = False

    # ゲームプレイヤーの手札の合計値を返す関数
    def totalvalue(self):
        i = 0
        self.total = 0
        self.acetotal = 0
        cardnum = len(self.cards)

        while i < cardnum:
            if (self.cards[i].rank == 'A'):
                self.acetotal += 1
            self.total += self.cards[i].value
            i += 1
        self.total -= 10 * self.usedace

        # プレイヤーのバースト判定の処理
        if (self.total > 21):
            if (self.acetotal - self.usedace > 0):
                self.total -= 10
                self.usedace += 1
                if (self.total > 21):
                    self.burst = True
            else:
                self.burst = True

\end{lstlisting}
\end{itemize}

\begin{itemize}
\item プレイヤークラス
\begin{lstlisting}
class Player(GamePlayer):
    # プレイヤーの初期化
    def __init__(self, name):
        self.name = name  # プレイヤーの名前
        self.totalwin = 0  # プレイヤーの勝利回数
        self.totallose = 0  # プレイヤーの敗北回数
        super().__init__()

    # プレイヤーがカードを受け取る時に使用する関数
    def dealedcard(self, card):
        self.cards.append(card)

    # プレイヤー側のヒットの処理
    def hit(self, dealer):
        self.dealedcard(dealer.dealcard())
        self.showhands()

    # プレイヤー側のスタンドの処理
    def stand(self):
        pass

    # プレイヤーの勝利回数を増やす
    def addtotalwin(self):
        self.totalwin += 1

    # プレイヤーの敗北回数を増やす
    def addtotallose(self):
        self.totallose += 1

\end{lstlisting}
\end{itemize}

\begin{itemize}
\item ディーラークラス(デック数有限)
\begin{lstlisting}
class Dealer(GamePlayer):
    # ディーラーの初期化
    def __init__(self, deckNum):
        self.deck = Deck(deckNum)
        # ディーラーがシャッフルする回数。今回は一万回シャッフルする。
        self.shufflenum = 10000
        self.deck.shuffle(deckNum * self.shufflenum)
        super().__init__()

    # カードを配る関数
    def dealcard(self):
        ''' デック数有限の際はこちらのコメントアウトを解除する '''
        card = self.deck.Cards[self.deck.current]
        self.deck.current += 1
        
        ''' デック数無限の際にはこちらのコメントアウトを解除する '''
        # randomcard = random.randrange(13);
        # card = Card(Card.RANKS[randomcard], Card.SUITS[0])
        # return card

    # 一番最初にカードを配る際の関数
    def firstdeal(self, player):
        super().__init__()
        for x in player:
            x.initialize()
        firstdeal = 2
        while firstdeal > 0:
            self.cards.append(self.dealcard())
            for x in player:
                x.cards.append(self.dealcard())
            firstdeal -= 1

    # 合計が17 を超えるまで引き続ける処理
    def continuehit(self):
        self.totalvalue()
        while (self.total < 17):
            self.cards.append(self.dealcard())
            self.totalvalue()

\end{lstlisting}
\end{itemize}

\begin{itemize}
\item ゲームマネージャークラス
\begin{lstlisting}
class GameManager:
    def __init__(self, players, dealer):
        self.players = players
        self.dealer = dealer
        self.checkdeal = True

    # 各プレイヤーとディーラーとの間で勝敗を決める
    def judge(self):
        for x in self.players:
            self.checkblackjack(x)
        self.checkblackjack(self.dealer)
        for player in self.players:
            if player.burst == True:
                player.addtotallose()
            elif player.burst == False and self.dealer.burst == True:
                player.addtotalwin()
            elif player.total > self.dealer.total:
                player.addtotalwin()
            elif player.total < self.dealer.total:
                player.addtotallose()
            elif player.total == self.dealer.total:
                if player.naturalbj and self.dealer.naturalbj:
                elif player.naturalbj and self.dealer.normalbj:
                    player.addtotalwin()
                elif player.normalbj and self.dealer.naturalbj:
                    player.addtotallose()
                elif player.normalbj and self.dealer.normalbj:
                    pass
                else:
                    pass
\end{lstlisting}
\end{itemize}
\newpage
\begin{itemize}
\item プレイヤーを表現するクラス
\begin{lstlisting}
class Player(GamePlayer):
    # プレイヤーの初期化
    def __init__(self, name, money=0, betMoney=0, tag="player"):
        # プレイヤー名
        self.name = name
        # 所持金
        self.money = money
        # ベット額(クローンに値を渡す際に使用する)
        self.betMoney = betMoney
        # 累計勝利回数、敗北回数、引き分け回数
        self.totalwin = 0
        self.totallose = 0
        self.totaldraw = 0
        self.totalsplit = 0
        self.totalsurrender = 0
        self.totalplayerhandlist = [0] * 12
        # 勝利したか負けたかの確認
        self.winlose = ""
        # プレイヤーとクローンを見分ける
        self.tag = tag
        self.debagtxt = ""
        super().__init__()

    # プレイヤーにカードを配るときに使用する関数
    def dealedcard(self, card):
        self.cards.append(card)
        self.totalvalue()

    # プレイヤー側のヒットの処理
    def hit(self, dealer):
        self.dealedcard(dealer.dealcard())
        self.debagtxt += "H"

    # プレイヤー側のスタンドの処理
    def stand(self):
        self.debagtxt += "S"
        pass

    # プレイヤ-側のダブルダウンの処理
    def doubledown(self, dealer):
        self.debagtxt += "D("
        self.betMoney *= 2
        self.hit(dealer)
        self.stand()
        self.debagtxt += ")"

    # サレンダーの処理
    def surrender(self):
        self.debagtxt += "R"
        self.totalsurrender += 1
        # self.surrenderflg = True
        self.money -= self.betMoney/2

    # プレイヤー側のベットの処理
    def bet(self, betMoney):
        self.betMoney = betMoney

    # プレイヤーのインシュランスの処理
    def insurance(self, dealer):
        if dealer.cards[0] + dealer.cards[1] == 21:
            self.money += self.betMoney
        else:
            self.money -= self.betMoney/2

    # 自身の手札を表示するUI
    def showhands(self):
        for x in self.cards:
             print('/', x.suit, x.rank)
        print("---total---: ", self.total, "\n")

    # プレイヤーの勝利回数を増やす
    def addtotalwin(self, money):
        self.money += money
        self.totalwin += 1
        self.winlose = "win"

    # プレイヤーの敗北回数を増やす
    def addtotallose(self, money):
        self.money -= money
        self.totallose += 1
        self.winlose = "lose"

    # プレイヤーの引き分け回数を増やす
    def addtotaldraw(self):
        self.totaldraw += 1
        self.winlose = "draw"

\end{lstlisting}
\end{itemize}
\newpage
\newpage
\begin{itemize}
\item デックを表現するクラス
\begin{lstlisting}
def main(strategy):
    # プレイヤーを作成
    p1 = Player("player1")

    # ゲームに参加するプレイヤーを表現
    players = [p1]

    # ディーラーの作成
    # 引数はゲームに使用するデッキの数を表現
    dealer = Dealer(6)

    # カットカードを表現
    # 今回はデッキの1/2の位置にカットカードを固定している
    cutcard = len(dealer.deck.Cards) / 2

    # ゲーム全体のループ回数
    totalGameNum = 40
    remainingGameNum = 0

    # 最小ベットの宣言
    minbet = 10

    # 最大ベットの宣言
    maxbet = 10000

    trial = 50000
    trialNum = trial

    split_strategy = [["H", "H", "P", "P", "P", "P", "H", "H", "H", "H"],  # 2,2
                      ["P", "P", "P", "P", "P", "P", "H", "H", "H", "H"],  # 3,3
                      ["H", "H", "H", "P", "P", "H", "H", "H", "H", "H"],  # 4,4
                      ["D", "D", "D", "D", "D", "D", "D", "D", "H", "H"],  # 5,5
                      ["P", "P", "P", "P", "P", "H", "H", "H", "H", "H"],  # 6,6
                      ["P", "P", "P", "P", "P", "P", "H", "H", "H", "H"],  # 7,7
                      ["P", "P", "P", "P", "P", "P", "P", "P", "P", "P"],  # 8,8
                      ["P", "P", "P", "P", "P", "S", "P", "P", "S", "S"],  # 9,9
                      ["S", "S", "S", "S", "S", "S", "S", "S", "S", "S"],  # 10,10
                      ["P", "P", "P", "P", "P", "P", "P", "P", "P", "P"]  # A,A
                      ]

    while True:
        # メインループ
        remainingGameNum = totalGameNum
        dealer.deck.shuffle(dealer.shufflenum)
        players[0].money = 1000
        dealer.IRC = 0
        if trialNum \% 100 == 0:
            print(trialNum)

        while True:
            if remainingGameNum \% 100 == 0:
                print(remainingGameNum)

            # ゲームを始める前にデッキの中からカットカードが出てきているかを確認し、出てきていれば、デッキをシャッフルする
            """
            # 有限デックの際のシャッフルを行うタイミングの記述
            if dealer.deck.current > cutcard:
                dealer.deck.shuffle(dealer.shufflenum)
                dealer.IRC = -20
            """

            # ランニングカウントによって賭け額を変化させる
            # 今回はHigh-Low法を適用している
            rc = 1
            if dealer.IRC <= 1:
                rc = 1
            else:
                rc = dealer.IRC

            if players[0].money == 0:
                money_list.append(None)
                card_num_list.append(None)

                # ループの処理
                remainingGameNum -= 1
                if remainingGameNum <= 0:
                    break

                # 各プレイヤーのベット
                for player in players:
                    if player.money < minbet * rc:
                        player.bet(player.money)
                    else:
                        player.bet(minbet * rc)

                # 参加プレイヤーの初期化を実行後、ディーラーが各プレイヤー(自身含む)に初期カードを配る
                dealer.firstdeal(players)

                # 各プレイヤーの点数を更新
                for player in players:
                    player.totalvalue()

                # サレンダーの回数を記録する
                surrenderCounter = 0

                # スプリットの回数とダブルダウンの回数を記録する
                spCounter = 0
                ddCounter = 0

                # スプリットするかどうかを確認する
                for i, player in enumerate(players):
                    if player.cards[0].rank == player.cards[1].rank:
                        usermessage = split_strategy[player.cards[0].value - 2][dealer.cards[0].value - 2]
                        if (usermessage == 'P' or usermessage == 'p') and player.money > 2 * player.betMoney:
                            # プレイヤーのクローンを作成し、ゲームに参加するプレイヤーとして追加登録する
                            playerClone = Player(player.name,money=player.money - player.betMoney, betMoney=player.betMoney, tag="clone")
                            players.insert(i + 1, playerClone)
                            spCounter += 1

                            # クローンにプレイヤーが所持しているカードを一枚渡す
                            playerClone.dealedcard(player.cards[1])
                            del player.cards[1]

                            # 使用済みAの数を初期化する
                            player.usedace = 0
                            playerClone.usedace = 0

                            # プレイヤーとクローンにカードを配り直す
                            player.dealedcard(dealer.dealcard())
                            playerClone.dealedcard(dealer.dealcard())

                for player in players:
                    if player.tag == "clone":
                        players[0].totalsplit += 1


                # 各プレイヤーに対して選択肢を提示する
                for player in players:
                    while True:
                        # プレイヤーの選択はベーシックストラテジーに沿って行われるものとする
                        # プレイヤーの手札にA(11)が残っている場合
                        if player.acetotal - player.usedace > 0:
                            usermessage = strategy[player.total + 6][dealer.cards[0].value - 2]
                        # プレイヤーの手札にA(11)が残っていない場合
                        else:
                            usermessage = strategy[player.total - 4][dealer.cards[0].value - 2]

                        # 一定条件下でプレイヤーが戦略を間違える際の処理


                        # プレイヤーの選択による行動の分岐を記述
                        # プレイヤーがヒットを選択した場合
                        if usermessage == 'H' or usermessage == 'h':
                            player.hit(dealer)
                            if player.burst:
                                break

                        # プレイヤーがスタンドを選択した場合
                        elif usermessage == 'S' or usermessage == 's':
                            player.stand()
                            break

                        # プレイヤーがダブルダウンを選択した場合
                        elif usermessage == 'D' or usermessage == 'd':
                            if players[0].money -spCounter*players[0].betMoney - ddCounter*players[0].betMoney > player.betMoney * 2:
                                # ヒット後にはダブルダウンの選択不可
                                if len(player.cards) == 2:
                                    player.doubledown(dealer)
                                    ddCounter += 1
                                    break
                                else:
                                    player.hit(dealer)
                                    if player.burst:
                                        break
                            else:
                                player.hit(dealer)
                                if player.burst:
                                    break

                        # プレイヤーがサレンダーを選択した場合
                        elif usermessage == "R" or usermessage == "r":
                            # ヒット後にはサレンダーの選択不可
                            if len(player.cards) == 2:
                                player.surrender()
                                if player.tag == "clone":
                                   players[0].totalsurrender += 1
                                surrenderCounter += 1
                                break
                            else:
                                player.hit(dealer)
                                if player.burst:
                                    break

                # ディーラーは17を超えるまでヒットを続ける
                dealer.continuehit()

                # GameManagerの初期化
                gamemanager: GameManager = GameManager(players, dealer)

                # 勝敗を判定する
                gamemanager.judge()

                # クローンを削除する
                while True:
                    cloneflg = False
                    for i, player in enumerate(players):
                        if player.tag == "clone":
                            del player
                            del players[i]
                            cloneflg = True
                    if not cloneflg:
                        break

                # ループの処理
                remainingGameNum -= 1
                if remainingGameNum <= 0:
                    break
        trialNum -= 1
\end{lstlisting}
\end{itemize}
\newpage
\end{itemize}