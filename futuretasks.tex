我々が一年を通して活動した結果、今後の課題は以下の2点であると判断した。

\begin{itemize}
\item 利得をプラスにできる戦略の探索
\item ブラックジャックをプレイする際に重要な能力の調査
\end{itemize}

このプロジェクトで最終的に一番優秀と判断された戦略を用いてシミュレーションをしたところ、最終的なプレイヤーの利得はマイナスであった。また、賭け金の賭け方からアプローチするため、High-Low法とKO法というカウンティング手法を、基本戦略、ヒットとスタンドのみの基本戦略、今回導出されたGA戦略の3つの戦略を用いて賭け金のシミュレーションを行ったが、いずれも利得がプラスになることはなかった。

我々のプロジェクトの目的は、ディーラーをやっつける、すなわちカジノ側から利益を得ることが大本の目的であるため、まだ目的を達成したとは言えない。

なぜこのような結果となったのか、我々の予想としては、今回の遺伝的アルゴリズムのシミュレーションにはエラー率という概念が導入されていないためと思われる。

複雑性は単にその戦略の複雑さを数値にしただけであるが、エラー率はそれに加え、人間がどれだけ戦略を間違えやすいかという要素が含まれている。

戦略の勝率を導出するために作成したシミュレータには、エラー率の概念が導入されているが、その戦略を探索する遺伝的アルゴリズムのプログラムには導入されていなかった。つまり、戦略を作成する段階と、その戦略を評価する段階で、評価指標が異なるものであったのである。そのため、作成段階では優秀とされた戦略が、評価するにあたって性能が良くないと判断されてしまったのである。

以上のことより、利得をプラスにするためには、勝率が50%以上の戦略を見つけだしゲームを行うか、カウンティングの手法を改変して利益を出すようにしなければならない。

そこで今後の課題として、遺伝的アルゴリズムにエラー率の概念を導入し、より勝率の高い戦略を見つけ出すことと、カウンティングの手法の中で優秀なものを、今回使用した遺伝的アルゴリズム、もしくは他の方法を用いて探索する事が挙げられた。

また、今回の複雑性の検証実験により、ブラックジャックをプレイする際、戦略表を記憶する上で、一般的な認知的判断能力は関係がないことがわかった。
つまり、別の指標を用いて、戦略表を記憶する能力を測ることができる可能性があるということである。
その能力とは何であるのかを調査することが2つ目の課題として挙げられた。

\bunseki{菱田美紗紀}
