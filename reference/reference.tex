\begin{thebibliography}{9}
  \bibitem{blakjack2} Baldwin,R. and W. Cantey and H. Maisel and J. McDermott(1956) "The Optimum Strategy in Blackjack", {\it{Journal of the American Statistical Association}},vol.51, no.275, 419-439.
  \bibitem{complexity} Chaitin,G.J. (1969)"On the Simplicity and Speed of Programs for Computing Infinite Sets of Natural Numbers", {\it{Journal of the Association for Computing Machinery}}, vol.16, no.3, 407-422.
  \bibitem{pythonDocument} Python Software Foundation(2018), "Python 3.6.5 ドキュメント", \verb|<|https://docs.python.jp/3/index.html\verb|>| 2018年7月1日アクセス.
  \bibitem{pythonrandom} Python Software Foundation(2018) "9.6. random - 擬似乱数の生成" \verb|<|https://docs.python.jp/3/ library/random.html\verb|>| 2018年6月20日アクセス
  \bibitem{basicstrategy} Thorp,E.(1962) {\it{Beat The Dealer:A Winning Strategy for the Game of Twenty One}},Vintage (宮崎三瑛(2006)『ディーラーをやっつけろ!』,パンローリング)
  \bibitem{pairwise} 青木繁伸(2010) "比率の差の多重比較(pairwise.prop.test の拡張)" \verb|<|http://aoki2.si.gunma-u.ac.jp/R/p\_multi\_comp2.html\verb|>| 2018年7月21日アクセス
  \bibitem{neuro2} 斎藤康毅 (2016) 『ゼロから作るDeep Learning Pythonで学ぶディープラーニングの理論と実装』森北出版株式会社
  \bibitem{blackjack1} 斎藤隆浩 (1999) 『新訂ブラックジャック必勝法』株式会社データハウス
 \bibitem{gaimprove} 佐藤浩,小野功,小林重信(1997) 『遺伝的アルゴリズムにおける世代交代モデルの提案と評価』,人工知能学会誌,12,5,pp.734-744
  \bibitem{zansa} 全人類がわかる統計学(2017) "カイ二乗検定を残差分析で評価する方法" \verb|<|https://to-kei.net/hypothesis-testing/chi2-test-residual-analysis/\verb|>| 2018年7月6日アクセス
  \bibitem{rway} 竹澤邦夫(2012) "19. 行列の作成" \verb|<|http://cse.naro.affrc.go.jp/takezawa/r-tips/r/19.html\verb|>| 2018年6月20日アクセス
  \bibitem{neuro1} 萩原将文 (1994) 『ニューロ・ファジィ・遺伝的アルゴリズム』産業図書株式会社
  \bibitem{random}広井誠(2007)"Algorithms with Python - 番外編:擬似乱数の検定" \verb|<|http://www.geocities.jp/m\_hiroi/light/pystat04.html\verb|>| 2018年6月20日アクセス
  \bibitem{mersenne} 松本眞(2013) "Mersenne Twister Home Page" \verb|<|http://www.math.sci.hiroshima-u.ac.jp/~m-mat/MT/what-is-mt.html\verb|>| 2018年6月20日アクセス
  \bibitem{statistics1} 山内光哉(1987) 『心理・教育のための統計法』株式会社サイエンス社
\end{thebibliography}
