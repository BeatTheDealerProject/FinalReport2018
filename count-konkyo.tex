\section{カウンティングの根拠}
\ \ カウンティングが有効であるという理由は2つある。1つ目に、プレイヤーは行動を選択でき、ディーラーは16以下の時ヒットをしなければならないというルールによるものである。例えば、デックの残りが全て8であり、プレイヤーはカウンティングでその状況を把握しているとする。この時、プレイヤーのハンドは8が2枚の16であり、ディーラーのアップカードも8となる。プレイヤーはここでヒットをすると8を引いてバーストになってしまうことが分かっているのでスタンドを選択する。一方、ディーラーのホールカードも8であるため、ディーラーのハンドも8が2枚の16である。しかしディーラーは16以下のためヒットを選択する。すると8を引くので24となりディーラーはバーストになってしまう。これは極端な例だがこの場合プレイヤーの勝利が確定している。なので行動を変えられるプレイヤーにとってカウンティングは有効である。
\ \ 2つ目に、ブラックジャックにはプレイヤーに有利なカード、ディーラーに有利なカード、プレイヤーに不利なカード、ディーラーに不利なカードが存在するという点である。例えば、10はプレイヤーに有利でディーラーに不利なカードである。ここで、10以外のカードが多く使用されデックに10が多く残っている状況を想定する。この時、プレイヤーは11以下の時ダブルダウンを行うと10を引く確率が高いので20や21など21に近く強いハンドになり勝率が上がる。また、プレイヤーは行動を変えられるので12以上の時はスタンドをすることでバーストを避けることができる。さらに、ナチュラルブラックジャックの確率も上がるため利得が高くなる。反対に、ディーラーはダブルダウンの選択肢がないので利得は上がらない。また、ディーラーは16以下の時ヒットをするというルールがあるため、12,13,14,15,16の時もヒットをするのでバースト率が上がる。さらにディーラーはナチュラルブラックジャックであっても配当が1.5倍にはならないため、利得が上がらない。以上のことから、10以外のカードが多く使用されデックに10が多く残っていると、プレイヤーの利得が上がりディーラーのバースト率が上がるため10はプレイヤーに有利でディーラーに不利なカードであるということが言える。次に、10が多く使用され2,3,4がデックに多く残っている状況を想定する。この時、プレイヤーはダブルダウンをすると13や14など21から遠く弱いハンドになるため勝率が下がる。また、ベーシックストラテジーではプレイヤーのハンドが17の時にはディーラーのハンドに関わらず全ての場合でスタンドとなっているのでプレイヤーはスタンドを選択するが、この場合ではヒットをしてもバースト率が低いためスタンドによって勝率を下げている。さらに、ナチュラルブラックジャックの確率が低いため利得が下がる。反対に、ディーラーは元々ダブルダウンをしないため利得が下がらない。また、バースト率が下がるためディーラーの勝率が上がる。さらにディーラーはナチュラルブラックジャックであっても配当が1.5倍にはならないため、利得が上がらない。
\ \ 以上のことから、2,3,4はプレイヤーに不利でディーラーに有利なカードであるということが言える。また、ブラックジャックにはプレイヤーに有利なカード、ディーラーに有利なカード、プレイヤーに不利なカード、ディーラーに不利なカードが存在するということがいえる。この2つの理由により、カウンティングは有効であるということが言える。
\bunseki{※轟木文弥}

\subsection{カードの重要度の検証}
ブラックジャックにはプレイヤーに有利なカード、ディーラーに有利なカード、プレイヤーに不利なカード、ディーラーに不利なカードが存在する。そこで本プロジェクトは、どのカードがどれほどプレイヤーにとって有利なカードかを検証した。検証方法として、普通のデ1ック、1デックから1種類カードを抜いたデックをそれぞれ用意し、シミュレーションした結果を比較した。比較対象となるデックは以下の通りである。

\begin{itemize}
  \item 普通の1デック
  \item Aを抜いたデック
  \item 2を抜いたデック
  \item 3を抜いたデック
  \item 4を抜いたデック
  \item 5を抜いたデック
  \item 6を抜いたデック
  \item 7を抜いたデック
  \item 8を抜いたデック
  \item 9を抜いたデック
  \item 10を抜いたデック
\end{itemize}

\subsection{仮説}
10,Aが残っていると勝率が上がる。逆に2,3,4が残っていると勝率が下がる。5,6,7,8,9は影響がないという仮説を設定する。これは次の3つに注目したからである。1つ目に、数値の小さいカードはダブルダウンの勝率が下がり、ディーラーのバースト率が下がることからプレイヤーにとって不利である。2つ目に、10はダブルダウンの勝率が上がり、ディーラーのバースト率が上がることからプレイヤーに有利である。3つ目に、Aはプレイヤーの方が1と11を自由に選択でき、ナチュラルブラックジャックの可能性もあるため有利である。この3つの考えから、仮説を設定した。
\bunseki{※轟木文弥}

\section{シミュレーション}
設定した仮説をそれぞれの数字をデックから抜いた場合のシミュレータを回して勝率を確認した。これによりカードの重要度を測った。デック数は6とした。戦略はダブルダウン、スプリット、サレンダーを含むベーシックストラテジーを使用した。シミュレーション回数は10万回とした。ただし、今回は賭け金を導入しなかったためダブルダウンは利得が2倍になるという点を勝利数2とすることで表現した。シミュレーション結果を\ref{pullcard}に示す。

\begin{figure}[H]
 \begin{center} 
  \includegraphics[width=0.7\linewidth]{./figure/pullcard}
  \caption{カードを抜いたデックの勝率\label{pullcard}}
 \end{center}
\end{figure}

8,9,10,Aを抜いたデックは普通のデックより勝率が下がり、2,3,4,5,6,7を抜いたデックは勝率が上がった。
得られたデータをカイ2乗検定と多重比較を用いてどこに差が存在するかを検証した。
結果として、5,6,10,Aには有意な差が存在することが確認された。
つまり、5,6はプレイヤーに不利なカードであり、10,Aはプレイヤーに有利なカードであるという結果が得られた。
\bunseki{※轟木文弥}

