\section{ブラックジャックのルール}

トランプの扱いについて
\begin{itemize}
\item ジョーカーはゲームでは使用しない
\item 52枚を1デックとして使用する
\item 2~10のカードは書いてある数字の通りに扱う
\item J・Q・Kは10として扱う
\item Aは11または1で都合の良いほうとして扱う
\end{itemize}
勝敗の決定について
\begin{itemize}
\item 手札の合計値が21以下で、合計値が大きい方の勝利
\item ディーラーとプレイヤーの手札の合計値が同じ場合は引き分けとなる
\item 手札の合計値が21を超えることをバーストという
\item バーストしたプレイヤーはその時点で負けとなる
\item ディーラーがバーストした場合はすべてのプレイヤーが勝ちとなる
\item プレイヤーとディーラーの両方がバーストしている場合、ディーラーの勝ちとなる
\item 最初の手札の合計値が21の場合ブラックジャックといい、最も強い手札となる
\end{itemize}
賭け金の扱い
\begin{itemize}
\item プレイヤーが勝つと賭け金の2倍が払い戻される
\item プレイヤーがブラックジャックで勝つと賭け金の2.5倍が払い戻される
\item ディーラーが勝つと賭け金を没収される
\item 引き分けの場合賭け金は賭けたプレイヤーに払い戻される
\end{itemize}
プレイヤーの選択肢
\begin{itemize}
\item ヒット:山札からカードを1枚手札に追加すること21を超えない限り、何度でもできる。
\item スタンド:カードを引かずに今の手札で勝負すること。
\item サレンダー:負けを認めることで、賭け金の半分をもらうことができる。最初の行動でのみ使える。
\item ダブルダウン:賭け金を2倍にして1度だけヒットをし、その後スタンドする。最初の行動でのみ使える。
\item スプリット:最初に配られた2枚のカードが同じ数字だった場合使用可能。最初の賭け金と同じ金額を追加して、それらを2つに分割して、それぞれで勝負することができる。
\item インシュランス:ディーラーの表向きのカード(アップカード)が「A」の場合使える。最初の賭け金の半分を使い、ディーラーがナチュラルブラックジャックになればその賭け金の2倍が払い戻される。自分の手札がブラックジャックである場合に行うインシュランスのことをイーブンマネーと呼ぶ。
\end{itemize}
ただし前期の活動ではプレイヤーの行動はヒットまたはスタンドに限定している。
\bunseki{柏田輝}

\subsection{ゲームの流れ}
まずデックをシャッフルし、カットカードと呼ばれるカードをランダムにデックに入れる。カットカードが近づくとデックがシャッフルされる。ゲーム開始時に各プレイヤーは賭け金をテーブルに置く。その後ディーラーは自身と全てのプレイヤーにカードを2枚ずつ配る。この時ディーラーのカードは1枚を裏向きに、もう1枚を表向きにして配る。ディーラーの表向きのカードをアップカードという。カードを配り終えるとプレイヤーの行動に移る。プレイヤーがスタンドもしくはバーストするかイーブンマネーだった場合、プレイヤーの行動は終了となる。全てのプレイヤーの行動が終了すると、ディーラーの行動に移る。ディーラーがスタンドもしくはバーストした場合、ディーラーの行動は終了となる。その後ディーラーは各プレイヤーと勝敗を確認し、それに応じた支払いが行われてゲームが終了する。
\bunseki{柏田輝}

\subsection{ディーラーの行動}
ディーラーの行動は常に一定となっている。ディーラーは手札の合計が17以上になるまでヒットを続けなければならないとルールで定められている。そのためディーラーの手札の合計値は最終的に17,18,19,20,21,バーストのいずれかとなる。
\bunseki{柏田輝}

