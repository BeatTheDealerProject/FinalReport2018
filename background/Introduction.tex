
\section{ブラックジャックの戦略の歴史}
斎藤(1999)によれば、ブラックジャックの戦略については1950年にメリーランド州のとある米国陸軍の研究所に所属していたRoger、Nash、Baldwinらが研究したものが始まりであるといわれている。その後計算機の性能向上により、ブラックジャックのシミュレーションが容易になったことでさらに戦略の研究は進んでいった。
ブラックジャックには主に有名な戦略が2つ存在する。

1つ目がベーシックストラテジーと呼ばれる戦略だ。ベーシックストラテジーはディーラーのアップカードと自分の手札の状況によってプレイヤーが選択するべき最適な行動を決定するという戦略である。

2つ目はカウンティングと呼ばれる戦略だ。カウンティングはブラックジャックのゲーム中で既に使われたカードを記憶することで、プレイヤーが有利になるように自分がとるべき戦略を決定していくというものである。

前期ではベーシックストラテジーについて調査、検証をした。後期ではカウンティングについて詳しく調査し、シミュレーションを行った。なおカウンティングについてはカジノ側に対策をされており、この戦略を使用していることがカジノ側に気づかれた場合、プレイヤーはカジノから退場させられることもあるなど、現在カウンティングをカジノで扱うことは難しくなっている。
\bunseki{伊藤晋之介}