%ソースが不明なのでとりあえず暫定削除
%\section{ブラックジャックの概要と歴史}
%ブラックジャックのルーツは1570年にさかのぼる。このころはまだ「ブラックジャック」という名称は使われていなかった。1875年に出版された「The American Hoyle of 1875」という書籍で「ブラックジャック」として紹介されたのが初出である。このゲームが考え出された当初は金銭を賭けることはなく、身内内で楽しむだけのゲームであった。
%はじめて金銭が賭けられるようになったのは1910年頃のアメリカ、インディアナ州であるといわれている。当時は競馬以外のギャンブルが違法であり、ブラックジャックも違法カジノでのみ行われていた。その後インディアナ州以外のアメリカ全土に広がっていった。そして現在世界中のカジノで合法的に楽しまれるポピュラーなゲームになった。
%\bunseki{※菱田美紗紀}

\section{ブラックジャックの戦略の歴史}
斎藤(1999)によれば、ブラックジャックの戦略については1950年にメリーランド州のとある米国陸軍の研究所に所属していたRoger、Nash、Baldwinらが研究したものが始まりであるといわれている。その後計算機の性能向上より、ブラックジャックのシミュレーションが容易になったことでさらに戦略の研究は進んでいった。
ブラックジャックには主に有名な戦略が2つ存在する。

1つ目がベーシックストラテジーと呼ばれる戦略だ。ベーシックストラテジーはディーラーのアップカードと自分の手札の状況によってプレイヤーが選択するべき最適な行動を決定するという戦略である。

2つ目はカウンティングと呼ばれる戦略だ。カウンティングはブラックジャックのゲーム中で既に使われたカードを記憶することで、プレイヤーが有利になるように戦略を決定していくというものである。

前期ではベーシックストラテジーについて調査をした。カウンティングについては後期に詳しく調査していく予定である。
なお、カウンティングはカジノ側に対策をされており、この戦略を使用していることがカジノ側に気づかれた場合、プレイヤーはカジノから退場させられることもある。
\bunseki{※伊藤晋之介}