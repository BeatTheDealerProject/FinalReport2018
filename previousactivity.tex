本プロジェクトで題材としているブラックジャックというゲームを
実際のプレイも交えて学習した。前章までで説明したベーシックストラテジーを
試すということもその中で行ったが、実際に使用してみると表を覚え、かつ
素早いゲームの進行に合わせながら実行するのは容易ではないという実感が得られた。
もちろんディーラー側に戦略の実行をさとられないようにするためには、ゲームの進行を
止めるなど違和感を持たせるような行動はできるだけ無くす必要がある。以上のことから、
戦略の単純化の必要性を再確認した。

シミュレータはPython3で作成を行った。前期までで得られたブラックジャックの戦略比較に用いた
数値はこのシミュレータによって得られた。また開発の効率化のためにバージョン管理システムである
Gitを導入し、Gitによるバージョン管理について学習した。導入の理由としては、プロジェクト学習で
プログラム開発を行う場合複数人で作業を行うためである。複数人の作業では、最新版のソースコードを
作業者同士で共有する必要がある。単純にプログラムファイル群をひとつにまとめて送付し合うという
方式もできるが、これではどの部分が変更されたのかの確認を行いにくい。Gitというシステムでは、誰が
どのようなタイミングでどのような変更を行ったかが記録されるため、エラーの解消を効率的に行うことができる。

また、中間報告書の作成作業にもGitを用いた。中間報告書の記述では、プログラムのソースコード以上に
文責者を明確にすることがプロジェクト学習として要求されている。また、報告書の記述はTeXであり、テキストファイル
であるため、Gitで管理することができる。単純にファイルをまとめて送付する場合よりも円滑な管理が行えたと思う。

以上のように、導入したバージョン管理システムGitはプロジェクト学習の多くの作業で有効だった。

シミュレータの正しさ、シミュレーション結果の分析のために統計学を学習した。

本プロジェクトにおいて、戦略の複雑性を評価することはとても重要な事項である。
複雑性の定義付けのためにChaitin(1969)によって定義されたコルモゴロフ複雑性の定義を参考にし、コルモゴロフ複雑性の定義と使われ方について調査し学習した。

本プロジェクトでは最適な戦略の探索を行うための技術の一つとしてニューラルネットワークを挙げ、斎藤(2016) の書籍を教科書とし、勉強会を行った。

\bunseki{米村祥裕}

