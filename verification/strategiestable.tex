\section{以降のシミュレーションに用いる戦略表について}
以下のシミュレーションを行うにあたって、BS、BS-HS、GA戦略の3つの戦略を用意した。

これから、戦略の内BSについて説明する。BS-HS戦略とGA戦略については前述している為簡単に説明する。

表の文字はそれぞれ以下のように対応している。

(H:ヒット, \ S:スタンド, \ D:ダブルダウン, \ P:スプリット)  


\bunseki{※渡邊凛}


\subsection{表の見方について}

表の文字はそれぞれ以下のように対応している。

H:ヒット, \ S:スタンド, \ D:ダブルダウン, \ P:スプリット
\\

ルールの項でも説明しているが、改めて説明する。

ヒットは山札からカードを1枚手札に追加することである。21を超えない限り、何度でもヒットすることができる。

スタンドはカードを引かずに自分の行動を終了することである。

スプリットは最初に配られた2枚のカードが同じ数字だった場合、最初の賭け金と同じ金額を追加して、それらのカードを2つに分割し、それぞれの手札で勝負するルールである。

ダブルダウンは賭け金を2倍にして1度のみヒットをし、その後強制的にスタンドするルールである。

\bunseki{※渡邊凛}


\subsection{BSについて}
この戦略は、Thorp(1962)の1デック用のベーシックストラテジーではなく、Jensen(2014)の4~8デック用のものを参考にしたものである。
この戦略表では、スプリット、ダブルダウンが追加されている。

これらの選択肢を戦略に追加することにより、同じ賭け金でもより期待値が高くなる。
ただし、選択肢が増える分表も大きくなり複雑性が増す為に人間にとって扱いづらく、誤る可能性が高いと言える。

\begin{table}[H]
    \begin{center}
    \caption{BS}
    \begin{tabular}{|c|c|c|c|c|c|c|c|c|c|c|c|c|}
    \hline
    \multicolumn{3}{|c|}{\multirow{2}{*}{}}                     & \multicolumn{10}{c|}{ディーラーのアップカード}     \\ \cline{4-13} 
    \multicolumn{3}{|c|}{}                                      & 2 & 3 & 4 & 5 & 6 & 7 & 8 & 9 & 10 & A \\ \hline
    \multirow{28}{*}{手札の合計} & \multirow{9}{*}{ハードハンド}   & 9     & H & H & H & H & H & H & H & H & H  & H \\ \cline{3-13} 
                            &                           & 10    & H & D & D & D & D & H & H & H & H  & H \\ \cline{3-13} 
                            &                           & 11    & D & D & D & D & D & D & D & D & H  & H \\ \cline{3-13} 
                            &                           & 12    & D & D & D & D & D & D & D & D & D  & D \\ \cline{3-13} 
                            &                           & 13    & H & H & S & S & S & H & H & H & H  & H \\ \cline{3-13} 
                            &                           & 14    & S & S & S & S & S & H & H & H & H  & H \\ \cline{3-13} 
                            &                           & 15    & S & S & S & S & S & H & H & H & H  & H \\ \cline{3-13} 
                            &                           & 16    & S & S & S & S & S & H & H & H & H  & H \\ \cline{3-13} 
                            &                           & 17以上  & S & S & S & S & S & S & S & S & S  & S \\ \cline{2-13} 
                            & \multirow{9}{*}{ソフトハンド}   & A,2   & H & H & H & D & D & H & H & H & H  & H \\ \cline{3-13} 
                            &                           & A,3   & H & H & H & D & D & H & H & H & H  & H \\ \cline{3-13} 
                            &                           & A,4   & H & H & D & D & D & H & H & H & H  & H \\ \cline{3-13} 
                            &                           & A,5   & H & H & D & D & D & H & H & H & H  & H \\ \cline{3-13} 
                            &                           & A,6   & H & D & D & D & D & H & H & H & H  & H \\ \cline{3-13} 
                            &                           & A,7   & S & D & D & D & D & S & S & H & H  & H \\ \cline{3-13} 
                            &                           & A,8   & S & S & S & S & S & S & S & S & S  & S \\ \cline{3-13} 
                            &                           & A,9   & S & S & S & S & S & S & S & S & S  & S \\ \cline{3-13} 
                            &                           & A,10  & S & S & S & S & S & S & S & S & S  & S \\ \cline{2-13} 
                            & \multirow{10}{*}{スプリット可能} & A,A   & P & P & P & P & P & P & P & P & P  & P \\ \cline{3-13} 
                            &                           & 2,2   & P & P & P & P & P & P & H & H & H  & H \\ \cline{3-13} 
                            &                           & 3,3   & P & P & P & P & P & P & H & H & H  & H \\ \cline{3-13} 
                            &                           & 4,4   & H & H & H & P & P & H & H & H & H  & H \\ \cline{3-13} 
                            &                           & 5,5   & D & D & D & D & D & D & D & D & H  & H \\ \cline{3-13} 
                            &                           & 6,6   & P & P & P & P & P & H & H & H & H  & H \\ \cline{3-13} 
                            &                           & 7,7   & P & P & P & P & P & P & H & H & H  & H \\ \cline{3-13} 
                            &                           & 8,8   & P & P & P & P & P & P & P & P & P  & P \\ \cline{3-13} 
                            &                           & 9,9   & P & P & P & P & P & S & P & P & S  & S \\ \cline{3-13} 
                            &                           & 10,10 & S & S & S & S & S & S & S & S & S  & S \\ \hline
    \end{tabular}
    \end{center}
\end{table}


\bunseki{※渡邊凛}


\subsection{BS-HSとGA戦略について}

BS-HSはヒットとスタンドのみのベーシックストラテジーであり、表\ref{kihonsenryaku}をそのまま使用している。GA戦略は遺伝的アルゴリズムによって生成された戦略表であり、表\ref{gastrategyhard}、\ref{gastrategysoft}をそのまま使用している。これらの戦略にはダブルダウンやスプリットはなく、ヒットとスタンドのみである。そのため、戦略としては単純になり、人間にとって扱いやすい戦略表となる。

\bunseki{※渡邊凛}