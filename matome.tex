\section{まとめ}
まず、本プロジェクトは、複雑性の実験により次の結果が得られた。
\begin{itemize}
    \item 結果1:複雑性とテストの成績には強い負の相関がある
    \item 結果2:CRTの正答率とテストの成績には相関がない
    \item 結果3:リスク回避性とテストの成績には相関がない
\end{itemize}

次に、ブラックジャックのシミュレータを作成した。作成にあたって、使用した疑似乱数を生成するメルセンヌツイスタが適切かどうか検証した。結果として、メルセンヌツイスタは周期が大きく、全て等しい確率で乱数生成していることが確認された。

次に、ベーシックストラテジーについて比較対象となる6つの戦略を考案し、検証を行った。これにより次の結果が得られた。
勝率のみを考慮した場合
\begin{itemize}
\item 結果1:一定の数字以上になるあでヒットする戦略よりも、ベーシックストラテジーとベーシックストラテジー改変1、ベーシックストラテジー改変2のほうが有意に高い勝率だった
\item 結果2:ベーシックストラテジーとベーシックストラテジー改変1、ベーシックストラテジー改変2のそれぞれの戦略間に有意な差はみられなかった
\item 結果3:ベーシックストラテジー改変1と18以上までヒットする戦略にはデック数無限とデック数1で勝率に有意な差があった
\end{itemize}

勝率を考慮すると、ベーシックストラテジーとベーシックストラテジー改変1、ベーシックストラテジー改変2の3つが勝率が高く、優秀な戦略であることが分かった。ただ、ベーシックストラテジー改変1はデック数の違いによって勝率に差がある。

複雑性を考慮して性能を評価した場合
\begin{itemize}
\item 結果4:15以上になるまでヒットする戦略が1番優秀であることが判明した。
\end{itemize}

複雑性を考慮すると、ベーシックストラテジーには改善の余地があることが判明した。

次に、遺伝的アルゴリズムを用いて新しい戦略を得た。得られた戦略とベーシックストラテジー、ダブルダウンやスプリットを含めたベーシックストラテジーと勝率を比較した。複雑性を考慮して新しい戦略を作成したが、比較の際には複雑性を考慮しなかったため、最も成績が悪いという結果となった。

次に、賭け金を導入してベーシックストラテジー、ダブルダウンやスプリットを含めたベーシックストラテジー、遺伝的アルゴリズムでえられた戦略で検証を行った。まずは賭け金を一定にした場合の所持金を調査した。結果として全ての戦略が初期所持金を下回った。次にカウンティングを用いることで利得を増やすことを試みた。結果として、既存のカウンティング手法を使うことで初期所持金を上回ることに成功した。次に新しいカウンティング手法を考案し検証した。結果として既存の手法より良い結果を得られることができなかった。

次に、複雑性を考慮してシミュレーションをするために、複雑性の実験データをもとにエラー率という概念を作成した。エラー率を適用してシミュレーションを行った結果、本プロジェクトが作成した、遺伝的アルゴリズムを用いて得られた戦略が最も優秀であるという結果が得られた。
\bunseki{轟木文弥}
